\pdfoutput=1
%The other issue is that some packages, such as microtype, produce different output under pdflatex. By default the arXiv goes from dvi to ps to pdf, so if you need pdflatex you have to set the \pdfoutput flag in the TeX file.
\newif\ifpersonal
\personaltrue % comment to remove personal notes
\RequirePackage[l2tabu,orthodox]{nag} %detect whether obsolete packages are used
\documentclass[10pt,a4paper,reqno]{amsart} %reqno places equation numbers on the right
\linespread{1.1}
\usepackage{amsmath,amsthm,amssymb,mathrsfs,mathtools,bm,eucal,tensor} % math related
\usepackage{microtype,fixltx2e,lmodern} % latex technical issues
\usepackage[utf8]{inputenc} % input encoding
\usepackage[T1]{fontenc} % font encoding
\usepackage{enumerate,comment,braket,xspace,tikz-cd,csquotes} % utilities
\usepackage[all,cmtip]{xy} % because the tikzcd options [shift left], [shift right] do not work on arXiv, we switched some diagrams to xymatrix
\usepackage[centering,vscale=0.7,hscale=0.8]{geometry}
%\usepackage[right]{showlabels}
\usepackage[hidelinks]{hyperref}
\usepackage[capitalize]{cleveref}

\theoremstyle{plain}
\newtheorem{thm-intro}{Theorem}
\newtheorem{thm}{Theorem}[section]
\newtheorem*{thm*}{Theorem}
\newtheorem{claim}[thm]{Claim}
\newtheorem{lem}[thm]{Lemma}
\newtheorem{prop}[thm]{Proposition}
\newtheorem{conj}[thm]{Conjecture}
\newtheorem{cor}[thm]{Corollary}
\newtheorem{assumption}[thm]{Assumption}
\theoremstyle{definition}
\newtheorem{defin}[thm]{Definition}
\newtheorem{notation}[thm]{Notation}
\newtheorem{eg}[thm]{Example}
\newtheorem{variant}[thm]{Variant}
\newtheorem{warning}[thm]{Warning}
\theoremstyle{remark}
\newtheorem{rem}[thm]{Remark}
\numberwithin{equation}{section}
\newtheorem{construction}[thm]{Construction}

% personal remarks

\ifpersonal
\newcommand{\personal}[1]{\textcolor[rgb]{0,0,1}{(Personal: #1)}}
\newcommand{\todo}[1]{\textcolor{red}{(Todo: #1)}}
\else
\newcommand*{\personal}[1]{\ignorespaces}
\newcommand*{\todo}[1]{\ignorespaces}
\fi

% Fonts
\newcommand{\C}{\mathbb C}
\newcommand{\CP}{\mathbb{CP}}
\newcommand{\F}{\mathbb F}
\newcommand{\Q}{\mathbb Q}
\newcommand{\R}{\mathbb R}
\newcommand{\Z}{\mathbb Z}

\newcommand{\rB}{\mathrm B}
\newcommand{\rH}{\mathrm H}
\newcommand{\rL}{\mathrm L}
\newcommand{\rR}{\mathrm R}
\newcommand{\rT}{\mathrm T}
\newcommand{\rb}{\mathrm b}
\newcommand{\rd}{\mathrm d}
\newcommand{\rI}{\mathrm I}
\newcommand{\rs}{\mathrm s}
\newcommand{\rt}{\mathrm t}


\newcommand{\fA}{\mathfrak A}
\newcommand{\fB}{\mathfrak B}
\newcommand{\fC}{\mathfrak C}
\newcommand{\fD}{\mathfrak D}
\newcommand{\fF}{\mathfrak F}
\newcommand{\fG}{\mathfrak G}
\newcommand{\fH}{\mathfrak H}
\newcommand{\fS}{\mathfrak S}
\newcommand{\fT}{\mathfrak T}
\newcommand{\fU}{\mathfrak U}
\newcommand{\fV}{\mathfrak V}
\newcommand{\fX}{\mathfrak X}
\newcommand{\fY}{\mathfrak Y}
\newcommand{\fW}{\mathfrak W}
\newcommand{\fZ}{\mathfrak Z}
\newcommand{\fa}{\mathfrak a}
\newcommand{\fb}{\mathfrak b}
\newcommand{\ff}{\mathfrak f}
\newcommand{\fm}{\mathfrak m}
\newcommand{\fs}{\mathfrak s}
\newcommand{\ft}{\mathfrak t}

\newcommand{\cA}{\mathcal A}
\newcommand{\cB}{\mathcal B}
\newcommand{\cC}{\mathcal C}
\newcommand{\cD}{\mathcal D}
\newcommand{\cE}{\mathcal E}
\newcommand{\cF}{\mathcal F}
\newcommand{\cH}{\mathcal H}
\newcommand{\cG}{\mathcal G}
\newcommand{\cI}{\mathcal I}
\newcommand{\cJ}{\mathcal J}
\newcommand{\cK}{\mathcal K}
\newcommand{\cL}{\mathcal L}
\newcommand{\cM}{\mathcal M}
\newcommand{\cN}{\mathcal N}
\newcommand{\cO}{\mathcal O}
\newcommand{\cP}{\mathcal P}
\newcommand{\cQ}{\mathcal Q}
\newcommand{\cR}{\mathcal R}
\newcommand{\cS}{\mathcal S}
\newcommand{\cT}{\mathcal T}
\newcommand{\cU}{\mathcal U}
\newcommand{\cV}{\mathcal V}
\newcommand{\cW}{\mathcal W}
\newcommand{\cX}{\mathcal X}
\newcommand{\cY}{\mathcal Y}
\newcommand{\cZ}{\mathcal Z}
\DeclareFontFamily{U}{BOONDOX-calo}{\skewchar\font=45 }
\DeclareFontShape{U}{BOONDOX-calo}{m}{n}{<-> s*[1.05] BOONDOX-r-calo}{}
\DeclareFontShape{U}{BOONDOX-calo}{b}{n}{<-> s*[1.05] BOONDOX-b-calo}{}
\DeclareMathAlphabet{\mathcalboondox}{U}{BOONDOX-calo}{m}{n}
%\DeclareMathAlphabet{\mathcalligra}{T1}{calligra}{m}{n}
\newcommand{\cf}{\mathcalboondox f}

\newcommand{\bbA}{\mathbb A}
\newcommand{\bbD}{\mathbb D}
\newcommand{\bbE}{\mathbb E}
\newcommand{\bbG}{\mathbb G}
\newcommand{\bbL}{\mathbb L}
\newcommand{\bbN}{\mathbb N}
\newcommand{\bbP}{\mathbb P}
\newcommand{\bbT}{\mathbb T}
\newcommand{\bbZ}{\mathbb Z}

\newcommand{\bA}{\mathbf A}
\newcommand{\bD}{\mathbf D}
\newcommand{\bP}{\mathbf P}
\newcommand{\bQ}{\mathbf Q}
\newcommand{\bT}{\mathbf T}
\newcommand{\bX}{\mathbf X}
\newcommand{\bY}{\mathbf Y}
\newcommand{\be}{\mathbf e}
\newcommand{\br}{\mathbf r}
\newcommand{\bu}{\mathbf u}
\newcommand{\balpha}{\bm{\alpha}}
\newcommand{\bDelta}{\bm{\Delta}}
\newcommand{\brho}{\bm{\rho}}

\newcommand{\sC}{\mathscr C}
\newcommand{\sX}{\mathscr X}
\newcommand{\sD}{\mathscr D}
\newcommand{\sU}{\mathscr U}


% Decorations

% Definition of \widebar from http://tex.stackexchange.com/questions/16337/can-i-get-a-widebar-without-using-the-mathabx-package/60253#60253
\makeatletter
\let\save@mathaccent\mathaccent
\newcommand*\if@single[3]{%
	\setbox0\hbox{${\mathaccent"0362{#1}}^H$}%
	\setbox2\hbox{${\mathaccent"0362{\kern0pt#1}}^H$}%
	\ifdim\ht0=\ht2 #3\else #2\fi
}
%The bar will be moved to the right by a half of \macc@kerna, which is computed by amsmath:
\newcommand*\rel@kern[1]{\kern#1\dimexpr\macc@kerna}
%If there's a superscript following the bar, then no negative kern may follow the bar;
%an additional {} makes sure that the superscript is high enough in this case:
\newcommand*\widebar[1]{\@ifnextchar^{{\wide@bar{#1}{0}}}{\wide@bar{#1}{1}}}
%Use a separate algorithm for single symbols:
\newcommand*\wide@bar[2]{\if@single{#1}{\wide@bar@{#1}{#2}{1}}{\wide@bar@{#1}{#2}{2}}}
\newcommand*\wide@bar@[3]{%
	\begingroup
	\def\mathaccent##1##2{%
		%Enable nesting of accents:
		\let\mathaccent\save@mathaccent
		%If there's more than a single symbol, use the first character instead (see below):
		\if#32 \let\macc@nucleus\first@char \fi
		%Determine the italic correction:
		\setbox\z@\hbox{$\macc@style{\macc@nucleus}_{}$}%
		\setbox\tw@\hbox{$\macc@style{\macc@nucleus}{}_{}$}%
		\dimen@\wd\tw@
		\advance\dimen@-\wd\z@
		%Now \dimen@ is the italic correction of the symbol.
		\divide\dimen@ 3
		\@tempdima\wd\tw@
		\advance\@tempdima-\scriptspace
		%Now \@tempdima is the width of the symbol.
		\divide\@tempdima 10
		\advance\dimen@-\@tempdima
		%Now \dimen@ = (italic correction / 3) - (Breite / 10)
		\ifdim\dimen@>\z@ \dimen@0pt\fi
		%The bar will be shortened in the case \dimen@<0 !
		\rel@kern{0.6}\kern-\dimen@
		\if#31
		\overline{\rel@kern{-0.6}\kern\dimen@\macc@nucleus\rel@kern{0.4}\kern\dimen@}%
		\advance\dimen@0.4\dimexpr\macc@kerna
		%Place the combined final kern (-\dimen@) if it is >0 or if a superscript follows:
		\let\final@kern#2%
		\ifdim\dimen@<\z@ \let\final@kern1\fi
		\if\final@kern1 \kern-\dimen@\fi
		\else
		\overline{\rel@kern{-0.6}\kern\dimen@#1}%
		\fi
	}%
	\macc@depth\@ne
	\let\math@bgroup\@empty \let\math@egroup\macc@set@skewchar
	\mathsurround\z@ \frozen@everymath{\mathgroup\macc@group\relax}%
	\macc@set@skewchar\relax
	\let\mathaccentV\macc@nested@a
	%The following initialises \macc@kerna and calls \mathaccent:
	\if#31
	\macc@nested@a\relax111{#1}%
	\else
	%If the argument consists of more than one symbol, and if the first token is
	%a letter, use that letter for the computations:
	\def\gobble@till@marker##1\endmarker{}%
	\futurelet\first@char\gobble@till@marker#1\endmarker
	\ifcat\noexpand\first@char A\else
	\def\first@char{}%
	\fi
	\macc@nested@a\relax111{\first@char}%
	\fi
	\endgroup
}
\makeatother


\newcommand{\oDelta}{\widebar\Delta}
\newcommand{\oGamma}{\widebar\Gamma}
\newcommand{\oSigma}{\widebar\Sigma}
\newcommand{\oalpha}{\widebar\alpha}
\newcommand{\obeta}{\widebar\beta}
\newcommand{\otau}{\widebar\tau}
\newcommand{\oC}{\widebar C}
\newcommand{\oD}{\widebar D}
\newcommand{\oE}{\widebar E}
\newcommand{\oG}{\widebar G}
\newcommand{\oM}{\widebar M}
\newcommand{\oS}{\widebar S}
\newcommand{\oU}{\widebar U}
\newcommand{\oW}{\widebar W}
\newcommand{\oX}{\widebar X}
\newcommand{\oY}{\widebar Y}


\newcommand{\ok}{\widebar k}
\newcommand{\ov}{\widebar v}
\newcommand{\ox}{\widebar x}
\newcommand{\oy}{\widebar y}
\newcommand{\oz}{\widebar z}

\newcommand{\hh}{\widehat h}
\newcommand{\hC}{\widehat C}
\newcommand{\hE}{\widehat E}
\newcommand{\hF}{\widehat F}
\newcommand{\hI}{\widehat I}
\newcommand{\hL}{\widehat L}
\newcommand{\hU}{\widehat U}
\newcommand{\hbeta}{\widehat\beta}
\newcommand{\hGamma}{\widehat\Gamma}


\newcommand{\ta}{\widetilde a}
\newcommand{\tb}{\widetilde b}
\newcommand{\tk}{\tilde k}
\newcommand{\tu}{\widetilde u}
\newcommand{\tv}{\tilde v}
\newcommand{\tw}{\widetilde w}
\newcommand{\tB}{\widetilde B}
\newcommand{\tD}{\widetilde D}
\newcommand{\tI}{\widetilde I}
\newcommand{\tP}{\widetilde P}
\newcommand{\tS}{\widetilde S}
\newcommand{\tU}{\widetilde U}
\newcommand{\tV}{\widetilde V}
\newcommand{\tW}{\widetilde W}
\newcommand{\tX}{\widetilde X}
\newcommand{\tfX}{\widetilde{\fX}}
\newcommand{\tfB}{\widetilde{\fB}}
\newcommand{\tsX}{\widetilde{\sX}}
\newcommand{\tH}{\widetilde H}
\newcommand{\tY}{\widetilde Y}
\newcommand{\talpha}{\widetilde{\alpha}}
\newcommand{\tbeta}{\widetilde{\beta}}
\newcommand{\tmu}{\widetilde{\mu}}
\newcommand{\tnu}{\widetilde{\nu}}
\newcommand{\tphi}{\widetilde{\phi}}
\newcommand{\ttau}{\widetilde{\tau}}


% Global tropicalization
\newcommand{\Ih}{I^\mathrm{h}}
\newcommand{\Iv}{I^\mathrm{v}}
\newcommand{\IX}{I_\fX}
\newcommand{\IY}{I_\fY}
\newcommand{\SD}{S_\fD}
\newcommand{\SX}{S_\fX}
\newcommand{\SsXH}{S_{(\sX,H)}}
\newcommand{\SY}{S_\fY}
\newcommand{\CsXH}{C_{(\sX,H)}}
\newcommand{\oSX}{\overline{\SX}}
\newcommand{\oIX}{\overline{\IX}}


% Vanishing cycles
\newcommand{\fXe}{\fX_\eta}
\newcommand{\fXs}{\fX_s}
\newcommand{\ofX}{\widebar{\fX}}
\newcommand{\ofXs}{\widebar{\fX}_s}
\newcommand{\fYe}{\fY_\eta}
\newcommand{\fYs}{\fY_s}
\newcommand{\fXbs}{\fX_{\bar s}}
\newcommand{\fXbe}{\fX_{\bar\eta}}
\newcommand{\fDe}{\fD_\eta}
\newcommand{\LX}{\Lambda_{\fX}}
\newcommand{\LXe}{\Lambda_{\fX_\eta}}
\newcommand{\LXs}{\Lambda_{\fXbs}}
\newcommand{\QXe}{\Q_{\ell,\fX_\eta}}
\newcommand{\QXbs}{\Q_{\ell,\fXbs}}
\newcommand{\sXe}{\sX_\eta}
\newcommand{\sXs}{\sX_s}
\newcommand{\LUe}{\Lambda_{\fU_\eta}}
\newcommand{\fCbs}{\fC_{\bar s}}
\newcommand{\QUe}{\Q_{\ell,\fU_\eta}}
\newcommand{\QCe}{\Q_{\ell,\fC_\eta}}
\newcommand{\QCs}{\Q_{\ell,\fCbs}}

% stacks

\newcommand{\hcC}{\mathrm h\cC}
\newcommand{\hcD}{\mathrm h\cD}
\newcommand{\PSh}{\mathrm{PSh}}
\newcommand{\Sh}{\mathrm{Sh}}
\newcommand{\Shv}{\mathrm{Shv}}
\newcommand{\Tuupperp}{\tensor*[^\cT]{u}{^p}}
\newcommand{\Tulowerp}{\tensor*[^\cT]{u}{_p}}
\newcommand{\Tuuppers}{\tensor*[^\cT]{u}{^s}}
\newcommand{\Tulowers}{\tensor*[^\cT]{u}{_s}}
\newcommand{\pu}{\tensor*[_p]{u}{}}
\newcommand{\su}{\tensor*[_s]{u}{}}
\newcommand{\Tpu}{\tensor*[^\cT_p]{u}{}}
\newcommand{\Tsu}{\tensor*[^\cT_s]{u}{}}
\newcommand{\Dfpull}{\tensor*[^\cD]{f}{^{-1}}}
\newcommand{\Dfpush}{\tensor*[^\cD]{f}{_*}}
\newcommand{\Duuppers}{\tensor*[^\cD]{u}{^s}}
\newcommand{\Dulowers}{\tensor*[^\cD]{u}{_s}}
\newcommand{\Geom}{\mathrm{Geom}}
\newcommand{\LPr}{\mathcal{P}\mathrm{r}^\rL}
\newcommand{\RPr}{\mathcal{P}\mathrm{r}^\rR}
\newcommand{\CX}{\cC_{/X}}
\newcommand{\CY}{\cC_{/Y}}
\newcommand{\CXP}{(\cC_{/X})_{\bP}}
\newcommand{\GeomXP}{(\mathrm{Geom}_{/X})_\bP}
\newcommand{\GeomYP}{(\mathrm{Geom}_{/Y})_\bP}
\newcommand{\infcat}{$\infty$-category\xspace}
\newcommand{\infcats}{$\infty$-categories\xspace}
\newcommand{\Span}{\mathrm{Span}}
\newcommand{\infsite}{$\infty$-site\xspace}
\newcommand{\infsites}{$\infty$-sites\xspace}
\newcommand{\inftopos}{$\infty$-topos\xspace}
\newcommand{\inftopoi}{$\infty$-topoi}
\newcommand{\pres}{{}^{\mathrm L} \mathcal P \mathrm{res}}
\newcommand{\Grpd}{\mathrm{Grpd}}
\newcommand{\sSet}{\mathrm{sSet}}
\newcommand{\rSet}{\mathrm{Set}}
\newcommand{\Ab}{\mathrm{Ab}}
\newcommand{\DAb}{\cD(\Ab)}
\newcommand{\tauan}{\tau_\mathrm{an}}
\newcommand{\qet}{\mathrm{q\acute{e}t}}
\newcommand{\tauet}{\tau_\mathrm{\acute{e}t}}
\newcommand{\tauqet}{\tau_\mathrm{q\acute{e}t}}
\newcommand{\bPsm}{\bP_\mathrm{sm}}
\newcommand{\bPqsm}{\bP_\mathrm{qsm}}
\newcommand{\Modh}{\mathrm{Mod}^\heartsuit}
\newcommand{\Mod}{\mathrm{Mod}}
\newcommand{\sMod}{\mathrm{sMod}}
\newcommand{\Coh}{\mathrm{Coh}}
\newcommand{\Cohb}{\mathrm{Coh}^{\mathsf b}}
\newcommand{\Cohh}{\mathrm{Coh}^\heartsuit}
\newcommand{\RcHom}{\rR\!\mathcal H\!\mathit{om}}
\newcommand{\kfiltered}{$\kappa$-filtered\xspace}
\newcommand{\St}{\mathrm{St}}
\newcommand{\Stn}{\mathrm{Stn}}
\newcommand{\Sch}{\mathrm{Sch}}
\newcommand{\FSch}{\mathrm{FSch}}
\newcommand{\Aff}{\mathrm{Aff}}
\newcommand{\Afflfp}{\mathrm{Aff}^{\mathrm{lfp}}}
\newcommand{\An}{\mathrm{An}}
\newcommand{\Afd}{\mathrm{Afd}}
\newcommand{\Top}{\mathcal T\mathrm{op}}
\newcommand{\Cohty}{\mathrm{Coh}^+(\fY) / \Coh^+(\fY)_\tors}
\newcommand{\Cohtu}{\mathrm{Coh}^+(\fU_n) / \Coh^+(\fU_n)_\tors}
\newcommand{\Nil}{\mathrm{Nil}}


%DFmlG

\newcommand{\ad}{\mathrm{adic}}
\newcommand{\dfDM}{\mathrm{dfDM}}
\newcommand{\dfSch}{\mathrm{dfSch}}
\newcommand{\rig}{\mathrm{rig}}
\newcommand{\rigg}{(-)^{\mathrm{rig}}}
\newcommand{\loc}{\mathrm{loc}}
\newcommand{\fXzar}{\fX^{\mathrm{dfSch}}_{\mathrm{dfZar}}}
\newcommand{\cTad}{\cT_{\mathrm{adic}}(k^\circ)}
\newcommand{\taft}{\mathrm{taft}}


% DAnG

\newcommand{\dAn}{\mathrm{dAn}}
\newcommand{\dAnc}{\mathrm{dAn}_{\mathbb C}}
\newcommand{\dAnk}{\mathrm{dAn}_k}
\newcommand{\Ank}{\mathrm{An}_k}
\newcommand{\cTan}{\cT_{\mathrm{an}}}
\newcommand{\cTank}{\cT_{\mathrm{an}}(k)}
\newcommand{\cTdisc}{\cT_{\mathrm{disc}}}
\newcommand{\cTdisck}{\cT_{\mathrm{disc}}(k)}
\newcommand{\cTet}{\cT_{\mathrm{\acute{e}t}}}
\newcommand{\cTetk}{\cT_{\mathrm{\acute{e}t}}(k)}
\newcommand{\Str}{\mathrm{Str}}
\newcommand{\wStr}[2]{\Fun'(#1, #2)} %wStr stands for weak structures, i.e. structures where we dropped the third conditions.
\newcommand{\Strloc}{\mathrm{Str}^\mathrm{loc}}
\newcommand{\RTop}{\tensor*[^\rR]{\Top}{}}
\newcommand{\LTop}{\tensor*[^\rL]{\Top}{}}
\newcommand{\RHTop}{\tensor*[^\rR]{\mathcal{H}\Top}{}}
\newcommand{\LRT}{\mathrm{LRT}}
\newcommand{\Tor}{\mathrm{Tor}}
\newcommand{\dAfd}{\mathrm{dAfd}}
\newcommand{\dAfdk}{\mathrm{dAfd}_k}
\newcommand{\dStn}{\mathrm{dStn}}
\newcommand{\biget}{\mathrm{big,\acute{e}t}}
\newcommand{\trunc}{\mathrm{t}_0}
\newcommand{\Hyp}{\mathrm{Hyp}}
\newcommand{\HSpec}{\mathrm{HSpec}}
\newcommand{\CAlg}{\mathrm{CAlg}}
\newcommand{\CAlgad}{\mathrm{CAlg}^{\mathrm{ad}}}
\newcommand{\Ring}{\mathrm{Ring}}
\newcommand{\CRing}{\mathrm{CRing}}
\newcommand{\sCRing}{\mathrm{sCRing}}
\newcommand{\trunctopoi}{\Spec^{\cG_{\mathrm{an}}^{\le 0}(k)}_{\cG_{\mathrm{an}(k)}}}
\newcommand{\Cat}{\mathrm{Cat}}
\newcommand{\Catinf}{\mathrm{Cat}_\infty}
\newcommand{\Catst}{\mathrm{Cat}_{\infty}^{\mathrm{st}}}
\newcommand{\final}{\mathrm{final}}
\newcommand{\initial}{\mathrm{initial}}
\newcommand{\anPreStk}{\mathrm{AnPreStk}}


% Analytic deformation theory

\newcommand{\fib}{\mathrm{fib}}
\newcommand{\DerAn}{\mathrm{Der}\an}
\newcommand{\anL}{\mathbb L\an}
\newcommand{\fCAlg}{\mathrm{fCAlg}}
\newcommand{\adL}{\mathbb L^{\mathrm{ad}}}
\newcommand{\AnRing}{\mathrm{AnRing}}
\newcommand{\SpecEtAn}{\mathrm{Spec}^{\cTan}_{\cTet}}
\newcommand{\cTanc}{\cTan(\mathbb C)}
\newcommand{\Zar}{_{\mathrm{Zar}}}
\newcommand{\dAff}{\mathrm{dAff}}
\newcommand{\afp}{^{\mathrm{afp}}}
\newcommand{\bfMap}{\mathbf{Map}}
\newcommand{\cHom}{\cH \mathrm{om}}
\newcommand{\dAnSt}{\mathrm{dAnSt}}
\newcommand{\PrL}{\mathcal P \mathrm{r}^{\mathrm{L}}}
\newcommand{\PrR}{\mathcal P \mathrm{r}^{\mathrm{R}}}
\newcommand{\cTAb}{\cT_{\Ab}}
\newcommand{\Catlex}{\Cat_\infty^{\mathrm{lex}}}
\newcommand{\underover}[1]{#1//#1}
\newcommand{\Perf}{\mathrm{Perf}}
\newcommand{\anNil}{\mathrm{AnNil}}
\newcommand{\anFMP}{\mathrm{AnFMP}}
\newcommand{\anFGrpd}{\mathrm{AnFGrpd}}


% Hilbert and Map

\newcommand{\fAfflfp}[1]{\mathrm{fAff}_{#1}^{\mathrm{lfp}}}
\newcommand{\fAff}[1]{\mathrm{fAff}_{#1}}
\newcommand{\Hilb}{\mathrm{Hilb}}
\newcommand{\dSch}{\mathrm{dSch}}
\newcommand{\FormalModels}{\mathrm{FM}}
\newcommand{\Ind}{\mathrm{Ind}}

% Non-archimedean Quantum K-theory

\newcommand{\vdim}{\mathrm{vdim}}
\newcommand{\cOvir}{\cO^{\mathrm{vir}}}
\newcommand{\bsM}{\widebar{\mathscr M}}
\newcommand{\fM}{\mathfrak M}
\newcommand{\tsigma}{\widetilde{\sigma}}

% Special symbols
\newcommand{\bcM}{\widebar{\mathcal M}}
\newcommand{\bcC}{\widebar{\mathcal C}}
\newcommand{\bcMgn}{\widebar{\mathcal M}_{g,n}}
\newcommand{\bcMol}{\widebar{\mathcal M}_{0,1}}
\newcommand{\bcMot}{\widebar{\mathcal M}_{0,3}}
\newcommand{\bcMof}{\widebar{\mathcal M}_{0,4}}
\newcommand{\bcMon}{\widebar{\mathcal M}_{0,n}}
\newcommand{\bcMgnprime}{\widebar{\mathcal M}_{g,n'}}
\newcommand{\bcMgnijprime}{\widebar{\mathcal M}_{g_{ij},n'_{ij}}}
\newcommand{\bMgnt}{\widebar{M}^\mathrm{trop}_{g,n}}
\newcommand{\Mmdisc}{M_{m\textrm{-disc}}}
\newcommand{\Gm}{\mathbb G_{\mathrm m}}
\newcommand{\Gmk}{\mathbb G_{\mathrm m/k}}
\newcommand{\Gmkprime}{\mathbb G_{\mathrm m/k'}}
\newcommand{\Gmnan}{(\Gm^n)\an}
\newcommand{\Gmknan}{(\Gmk^n)\an}
\newcommand{\Lin}{\mathit{Lin}}
\newcommand{\Simp}{\mathit{Simp}}
\newcommand{\vol}{\mathit{vol}}
\newcommand{\LanD}{\mathcal L_{an}^D}

% Categories

\newcommand{\cart}{\mathrm{cart}}
\DeclareMathOperator{\Tw}{Tw}
\DeclareMathOperator{\Exc}{Exc}
\newcommand{\fin}{\mathrm{fin}}
\newcommand{\lex}{\mathrm{lex}}
\DeclareMathOperator{\Ob}{Ob}
\newcommand{\ind}{\mathrm{Ind}}
\newcommand{\pro}{\mathrm{Pro}}
\newcommand{\cofib}{\mathrm{cofib}}
\newcommand{\QCoh}{\mathrm{QCoh}}
\newcommand{\QCohcpl}{\mathrm{QCoh}_{\mathrm{cpl}}}

% Shorthands
\newcommand{\kc}{k^\circ}
\newcommand{\llb}{[\![}
\newcommand{\rrb}{]\!]}
\newcommand{\llp}{(\!(}
\newcommand{\rrp}{)\!)}
\newcommand{\an}{^\mathrm{an}}
\newcommand{\alg}{^\mathrm{alg}}
\newcommand{\loweralg}{_\mathrm{alg}}
\newcommand{\bad}{^\mathrm{bad}}
\newcommand{\ess}{^\mathrm{ess}}
\newcommand{\ness}{^\mathrm{ness}}
\newcommand{\et}{_\mathrm{\acute{e}t}}
\newcommand{\fet}{_\mathrm{f\acute{e}t}}
\newcommand{\ev}{\mathit{ev}}
%\newcommand{\eistar}{\mathbf e_i^*}
%\newcommand{\ejstar}{\mathbf e_j^*}
%\newcommand{\ekstar}{\mathbf e_k^*}
\newcommand{\mult}{\mathit{mult}}
\newcommand{\inv}{^{-1}}
\newcommand{\id}{\mathrm{id}}
\newcommand{\gn}{$n$-pointed genus $g$ }
\newcommand{\gnprime}{$n'$-pointed genus $g$ }
\newcommand{\GW}{\mathrm{GW}}
\newcommand{\GWon}{\GW_{0,n}}
\newcommand{\canal}{$\mathbb C$-analytic\xspace}
\newcommand{\nanal}{non-archimedean analytic\xspace}
\newcommand{\kanal}{$k$-analytic\xspace}
\newcommand{\ddim}{$d$-dimensional\xspace}
\newcommand{\ndim}{$n$-dimensional\xspace}
\newcommand{\narch}{non-archimedean\xspace}
\newcommand{\nminusone}{$(n\!-\!1)$}
\newcommand{\nminustwo}{$(n\!-\!2)$}
\newcommand{\red}{\mathrm{red}}
\renewcommand{\th}{^\mathrm{\tiny th}}
\newcommand{\Wall}{\mathit{Wall}}
\newcommand{\vlb}{virtual line bundle\xspace}
\newcommand{\mvlb}{metrized \vlb}
\newcommand{\wrt}{with respect to\xspace}
\newcommand{\Zaffine}{$\mathbb Z$-affine\xspace}
\newcommand{\sw}{^\mathrm{sw}}
\newcommand{\Trop}{\mathrm{Trop}}
\newcommand{\trop}{^\mathrm{trop}}
\newcommand{\op}{^\mathrm{op}}
\newcommand{\Cech}{\check{\mathcal C}}
\newcommand{\DM}{Deligne-Mumford\xspace}
\providecommand{\abs}[1]{\lvert#1\rvert}
\providecommand{\norm}[1]{\lVert#1\rVert}
\newcommand{\tors}{\mathrm{tors}}
\newcommand{\cpl}{\mathrm{cpl}}
\newcommand{\cl}{\mathrm{cl}}


% Arrows
\newcommand*{\longhookrightarrow}{\ensuremath{\lhook\joinrel\relbar\joinrel\rightarrow}}
\newcommand*{\DashedArrow}[1][]{\mathbin{\tikz [baseline=-0.25ex,-latex, dashed,#1] \draw [#1] (0pt,0.5ex) -- (1.3em,0.5ex);}}

\usetikzlibrary{decorations.markings} %arrows for open immersions and closed immersions
\tikzset{
  closed/.style = {decoration = {markings, mark = at position 0.5 with { \node[transform shape, xscale = .8, yscale=.4] {/}; } }, postaction = {decorate} },
  open/.style = {decoration = {markings, mark = at position 0.5 with { \node[transform shape, scale = .7] {$\circ$}; } }, postaction = {decorate} }
}


%Operators
\DeclareMathOperator{\Anc}{Anc}
\DeclareMathOperator{\Area}{Area}
\DeclareMathOperator{\Aut}{Aut}
\DeclareMathOperator{\Bl}{Bl}
\DeclareMathOperator{\CH}{CH}
\DeclareMathOperator{\Coker}{Coker}
\DeclareMathOperator{\codim}{codim}
\DeclareMathOperator{\cosk}{cosk}
\DeclareMathOperator{\Div}{Div}
\DeclareMathOperator{\dist}{dist}
\DeclareMathOperator{\Ext}{Ext}
\DeclareMathOperator{\Fun}{Fun}
\DeclareMathOperator{\FunR}{Fun^R}
\DeclareMathOperator{\FunL}{Fun^L}
\DeclareMathOperator{\Gal}{Gal}
\DeclareMathOperator{\Hom}{Hom}
\DeclareMathOperator{\Image}{Im}
\DeclareMathOperator{\Int}{Int}
\DeclareMathOperator{\Isom}{Isom}
\DeclareMathOperator{\Ker}{Ker}
\DeclareMathOperator{\Map}{Map}
\DeclareMathOperator{\Mor}{Mor}
\DeclareMathOperator{\NE}{NE}
\DeclareMathOperator{\oStar}{\widebar{\Star}}
\DeclareMathOperator{\Pic}{Pic}
\DeclareMathOperator{\Proj}{Proj}
\DeclareMathOperator{\rank}{rank}
\DeclareMathOperator{\RHom}{RHom}
\DeclareMathOperator{\Sp}{Sp}
\DeclareMathOperator{\Spa}{Spa}
\DeclareMathOperator{\SpB}{Sp_\mathrm{B}}
\DeclareMathOperator{\Spec}{Spec}
\DeclareMathOperator{\Spf}{Spf}
\DeclareMathOperator{\Star}{Star}
\DeclareMathOperator{\supp}{supp}
\DeclareMathOperator{\Sym}{Sym}
\DeclareMathOperator{\val}{val}

\DeclareMathOperator*{\colim}{colim}
\DeclareMathOperator*{\holim}{holim}
\DeclareMathOperator*{\hocolim}{hocolim}
\DeclareMathOperator*{\cotimes}{\widehat{\otimes}}


\begin{document}

\title{Spreading out the Hodge filtration in rigid analytic geometry}

\author{Jorge ANT\'ONIO}
\address{Jorge ANT\'ONIO, IRMA, UMR 7501
7 rue René-Descartes
67084 Strasbourg Cedex}
\email{jorgeantonio@unistra.fr}


\begin{abstract}

\end{abstract}

\maketitle

\tableofcontents

\section{Introduction}

In this paper, we will provide a rigid analytic construction of the deformation to the normal cone, studied in \cite{Gaitsgory_Study_II}.
Our goal is to use this geometric construction to deduce certain important results concerning both \emph{rigid analytic} and
\emph{over-convergent} (Hodge complete)
\emph{derived de Rham cohomology} of rigid analytic spaces over a non-archimedean field of characteristic zero.
We will then exploit this ideas to come up with analogues concerning \emph{derived rigid cohomology} of finite type schemes over a perfect field
in characteristic zero. In particular, our main goal is to extrapolate the main result of \cite{Bhatt_Derived_Completions} to the setting of
derived rigid cohomology.

\subsection{Preliminaries}
Let $\cX$ be an \inftopos. The notion of a \emph{local $\cTank$-structure on $\cX$} was first introduced in \cite[Definition 2.4]{Porta_Yu_Derived_non-archimedean_analytic_spaces},
see also \cite[\S 2]{antonio2018p}.

Let  $\cO \in \Strloc_{\cTank}(\cX)$ be a local $\cTank$-structure on $\cX$. Since the pregeometry $\cTank$ is compatible
with $n$-truncations, cf. \cite[Theorem 3.23]{Porta_Yu_Derived_non-archimedean_analytic_spaces}, it follows that
$\pi_0(\cO) \in \Strloc_{\cTank}(\cX)$, as well.

Denote by $\cJ \subseteq \pi_0(\cO)$, the \emph{Jacobson ideal} of $\pi_0(\cO \alg)$, which can be naturally regarded as an object
in the \infcat
    \[
        \Mod_{\pi_0(\cO \alg)} \simeq \Mod_{\pi_0(\cO)},  
    \]
for a justification of the latter equivalence, see for instance \cite[Theorem 4.5]{Porta_Yu_Representability} .
Since the \infcat $\Str_{\cTank}(\cX)$ is a presentable \infcat we can consider the quotient
    \[
        \pi_0(\cO)_\red \coloneqq \pi_0(\cO)/\cJ \in \Str_{\cTank}(\cX)  ,
    \]
which we refer to the \emph{reduced $\cTank$-structure on $\cX$ associated to $\pi_0(\cO)$}. Moreover, the corresponding \emph{underlying algebra} satisfies
    \[(\pi_0(\cO)_{\red})\alg \simeq \pi_0(\cO)\alg/ \cJ \in \Str_{\cTdisck}(\cX).\]
One can further prove that
$\pi_0(\cO)_\red \in \Str_{\cTank}(\cX)$ actually lies in the full subcategory $\Strloc_{\cTank}(\cX)$.

\begin{defin}
    Let $Z = (\cZ, \cO_Z) \in \RTop(\cTan(k))$ denote a $\cTank$-structured \inftopos. We define the \emph{reduced $\cTank$-structure \inftopos} as
        \[
            Z_\red \coloneqq (\cZ, \pi_0(\cO_Z)_\red) \in \RTop(\cTank).
        \]
    We shall denote by $\Afd_k^\red$ (resp., $\Ank^\red$) the full subcategory of $\dAfd_k$ (resp., $\dAnk$)
    spanned by reduced $k$-affinoid (resp., $k$-analytic spaces).
\end{defin}

\begin{notation}
    Let $(-)^\red \colon \dAnk \to \Ank^\red$ denote the functor obtained by the formula
        \[
            Z = (\cZ, \cO_Z) \in \dAnk\mapsto Z_\red = (\cZ, \pi_0(\cZ)_\red) \in \Ank^\red.
        \]
    We shall refer to it as the \emph{underlying reduced $k$-analytic space}.
\end{notation}

\begin{lem}
    Let $f \colon X \to Y$ be a Zariski open immersion of derived $k$-analytic spaces. Then $f^\red \colon X^\red \to Y^\red$ is also
    a Zariski open immersion.
\end{lem}

\begin{proof}
    By the definitions, it is clear that the truncation
        \[
            \trunc(f) \colon \trunc(X) \to \trunc(Y),  
        \]
    is a Zariski open immersion of ordinary $k$-analytic spaces. In the case of ordinary $k$-analytic spaces it is clear from the construction
    that the reduction of Zariski open immersions is again a Zariski open immersion.
\end{proof}

\begin{defin}
    In \cite[Definition 5.41]{Porta_Yu_Representability} the authors introduced the notion of a square-zero extension between $\cTank$-structured
    \inftopoi. In particular, given a morphism $f \colon Z \to Z'$ in $\RTop(\cTank)$, we shall say that $f$ \emph{has the structure of
    a square-zero extension} if $f$ exhibits $Z'$ as a square-zero extension of $Z$.
\end{defin}

Recall the definition of the \infcats of derived $k$-affinoid and derived $k$-analytic spaces given
in \cite[Definition 7.3 and Definition 2.5.]{Porta_Yu_Derived_non-archimedean_analytic_spaces}, respectively.

\begin{rem} \label{rem:construction_of_nilpotent_extensions_as_square_zero_extensions}
    Let $X \in \Ank$. Let $\cJ \subseteq \cO_X$ be an ideal satisfying $\cJ^2 = 0$. Consider the fiber sequence
        \[
            \cJ \to \cO_X \to \cO_X/\cJ,  
        \]
    in the \infcat $\Coh^+(X)$. It corresponds to a well defined morphism $d \colon \cO_X / \cJ \to \cJ[1]$ admitting $\cO_X$ as fiber. The morphism $d$ defines
    a derivation $d \colon \bbL\an_{\cO_X/ \cJ} \to \cJ[1]$, by pre-composing with the natural map $\cO_X/ \cJ \to \bbL\an_{\cO_X/ \cJ}$.
    In particular, we can consider the square-zero extension of $\cO_X$ by $\cJ$ induced by $\cJ$ defined by $d$. The latter object
    must then be equivalent to $\cO_X$ itself. We conclude that $\cO_X$ is a square-zero extension of $\cO_X/\cJ$.
\end{rem}

\begin{lem} \label{lem:derived_k_analytic_space_whose_reduction_is_affinoid_is_also_affinoid}
    Let $Z \coloneqq (\cZ, \cO_Z) \in \RTop(\cTank)$ denote a $\cTank$-structure \inftopos. Suppose that the reduction
    $Z_\red$ is equivalent to a derived $k$-affinoid space. Then the truncation $\trunc(Z)$ is isomorphic to an ordinary $k$-affinoid space.
    If we assume further that for every $i>0$, the homotopy sheaves $\pi_i(\cO_Z)$ are
    coherent $\pi_0(\cO_Z)$-modules, then $Z$ itself is equivalent to a derived $k$-affinoid space.
\end{lem}

\begin{proof} We first observe that the second claim of the Lemma follows readily from the first one. We thus are thus reduced to prove that $\trunc(Z)$ is isomorphic to an ordinary $k$-affinoid space.
    Let $\cJ \subseteq \pi_0(\cO_Z)$, denote the coherent ideal sheaf associated to the closed immersion $Z_\red \hookrightarrow Z$. Notice that the ideal $\cJ$
    agrees with the Jacobson ideal of $\pi_0(\cO_Z)$. Since derived $k$-analytic spaces are Noetherian, it follows that there exists
    a sufficiently large integer $n \ge 2$ such that
        \[
            \cJ^n = 0.  
        \]
    Arguing by induction we can suppose that $n = 2$, that is to say that
        \[\cJ^2 = 0.\]
    In particular, \cref{rem:construction_of_nilpotent_extensions_as_square_zero_extensions} implies that the above map has
    the natural morphism $Z_\red \to Z$ has the structure of a square zero extension.
    The assertion now follows from \cite[Proposition 6.1]{Porta_Yu_Representability}
    and its proof.
\end{proof}

\begin{rem}
    We observe that the converse of \cref{lem:derived_k_analytic_space_whose_reduction_is_affinoid_is_also_affinoid} holds true.
    Indeed, the natural morphism $Z_\red \to Z$ is a closed immersion. In particular, if $Z \in \dAfd_k$ we deduce readily from
    that $Z_\red \in \dAfd_k$, as well.
\end{rem}

\begin{defin}
    Let $f \colon X \to Y$ be a morhpism in the \infcat $\dAnk$. We shall say that $f$ is an \emph{affine morphism} if
    for every morphism $Z \to Y$ in $\dAnk$ such that $Z$ is equivalent to a derived $k$-affinoid space, the pullback
        \[
            Z' \coloneqq Z \times_Y X \in \dAnk,  
        \]
    is also equivalent to a derived $k$-affinoid space.
\end{defin}

\begin{notation}
    Let $f \colon X \to Y$ be a morphism of derived $k$-analytic spaces. We shall denote by
        \[
            f^\# \colon \cO_Y \to f_* \cO_X,
        \]  
    the induced morphism at the level of $\cTank$-structures.
\end{notation}

\begin{lem} \label{lem:affine_morphisms_are_compatible_with_Zariski_localization_on_the_base}
    Let $f \colon X \to Y$ be an affine morphism in $\dAn_k$. Suppose that we are given a Zariski open immersion
    $g \colon Z \to Y$ such that $Z \in \dAfdk$ which corresponds to the completement of the zero locus of a
    section $s \in \pi_0(\cO_Y)$. Then the fiber product
        \[
            Z' \coloneqq Z \times_Y X \in \dAn_k,  
        \]
    is equivalent to a derived $k$-affinoid space and moreover $\Gamma(Z', \cO_{Z'}\alg) \simeq B[1/f^\#(s)]$, where
    $B \coloneqq \Gamma(X, \cO_X\alg)$.
\end{lem}

\begin{proof} The first assertion of the Lemma follows readily from the definition of affine morphisms.
    We shall now prove the second claim. Let $A \coloneqq \Gamma(Y, \cO_Y\alg)$. In this case, we have a natural equivalence of
    derived $k$-algebras
        \[
            A[1/f] \simeq \Gamma(Z, \cO_Z\alg).  
        \]
    Since Zariski open immersions are stable under pullbacks, it follows that the natural morphism $g' \colon Z' \to X$ is itself a Zariski
    open immersion. In particular, it follows that we can identify
        \[
            \Gamma(Z', \cO_{Z'}) \simeq B[1/t],
        \]
    where $t \in \pi_0(B)$. In order to conclude the proof, we observe that the $0$-th truncation, $\trunc(g)$, is again a Zariski
    open immersion. For this reason, one should have forcibly that $t = f^\#(s)$, by the universal
    property of fiber products of ordinary $k$-analytic spaces.
\end{proof}


\section{Non-archimedean differential geometry}

\subsection{Analytic formal moduli problems under a base}
In this \S, we will study the notion of \emph{analytic formal moduli problems under} a fixed derived $k$-analytic space. The
results presented here will prove to be crucial for the study of the deformation to the normal cone in the $k$-analytic
setting, presented in the next section.
We start with the following definition:

\begin{defin}
    Let $f \colon X \to Y$ be a morphism in $\dAnk$. We say that $f$ is a \emph{nil-isomorphism} if $f_\red \colon X_\red \to
    Y_\red$ is an isomorphism of $k$-analytic spaces. We denote by $\anNil_{/ X}$ the full subcategory of $(\dAnk)^{\mathrm{ft}}_{X/}$
    spanned by nil-isomorphisms $X \to Y$ of finite type.
\end{defin}

\begin{lem} \label{lem:nil-isos_are_affine_morphisms}
    Let $f \colon X \to Y$ be a nil-isomorphism in $\dAnk$. Then:
    \begin{enumerate}
        \item Given any morphism $Z \to Y$ in $\dAnk$, the induced morphism
            \[
                Z \times_X Y \to Z,  
            \]
        is again an nil-isomoprhism.
        \item $f$ is an affine morphism.
        \item $f$ is a finite morphism.
    \end{enumerate}
\end{lem}

\begin{proof} To prove (i), it suffices to prove that
    the functor $(\textrm{-})^\red \colon \dAnk \to \Ank^{\red}$ commutes with finite limits. The truncation functor
        \[
            \trunc \colon \dAnk \to \Ank,  
        \]
    commutes with finite limits. So we further reduce ourselves to the prove that the usual underlying reduced functor
        \[
            (-)^\red \colon \Ank \to \Ank^\red,
        \]
    commutes with finite limits. By construction,
    the latter assertion is equivalent to the claim that
    the complete tensor product of ordinary $k$-affinoid algebras commutes with the operation of taking the quotient by the Jacobson radical, which is immediate.

    We now prove (ii). Let $Z \to Y$ be a Zariski open immersion such that $Z$ is a derived $k$-affinoid space. Then we claim that the pullback
    $Z \times_X Y$ is again a derived $k$-affinoid space. Thanks to \cref{lem:derived_k_analytic_space_whose_reduction_is_affinoid_is_also_affinoid}
    we reduced to prove that $(Z \times_X Y)_\red$ is equivalent to an
    ordinary $k$-affinoid space. Thanks to (i), we deduce that the induced morphism
        \[
            (Z \times_X Y)_\red \to Z_\red,  
        \]
    is an isomorphism of ordinary $k$-analytic spaces. In particular, $(Z \times_X Y)_\red$ is a $k$-affinoid space. The result now follows from
    \cref{lem:derived_k_analytic_space_whose_reduction_is_affinoid_is_also_affinoid}.

    To prove (iii), we shall show that the induced morphism on the $0$-th truncations $\trunc(X) \to \trunc(Y)$ is a finite morphism of ordinary $k$-affinoid spaces.
    But this follows immeaditely from the fact that both $\trunc(X)$ and $\trunc(Y)$ can be obtained from the reduced $X_\red$ by means of a finite sequence of finite
    coherent $X_\red$-modules.
\end{proof}


\begin{defin}
    A morphism $X \to Y$ be a morphism in $\dAnk$ is called a \emph{nil-embedding} if the induced map of ordinary $k$-analytic spaces
    $\trunc(X) \to \trunc(Y)$ is a closed immersion, such that the ideal of $\trunc(X)$ in $\trunc(Y)$ is nilpotent. 
\end{defin}

\begin{prop} \label{prop:filtered_colimit_for_nil-embeddings}
    Let $f \colon X \to Y$ be a nil-embedding of derived $k$-analytic spaces. Then there exists a sequence of morphisms
        \[X = X_0^0 \hookrightarrow X_0^1 \hookrightarrow \dots \hookrightarrow X_0^n = X_0 
        \hookrightarrow X_1 \dots X_n \hookrightarrow \dots \hookrightarrow Y,\]
    such that for each $0 \le i \le n$ the morphism $X_0^i \hookrightarrow X_0^{i+1}$ has the structure of a square zero extension.
    Similarly, for every $i \ge 0$, the morphism $X_i \hookrightarrow X_{i+1}$ has the structure of a square-zero extension.
    Furthermore, the induced morphisms $\mathrm{t}_{\le i}(X_i) \to \mathrm{t}_{\le i}(Y)$ are equivalences of derived
    $k$-analytic spaces.
\end{prop}

\begin{proof}
    The proof follows the same scheme of reasoning as of \cite[Proposition 5.5.3]{Gaitsgory_Study_II}. For the sake of completeness we present the complete here.
    Consider the induced morphism on the underlying truncations
        \[
            \trunc(f) \colon \trunc(X) \to \trunc(Y).   
        \]
    By construction, there exists a sufficiently large integer $n \ge 0$ such that
        \[
            \cJ^{n+1} = 0,  
        \]
    where $\cJ \subseteq \pi_0(\cO_Y)$ denotes the ideal associated to the nil-embedding $\trunc(f)$.
    Therefore, we can factor the latter as a finite sequence of square-zero extensions of ordinary $k$-analytic spaces
        \[
            \trunc(X) \hookrightarrow X_0^{\mathrm{ord}, 0} \hookrightarrow \dots \hookrightarrow X^{\mathrm{ord}, n}_0 = \trunc(Y),
        \]
    as in the proof of \cref{lem:derived_k_analytic_space_whose_reduction_is_affinoid_is_also_affinoid}. For each $0 \le i \le n$, we set
        \[
            X_0^i \coloneqq X \bigsqcup_{\trunc(X)} X_0^{\mathrm{ord}, i}.
        \]
    By construction, we have that the natural morphism $\trunc(X_0^n) \to \trunc(Y)$ is an isomorphism of ordinary $k$-analytic spaces.
    We now argue by induction on the Postnikov towers associated to the morphism $f \colon X \to Y$.
    Suppose that for a certain integer $i \ge 0$, we have constructed a derived $k$-analytic space $X_i$ together with morphisms $g_i \colon
    X \to X_i$ and $h_i \colon X_i \to Y$ such that $f \simeq h_i \circ g_i$
    and the induced morphism
        \[
            \mathrm{t}_{\le i}(X_i) \to \mathrm{t}_{\le i}(Y)
        \]
    is an equivalence of derived $k$-analytic spaces. We shall proceed as follows: by the assumption that $h_i$ is $(i+1)$-connective, we deduce from
    \cite[Proposition 5.34]{Porta_Yu_Representability} the existence of a natural equivalence
        \[
            \tau_{\le i}(\bbL_{X_i/Y}\an) \simeq 0,
        \]
    in $\Mod_{\cO_{X_i}}$. Consider the natural fiber sequence
        \[
            h_i^* \bbL\an_{Y} \to \bbL\an_{X_i} \to \bbL\an_{X_i/Y},
        \]
    in $\Mod_{\cO_{X_i}}$. The natural morphism
        \[
            \bbL\an_{X_i/ Y} \to \pi_{i+1}(\bbL\an_{X_i/Y})[i+1],  
        \]
    induces a morphism $\bbL\an_{X_i} \to \pi_{i+1}(\bbL\an_{X_i/Y})[i+1]$, such that the composite
        \begin{equation} \label{eq:fiber_sequence_of_cotangent_complexes_to_produce_the_existence_of_the_desired_square_zero_extension_approximating_Y_in_degree_i+1}
            h_i^* \bbL\an_{Y} \to \bbL\an_{X_i} \to \pi_{i+1}(\bbL\an_{X_i/Y}),  
        \end{equation}
    is null-homotopic, in $\Mod_{\cO_{X_i}}$. The existence of
    \eqref{eq:fiber_sequence_of_cotangent_complexes_to_produce_the_existence_of_the_desired_square_zero_extension_approximating_Y_in_degree_i+1}
    produces a square-zero extension
        \[
            X_i  \to X_{i+1},
        \]
    together with a morphism $h_{i+1} \colon X_{i+1} \to Y$, factoring $h_i \colon X_i \to Y$. We  are reduced to show that the morphism
        \[
            \cO_Y \to h_{i+1, *}(\cO_{X_{i+1}}), 
        \]
    is $(i+2)$-connective. Consider the commutative diagram
        \begin{equation} \label{eq:commutative_diagram_of_fiber_sequences_exhibiting_O_X_i+1_as_an_approximation_of_level_i+1_of_Y}
        \begin{tikzcd}
            h_{i, *}(\pi_{i+1}(\bbL\an_{X_i/Y}))[i] \ar{r} & h_{i+1, *}(\cO_{X_{i+1}}) \ar{r} & h_{i, *}(\cO_{X_i}) \\
            \cI \ar{r} \ar{u}{s_i} & \cO_Y \ar{r} \ar{u} & h_*(\cO_{X_i}) \ar{u} \\
            \cJ \ar{r} \ar{u} & \cJ \ar{r} \ar{u} & 0 \ar{u}
        \end{tikzcd},
        \end{equation}
    where both the vertical and horizontal composites are fiber sequences. Thanks to \cite[Proposition 5.34]{Porta_Yu_Representability} we can identify the natural
    morphism    
        \[
           s_i \colon \cI \to h_{i, *}(\pi_{i+1}(\bbL\an_{X_i/Y}))[i]
        \]
    with the natural morphism $\cI \to \tau_{\ge i}(I)$. We deduce that the fiber of the morphism $s_i$ must be necessarily $(i+1)$-connective. The latter observation
    combined with the structure of \eqref{eq:commutative_diagram_of_fiber_sequences_exhibiting_O_X_i+1_as_an_approximation_of_level_i+1_of_Y}
    implies that $h_{i+1} \colon X_{i+1} \to Y$ induces an equivalence of derived $k$-analytic spaces
        \[
            \rt_{\le i+1}(X_{i+1}) \to \rt_{\le i+1}(Y),  
        \]
    as desired.
\end{proof}

\begin{cor}
    Let $X \in \dAnk$. Then the natural morphism
        \[
            X_\red \to X,  
        \]
    in $\dAnk$, can be \emph{approximated} by successive square zero extensions.
\end{cor}

\begin{proof}
    The assertion of the Corollary follows readily from \cref{prop:filtered_colimit_for_nil-embeddings} by observing that the
    canonical morphism $X_\red \to X$ has the structure of a nil-embedding.
\end{proof}


\begin{lem} \label{lem:f^*_admits_a_right_adjoint_whenever_f_is_nil-iso}
    Let $f \colon S \to S'$ be a nil-isomorphism between derived $k$-analytic spaces. Then the pullback functor
        \[
            f^* \colon \Coh^+(S') \to \Coh^+(S),  
        \]
    admits a well defined right adjoint, $f_*$.
\end{lem}

\begin{proof}
    Since $f \colon S \to S'$ is a nil-isomorphism, we conclude from \cref{lem:nil-isos_are_affine_morphisms} that $f$ is an affine morphism
    between derived $k$-analytic spaces. By Zariski descent of $\Coh^+$, cf. \cite[Theorem 3.7]{Antonio_Porta_Nonarchimedean_Hilbert},
    together with \cref{lem:affine_morphisms_are_compatible_with_Zariski_localization_on_the_base} we reduce the statement of the Lemma to the case
    where both $S$ and $S'$ are equivalent to derived $k$-affinoid spaces. In this case, by Tate acyclicity theorem we reduce ourselves
    to show that the usual base change functor
        \[
           f^* \colon  \Coh^+(A) \to \Coh^+(B)  ,
        \]
    where $A \coloneqq \Gamma(S, \cO_S \alg)$ and $B \coloneqq \Gamma(S', \cO_{S'} \alg)$, admits a right adjoint. The result now follows from the observation
    that the canonical induced morphism $\pi_0(A) \to \pi_0(B)$ is a finite morphism of ordinary rings. Indeed, the latter morphism can be obtained
    by means of a finite sequence of (classical) square-zero extensions with respect to the corresponding Jacobson ideals of both $\pi_0(A)$ and $\pi_0(B)$.
    Such ideals are necessarily finitely generated as $\pi_0(A)$-modules, and the result follows.
\end{proof}

\begin{lem} \label{lem:pushouts_of_square_zero_extensions_have_the_structure_of_a_square_zero_extension}
    Let $f \colon S \to S'$ be a square-zero extension and $g \colon S \to T$ a nil-isomorphism in $\dAnk$. Suppose we are given a pushout diagram
        \[
        \begin{tikzcd}
            S \ar{r}{f} \ar{d} & S' \ar{d} \\
            T \ar{r} & T'  
        \end{tikzcd},
        \]
    in $\dAnk$. Then the induced morphism $T \to T'$ is a square-zero extension.
\end{lem}

\begin{proof}
    Since $g$ is a nil-isomorphism of derived $k$-analytic spaces, \cref{lem:f^*_admits_a_right_adjoint_whenever_f_is_nil-iso}
    implies that the pullback functor $g^* \colon \Coh^+(T) \to \Coh^+(S)$ admits a well defined right adjoint
        \[
            g_* \colon \Coh^+(S ) \to \Coh^+(T) .
        \]
    Let $\cF \in \Coh^+(S)^{\ge 0}$ and $d \colon \bbL\an_S \to \cF$ be a derivation
    associated with the morphism $f \colon S \to S'$. Consider now the natural composite
        \[
            d ' \colon \bbL\an_T \to g_* (\bbL\an_S) \xrightarrow{g_*(d)} g_* (\cF),  
        \]
    in the \infcat $\Coh^+(T)$. By the universal property of the analytic cotangent complex, we deduce the existence of a square-zero extension
        \[
            T \to T',  
        \]
    in the \infcat $\dAnk$. Let $X \in \dAnk$ together with morphisms $S' \to X$ and $T \to X$ compatible with both $f$ and $g$. By the universal property of
    the relative analytic cotangent complex, the morphism $S' \to X$ induces a uniquely defined (up to a contractible indeterminacy space)
        \[
            \bbL\an_{S/X} \to \cF,
        \]
    in $\Coh^+(S)$, such that the compositve $\bbL\an_S \to \bbL\an_{S/X} \to \cF$ agrees with $d$. By applying the right adjoint $g_*$ above we obtain a
    commutative diagram
        \[
        \begin{tikzcd}
            \bbL\an_T \ar{r}{\mathrm{can}} \ar{d} & \bbL\an_{T/X} \ar{d} \ar{rd}{d''} & \\
            g_*(\bbL\an_S) \ar{r} & g_*(\bbL\an_{S/X}) \ar{r} & g_*(\cF),
        \end{tikzcd}
        \]
    in the \infcat $\Coh^+(T)$. From this, we conclude again by the universal property of the relative analytic cotangent complex the existence
    of a natural morphism $T' \to X$ extending both $T \to X$ and $S' \to X$ and compatible with the restriction to $S$. The latter assertion is equivalent to state
    that the commutative square
        \[
        \begin{tikzcd}
            S \ar{r} \ar{d} & S' \ar{d} \\
            T \ar{r} & T',
        \end{tikzcd}
        \]
    is a pushout diagram in $\dAnk$. The proof is thus concluded.
\end{proof}

\begin{prop} \label{prop:existence_of_pushouts_along_closed_nil-isomorphisms}
    Let $f \colon X \to Y$ be a nil-embedding of derived $k$-analytic spaces. Let
    $g \colon X \to Z$ be a finite morphism in $\dAnk$. The the diagram
        \[
        \begin{tikzcd}
            X \ar{r}{f} \ar{d}{g} & Y \\
            Z
        \end{tikzcd}  
        \]
    admits a colimit in $\dAnk$, denoted $Z'$. Moreover, the natural morphism
    $Z \to Z'$ is also a nil-embedding.
\end{prop}


\begin{proof} The \infcat of $\cTank$-structured \inftopos $\RTop(\cTank)$ is a presentable \infcat. Consider the pushout diagram
        \[
        \begin{tikzcd}
            X \ar{r}{f} \ar{d}{g} & Y \\
            Z \ar{r} & Z',
        \end{tikzcd}
        \]
    in the \infcat $\RTop(\cTank)$. By construction, the underlying \inftopos of $Z'$ can be computed as the pushout in the \infcat $\RTop$ of
    the induced diagram on the underlying \inftopoi of $X$, $Z$ and $Y$. Moreover, since $g$ is a nil-isomorphism it induces an equivalence on underlying \inftopoi
    of both $X$ and $Y$. It follows that the induced morphism $Z \to Z'$ in $\RTop(\cTank)$ induces an equivalence on the underlying \inftopoi.
    Moreover, it follows essentially by construction that we have a natural equivalence
        \[
            \cO_{Z'} \simeq g_*(\cO_Y) \times_{g_*(\cO_Y)} \cO_Z \in \Str_{\cTank}\loc(Z).
        \]
    As effective epimorphisms are preserved under fiber products in an \inftopos, it follows that the natural morphism
        \[
            \cO_{Z'} \to \cO_Z,  
        \]
    is an effective epimorphism (since $g_*(\cO_Y) \to g_*(\cO_X)$ it is so).
    Consider now the commutative diagram of fiber sequences
        \[
        \begin{tikzcd}
            \cJ ' \ar{r} \ar{d} & \cO_{Z'} \ar{r} \ar{d}  & \cO_Z \ar{d} \\
            \cJ \ar{r} & g_*(\cO_Y) \ar{r} & g_*(\cO_X),
        \end{tikzcd}
        \]
    in the stable \infcat $\Mod_{\cO_Z'}$. Since the right commutative square is a pullback square it follows that the morphism
        \[
            \cJ' \to \cJ,  
        \]
    is an equivalence. In particular, $\pi_0(\cJ')$ is a finitely generated
    nilpotent ideal of $\pi_0(\cO\alg_{\cJ'})$. Indeed, finitely generation follows from our assumption that $g$ is a finite morphism.
    Thanks to \cref{lem:derived_k_analytic_space_whose_reduction_is_affinoid_is_also_affinoid},
    it follows that $\trunc(Z')$ is an ordinary $k$-analytic space and the morphism $\trunc(Z') \to \trunc(Z)$ is a nil-embedding. We are thus reduced to show that
    for every $i>0$, the homotopy sheaf $\pi_i(\cO_{Z'}) \in \Coh^+(\trunc(Z'))$. But this follows immediately from the existence of a fiber sequence
        \[
            \cO_{Z'} \to g_*(\cO_Y) \oplus \cO_Z \to g_*(\cO_X),  
        \]
    in the \infcat $\Mod_{\cO_{Z'}}$ together with the fact that $g_*(\cO_Y)$ and $g_*(\cO_Z)$ have coherent homotopy sheaves, by our assumption that $g$ is a
    finite morphism combined with \cref{lem:nil-isos_are_affine_morphisms}.
\end{proof}

\begin{defin} \label{defin:analytic_formal_moduli_problems_under}
    An \emph{analytic formal moduli problem under $X$} corresponds to the datum of a functor
        \[F \colon (\anNil_{X/})\op \to \cS,\]
    satisfying the following two conditions:
    \begin{enumerate}
        \item $F(X) \simeq *$ in $\cS$;
        \item $F \simeq \mathbf{res}^{< \infty} \circ F$, where $\mathbf{res}^{< \infty}_!$ denotes the right Kan extension along the natural inclusion
        \item Given any pushout diagram
            \[\begin{tikzcd}
                S \ar{r}{f} \ar{d} & S' \ar{d} \\
                T \ar{r} & T',
            \end{tikzcd}\]
        in $\anNil_{X/}$ for which $f$ is has the structure of a square zero extension, the induced morphism
            \[F(T') \to F(T) \times_{F(S)}F(S),\]
        is an equivalence in $\cS$.
    \end{enumerate}
    We shall denote by $\anFMP_{X/}$ the full subcategory of $\Fun((\anNil_{X/ })\op, \cS)$ spanned by analytic formal moduli problems
    under $X$.
\end{defin}

\begin{construction}
We have a composite diagram
    \[
        h \colon \anNil_{X/} \to \dAnk \hookrightarrow \anPreStk.
    \]
Therefore, given any analytic pre-stack regarded as a limit-preserving functor $F \colon \anPreStk\op \to \cS$, one can consider its restriction to the \infcat
$\anNil_{X/}$:
    \[
        F \circ h \colon \anNil_{X/} \op \to \cS.      
    \]
We have thus a natural restriction functor
    \[
        h_* \colon \anPreStk \to \Fun(\anNil_{X/} \op, \cS).  
    \]
\end{construction}


\begin{eg}
    Let $X \in \dAnk$. As in the algebraic case, we can consider the \emph{de Rham pre-stack associated to $X$}, $X_\mathrm{dR} \colon \dAfd_k\op \to \cS$,
    determined by the formula
        \[
            X_{\mathrm{dR}}(Z) \coloneqq X(Z_\red), \quad Z \in \dAfd_k.  
        \]
    We have a natural morphism $X \to X_\mathrm{dR}$ induced from the natural morphism $Z_\red \to Z$.
    We claim that $h_*(X_\mathrm{dR}) \in \Fun(\anNil_{X/}\op, \cS)$ belongs to the full subcategory $\anFMP_{X/}$. Indeed, in this case it is clear that
    $h_*(X_\red)$ is the final object in $\anFMP_{X/}$ which clearly satisfies conditions i) and ii) in \cref{defin:analytic_formal_moduli_problems_under}.
\end{eg}

\begin{notation}
    We set $\anNil_{X/}^\cl \subseteq \anNil_{X/}$ to be the full subcategory spanned by those objects corresponding to nil-embeddings of the form
        \[
            X \to S,  
        \]
    in $\dAnk$.
\end{notation}

\begin{prop} \label{prop:analytic_FMP_under_X_are_ind_inf_schemes}
    Let $Y \in \anNil_{X/}$. The following assertions hold:
    \begin{enumerate}
        \item Then the inclusion functor
            \[
              \anNil^{\cl}_{X//Y} \hookrightarrow \anNil_{X//Y} , 
            \]
        is cofinal.
        \item The natural morphism
            \[
               \colim_{Z \in \anNil_{X//Y}^\cl} Z \to Y,  
            \]
        is an equivalence in $\Fun((\anNil_{X//Y})\op, \cS)$.
        \item The \infcat $\anNil_{X//Y}^\cl$ is filtered.
    \end{enumerate}
\end{prop}

\begin{proof}
    We start by proving claim (i). Consider the usual restriction along the natural morphism $X_\red \to X$ functor 
        \[\mathbf{res} \colon \anNil_{X/} \to \anNil_{X_\red/}.\]
    Such functor admits a well defined left adjoint
        \[\mathbf{push} \colon \anNil_{X_\red/} \to \anNil_{X/},\]
    which is determined by the formula
        \[
            (X_\red \to T) \in \anNil_{X_\red /} \mapsto (X \to T') \in \anNil_{X/},   
        \]
    where we set
        \begin{equation} \label{eq:set_definition_of_T'_as_pushout_of_T_along_the_inclusion_X_red_to_X}
            T' \coloneqq X \bigsqcup_{X_\red} T \in \anNil_{X/}.  
        \end{equation}
    We claim that $T' \in \anNil_{X/}$ belongs to the full subcategory $\anNil^\cl_{X/} \subseteq \anNil_{X/}$.
    Indeed, since the structural morphism
        $X_\red \to T$,
    is necessarily a nil-embedding we deduce that the claim follows readily from \cref{prop:existence_of_pushouts_along_closed_nil-isomorphisms}.
    We shall denote
        \[
            \mathbf{res}_!(Y) \colon \anNil\op_{X_\red /} \to \cS,
        \]
    the left Kan extension of $Y$ along the functor $\mathbf{res}$ above. By the colimit formula
    for left Kan extensions, c.f. \cite[Lemma 4.3.2.13]{HTT}, it follows that $\mathbf{res}_!(Y)$ is given by the formula
        \[
            (X_\red \to T) \in  \anNil_{X_\red /} \mapsto Y(T') \in \cS,
        \]
    where $T'$ is as in \eqref{eq:set_definition_of_T'_as_pushout_of_T_along_the_inclusion_X_red_to_X}.
    Let $g \colon X_\red \to T$ in $\anNil_{X_\red/}$ and assume that $g$ factors through the natural morphism $X_\red \to X$. Then we have a natural morphism
        \[
           i_{T, *} \colon Y(T) \to \mathbf{res}_!(Y)(T) ,  
        \]
    in $\cS$, which exhibits the former as a retract of the latter. Denote by
        \[
            p_{T, *} \colon \mathbf{res}_!(Y)(T) \to Y(T),  
        \]
    be a right inverse to $i_{S, *}$. Consider the functor
        \[
            \mathbf{res}_Y \colon \anNil_{X//Y} \to \anNil_{X_\red / /\mathbf{res}_!(Y)},  
        \]
    given by the formula
        \[
            (X \to S \to Y) \in \anNil_{X//Y} \mapsto (X_\red \to S \xrightarrow{f} \mathbf{res}_!(Y)),  
        \]
    where $f \colon S \to \mathbf{res}_!(Y)$ corresponds to the morphism
        \[  
            S_X \xrightarrow{p_S} S \to Y,
        \]
    where $S_X \coloneqq X \bigsqcup_{X_\red} S$. We claim that the functor $\mathbf{res}_Y$ is a right adjoint to the functor
        \[
            \mathbf{push}_Y \colon \anNil_{X_\red // \mathbf{res}_!(Y)} \to \anNil_{X//Y},  
        \]
    the latter given by the formula
        \[
            (X_\red \to T \to \mathbf{res}_!(Y)) \in \anNil_{X_\red // \mathbf{res}_!(Y)} \mapsto (X \to T_X \to Y)  \in \anNil_{X//Y}.
        \]
    Indeed, the datum of a morphism
        \[
            (X_\red \to T \to \mathbf{res}_!(Y)) \to \mathbf{res}_Y(X \to S \to Y),  
        \]
    in $\anNil_{X_\red / \mathbf{res}_!(Y)}$ corresponds to the datum of a commutative diagram
        \[
        \begin{tikzcd}
          X_\red \ar{r} \ar{d} & T \ar{r} \ar{d} & \mathbf{res}_!(Y) \ar{d}{=} \\
          X_\red \ar{r} & S \ar{r} & \mathbf{res}_!(Y),  
        \end{tikzcd}
        \]
    where the right bottom morphism corresponds to the composite $S_X \to S \to Y$. For this reason, the given datum is equivalent to a commutative diagram
        \[
        \begin{tikzcd}
            X_\red \ar{r} \ar{d} & T \ar{r} &  T_X \ar{r} \ar{rd} \ar{d}  & Y \ar{rd}{=} \\
            X_\red \ar{r} & S \ar{r} &  S_X \ar{r} & S \ar{r} & Y,
        \end{tikzcd}
        \]
    which on the other hand is equivalent to the datum of a commutative diagram
        \[
        \begin{tikzcd}
            X \ar{r} \ar{d}{=} & T_X \ar{r} \ar{d} & Y \ar{d}{=} \\
            X \ar{r} & S \ar{r} & Y
        \end{tikzcd}
        \]
    The previous observations combined together then imply that we have a well defined adjunction
        \[
            \mathbf{res} \colon \anNil_{X/ /Y} \rightleftarrows \anNil_{X_\red // \mathbf{res}_!(Y)} \colon \mathrm{push}.
        \]
    We thus conclude that $\anNil_{X//Y} \to \anNil_{X_\red // \mathbf{res}_!(Y)}$ is a cofinal functor (as it admits a left adjoint). Claim (i) now follows
    immediately from the observation that the functor
        \[
            \mathrm{push} \colon \anNil_{X_\red / \mathbf{res}_!(Y)} \to \anNil_{X//Y},  
        \]
    factors through the natural inclusion $\anNil_{X//Y}^\cl \to \anNil_{X//Y}$.
    Claim (ii) follows immediately from (i) combined with Yoneda Lemma. To prove (iii) we shall make use of \cite[Lemma 5.3.1.12]{HTT}. Let
        \[
            F \colon \partial \Delta^n \to \anNil^\cl_{X//Y}.  
        \]
    For each $[m] \in \Delta^{n}$, denote by $S_m \coloneqq F([m]) $ in $\anNil^\cl_{X//Y}$. We then have that the pushout
        \[
            S_n \bigsqcup_X S_{n-1},  
        \]
    exists in $\anNil^\cl_{X/}$. We wish to show that $S_n \bigsqcup_X S_n$ admits a morphism
        \[S_n \bigsqcup_X S_{n-1} \to Y,\]
    compatible with the diagram $F$. In order to show this, we can filter the diagram $F$ by diagrams $F_i \to F$ such that $X \to F_0$ is formed by square-zero
    extensions and so are each $F_i \to F_{i+1}$. Moreover, by the fact that $Y$ satisfies condition (ii) in \cref{defin:analytic_formal_moduli_problems_under}
    it follows that we can find a well defined morphism
        \[
            S_n \bigsqcup_X S_{n-1} \to Y,  
        \]
    which is compatible with $F$, as desired.
\end{proof}


\begin{construction}\label{const:anFMP_as_ind_inf_schemes} Let $X \in \dAnk$. Consider the natural functor
    \[
        F \colon \anNil_{X/} \op \to \dAnk \op.
    \]
Left Kan extension along $F$ induces a functor
    \[
        F_! \colon \Fun( \anNil_{X/ }\op, \cS) \to \Fun  (\dAnk\op, \cS),
    \]
and thus an induced functor
    \[
        F_! \colon \anFMP_{X/} \to \Fun(\dAnk\op, \cS),
    \]
as well. We denote the latter \infcat by $\anPreStk_k$, the \infcat
of \emph{$k$-analytic pre-stacks}. \cref{prop:analytic_FMP_under_X_are_ind_inf_schemes} implies that the functor $F_!$ preserves filtered colimits.
In particular, if we regard $Y$ as a $k$-analytic prestack can be presented
as an \emph{ind}-\emph{inf}-object in the \infcat $\dAnk$, i.e., it can be written as a filtered colimit of nil-embeddings $X \to Z$.
We refer the reader to \cite{Gaitsgory_Study_II} for a precise meaning
of the latter notion in the algebraic setting.
\end{construction}



\begin{defin}
    Let $Y \in \anFMP_{X/}$ denote an analytic formal moduli problem under $X$. The \emph{relative pro-analytic cotangent complex of $Y$ under $X$} is defined as the pro-object
        \[
            \bbL\an_{X/Y} \coloneqq \{ \bbL\an_{X/Z} \}_{Z \in \anNil^\cl_{X//Y}}  \in \pro(\Coh^+(X)),
        \]
    where, for each $Z \in \anNil^\cl_{X//Y}$
        \[\bbL\an_{X/Z} \in \Coh^+(X),\]
    denotes the usual analytic cotangent complex associated to
    the structural morphism $X \to Z$ in $\anNil^\cl_{X//Y}$.
\end{defin}




\begin{rem}
    Let $Y \in \anFMP_{X/}$. Let $Z \in \dAnk$, there exists a natural morphism
        \[
            \bbL\an_X \to \bbL\an_{X/Z} ,  
        \]
    in $\Coh^+(X)$. Passing to the limit over $Z \in \anNil^\cl_{X//Z}$, we obtain a natural map
        \[
            \bbL\an_X \to \bbL\an_{X/Y},  
        \]
    in $\pro(\Coh^+(X))$, as well.
\end{rem}

The following result provides justifies our choice of terminology for the object $\bbL\an_{X/Y} \in \pro(\Coh^+(X))$:

\begin{lem} \label{lem:pro_cot_complex_classifies_nil_extensions_for_analytic_moduli_problems}
    Let $Y \in \anFMP_{X/}$. Let $X \hookrightarrow S$ be a square zero extension associated to an analytic derivation
        \[
            d \colon \bbL\an_S \to \cF ,  
        \]
    where $\cF \in \Coh^+(X)^{\ge 0}$. Then there exists a natural morphism
        \[
            \Map_{\anFMP_{X/}}(S, Y) \to \Map_{\pro(\Coh^+(X))}(\bbL\an_{X/Y}, \cF) \times_{\Map_{\Coh^+(X)}(\bbL\an, \cF)} \{ d \}
        \]
    which is furthermore an equivalence in the \infcat $\cS$.
\end{lem}

\begin{proof}
    Thanks to \cref{prop:analytic_FMP_under_X_are_ind_inf_schemes} we can identify the space of liftings of the map $X \to Y$ along $X \to S$ with the mapping space
        \[
            \Map_{\anFMP_{X/}}(S, Y) \simeq \colim_{Z \in \anNil_{X//Y}} \Map_{\anNil_{X/}} (S, Z).  
        \]
    Fix $Z \in \anNil_{X//Y}$. Then we have a natural identification of mapping spaces
        \begin{align} \label{eq:identification_of_universal_property_of_rel_an_cotagent_complex}
            \Map_{\anNil_{X/}} (S, Z) & \simeq \Map_{(\dAnk)_{X/}} (S, Z) \\
                                      & \simeq \Map_{\Coh^+(X)}(\bbL\an_{X/Z}, \cF) \times_{\Map_{\Coh^+(X)}(\bbL\an_X, \cF)} \{ d \},
        \end{align}
    see \cite[\S 5.4]{Porta_Yu_Representability} for a justification of the latter assertion.
    Passing to the colimit over $Z \in \anNil_{X//Y}^\cl$, we conclude that
        \[
            \Map_{\anFMP_{X/}}(S, Y) \simeq \Map_{\pro(\Coh^+(X))}(\bbL\an_{X/Y}, \cF) \times_{\Map_{\Coh^+(X)}(\bbL\an, \cF)} \{ d \},
        \]
    as desired.
\end{proof}

\begin{construction} \label{rem:morphisms_of_AnFMP_induce_transition_morphisms_on_relative_analytic_cot_complexes}
    Let $f \colon Y \to Z$ denote a morphism in $\anFMP_{X/}$. Then, for every $S \in \anNil_{X//Y}^\cl$ the induced morphism
        \[
            S \to Z,   
        \]
    in $\anFMP_{X/}$ factors through some $S' \in \anNil^\cl_{X/Z}$. For this reason, we obtain a natural morphism
        \[
            \bbL\an_{X/S'} \to \bbL\an_{X/S},  
        \]
    in the \infcat $\Coh^+(X)$. Passing to the limit over $S \in \anNil^\cl_{X//Y}$ we obtain a canonically defined morphism
        \[
            \theta (f) \colon \bbL\an_{X/Z} \to \bbL\an_{X/Y},  
        \]
    in $\pro(\Coh^+(X))$.
\end{construction}

\begin{prop}
    Let $f \colon Y \to Z$ be a morphism in the \infcat $\anFMP_{X/}$. Suppose that $f$ induces an equivalence of relative pro-analytic cotangent complexes via
    \cref{rem:morphisms_of_AnFMP_induce_transition_morphisms_on_relative_analytic_cot_complexes}.
    Then $f$ is itself an equivalence of analytic formal moduli problems under $X$.
\end{prop}

\begin{proof}
    Thanks to \cref{prop:analytic_FMP_under_X_are_ind_inf_schemes} we are reduced to show that given any
        \[
            S \in \anNil_{X//Z}^\cl,  
        \]
    the structural morphism $g_S \colon X \to S$ admits a unique extension $S \to Y$ which factors the structural morphism $X \to Y$. Thanks to
    \cref{prop:filtered_colimit_for_nil-embeddings} we can reduce ourselves to the case where $X \to S$ has the structure of
    a square zero extension. In this case, the result follows from \cref{lem:pro_cot_complex_classifies_nil_extensions_for_analytic_moduli_problems}
    combined with our hypothesis.
\end{proof}

\begin{defin}
    Let $Y \in \anPreStk$, we shall say that $Y$ is \emph{infinitesimally cartesian} if it satisfies \cite[Definition 7.3]{Porta_Yu_Representability}.
\end{defin}

\begin{prop} \label{prop:sufficient_conditions_for_a_prestack_to_be_equiv_to_an_analytic_FMP}
    Let $Y \in \anPreStk_{X/}^{< \infty}$. Assume further that $Y$ is infinitesimally cartesian and it admits a relative pro-cotangent complex, $\bbL_{X/Y}\an \in \pro(\Coh^+(X))$.
    Then $Y$ is equivalent to an analytic formal moduli problem under $X$.
\end{prop}

\begin{proof}
    We must prove that given a pushout diagram
        \[
        \begin{tikzcd}
            S \ar{r}{f} \ar{d}{g} & S' \ar{d} \\
            T \ar{r} & T'  
        \end{tikzcd}
        \]
    in the \infcat $\anNil_{X/}$, where $f$ has the structure of a square-zero extension, then the natural morphism
        \[
            Y(T') \to Y(T) \times_{Y(S)} Y(S'),  
        \]
    is an equivalence in the \infcat $\cS$. Suppose further that $S \hookrightarrow S'$ is associated to some derivation
    $d \colon \bbL\an_S \to \cF$ for some $\cF \in \Coh^+(S)^{\ge 0}$.
    Thanks to \cref{lem:pushouts_of_square_zero_extensions_have_the_structure_of_a_square_zero_extension} we deduce
    that the induced morphism $T \to T'$ admits a structure of a square-zero extension.
    Then, by our assumptions of $Y$ being infinitesimally cartesian and admitting a relative pro-cotangent complex, we have a chain of natural equivalences of the form.
        \begin{align*}
            Y(T') & \simeq \bigsqcup_{f \colon T \to Y} \Map_{T/ } (T', Y) \\
                  & \simeq \bigsqcup_{ f \colon T \to Y} \Map_{\pro(\Coh^+(T))_{ \bbL\an_Z / }}(\bbL\an_{T/Y}, g_*(\cF)) \\
                  & \simeq \bigsqcup_{f \colon T \to Y} \Map_{\pro(\Coh^+(S))_{g^* \bbL\an_T / }} (g^*\bbL\an_{T/Y}, \cF) \\
                  & \simeq \bigsqcup_{f \colon T \to Y } \Map_{\pro(\Coh^+(S))_{\bbL\an_S/}} (\bbL\an_{S, Y}, \cF) \\
                  & \simeq \bigsqcup_{f \colon T \to Y} \Map_{S/ }(S', Y) \\
                  & \simeq Y(T) \times_{Y(S)} Y(S'),
        \end{align*}
    where the third equivalence follows from the existence of a commutative diagram between fiber sequences
        \[
        \begin{tikzcd}    
            g^* f^*\bbL\an_Y \ar{r} \ar{d}{=} & g^* \bbL\an_{T} \ar{r} \ar{d} &  g^* \bbL\an_{T/Y} \ar{d} \\
            (f\circ g)^*\bbL\an_Y \ar{r} & \bbL\an_{S} \ar{r} & \bbL\an_{S/Y},
        \end{tikzcd}
        \]
    in the \infcat $\pro(\Coh^+(S))$ combined with the fact that the derivation $d_T \colon \bbL\an_T \to g_*(\cF)$ is induced from
        \[
            d \colon \bbL\an_S \to \cF,
        \]
    as in the proof of \cref{lem:pushouts_of_square_zero_extensions_have_the_structure_of_a_square_zero_extension}. The result now follows.
\end{proof}


\subsection{Analytic formal moduli problems over a base}
Let $X \in \dAnk$ denote a derived $k$-analytic space. In \cite[Definition 6.11]{Porta_Yu_NQK}
the authors introduced the \infcat of \emph{analytic formal moduli problems over $X$}, which we shall denote by $\anFMP_{/X}$.

\begin{notation}
    Let $X \in \dAnk$. We shall denote by $\anNil_{/X}$ the full subcategory of $(\dAnk)_{/X}$ spanned by nil-isomorphisms
        \[
            Z \to X.  
        \]
\end{notation}

\begin{defin}
    We shall denote by $\anNil_{/X}^\cl \subseteq \anNil_{/X}$ the faithful subcategory in which we only allow morphisms
        \[
            S \to S'  
        \]
    in $\anNil_{/X}$ which are closed nil-isomorphisms.
\end{defin}

We start with the analogue of \cref{prop:analytic_FMP_under_X_are_ind_inf_schemes} in the setting of analytic formal moduli problems over $X$:

\begin{prop} \label{prop:required_conditions_for_formal_moduli_problems}
    Let $Y \in \anFMP_{/X}$. The following assertions hold:
    \begin{enumerate}
        \item The inclusion functor
            \[
                (\anNil^{\cl}_{/X})_{/Y}  \to (\anNil^{\cl}_{/X})_{/Y},
            \]
        is cofinal.
        \item The natural morphism
            \[
                \colim_{Z \in (\anNil^\cl_{/X})_{/Y}}  Z \to Y,
            \]
        is an equivalence in the \infcat $\anFMP_{/X}$.
        \item The \infcat $\anNil_{/X}^\cl$ is filtered.
    \end{enumerate}
\end{prop}

\begin{proof}
    We first prove assertion (i). Let $S \to Z$ be a morphism in $(\anNil^\cl_{/X})_{/Y}$. Consider the pushout diagram
        \begin{equation} \label{eq:diagram_pushout_of_nil_isomorphisms_with_ltv_being_the_reduced_subspace}
        \begin{tikzcd}
            S_\red \ar{r} \ar{d} & S \ar{d} \\
            Z \ar{r} & Z',
        \end{tikzcd}
        \end{equation}
    in the \infcat $\anNil_{/X}$ whose existence is guaranteed by \cref{prop:existence_of_pushouts_along_closed_nil-isomorphisms}. Since
    the upper horizontal morphism in \eqref{eq:diagram_pushout_of_nil_isomorphisms_with_ltv_being_the_reduced_subspace} is a closed nil-isomorphism, we can reduce
    ourselves to the case where the latter is an actual square-zero extension. Indeed, the latter assertion follows by
    arguing by induction combined with \cref{prop:filtered_colimit_for_nil-embeddings}.
    Since $Y$ is assumed to be an analytic formal moduli problem over $X$ we then deduce that the canonical morphism
        \begin{align*}
            Y(Z') & \to Y(Z) \times_{Y(S_\red)} Y(S) \\
            &\simeq Y(Z) \times Y(S),  
        \end{align*}
    is an equivalence (we implicitly used above the fact that $S_\red \simeq X_\red$). As a consequence the object $(Z' \to X)$ in $\anNil_{/X}$ admits an induced morphism
    $Z' \to Y$ making the required diagram commute.
    Thanks \cref{prop:existence_of_pushouts_along_closed_nil-isomorphisms} we deduce that both
    $S \to Z'$ and $Z \to Z'$ are closed nil-isomorphisms. Therefore, we can factor the diagram
        \[
        \begin{tikzcd}
            S \ar{rr} \ar{rd} & & Z \ar{dl} \\
                &               Y       &  
        \end{tikzcd}
        \]
    via a closed nil-isomorphism $Z \to Z'$. We conclude that the inclusion functor $(\anNil^\cl_{/X})_{/Y} \to (\anNil_{/X})_{/Y}$ is cofinal.
    It is clear that assertion (ii) follows immediately from (i). We now prove (iii). Let 
        \[\theta \colon K \to (\anNil^\cl_{/X})_{/Y},\]
    be a functor where
    $K$ is a finite \infcat. We must show that $\theta$ can be extended to a functor
        \[\theta^{\rhd} \colon K^{\rhd} \to (\anNil^\cl_{/X})_{/Y}.\]
    Thanks to \cref{prop:filtered_colimit_for_nil-embeddings} we are allowed to reduce ourselves to the case where morphisms indexed by $K$
    are square-zero extensions. The result now follows from the fact that $Y$ being an analytic moduli problem sends finite colimits along square-zero extensions
    to finite limits.
\end{proof}

Just as in the previous section we deduce that every analytic formal moduli problem over $X$ admits the structure of an \emph{ind}-\emph{inf}-object
in $\anPreStk_k$:

\begin{cor} \label{cor:formal_moduli_problems_over_X_are_ind_inf_objects}
    Let $Y \in (\anPreStk_k)_{/X}$. Then $Y$ is equivalent to an analytic formal moduli problem over $X$ if and only if there exists
    a presentation $Y \colim_{i \in I} Z_i$, where $I$ is a filtered \infcat and for every $i \to j$ in $I$, the induced morphism
        \[
          Z_i \to Z_j,  
        \]
    is a closed embedding of derived $k$-affinoid spaces that are nil-isomorphic to $X$.
\end{cor}

\begin{proof}
    It follows immediately from \cref{prop:required_conditions_for_formal_moduli_problems} (ii).
\end{proof}

\begin{defin}
    Let $Y \in \anFMP_{/X}$. We define the \infcat of \emph{coherent modules on $Y$}, denoted $\Coh^+(Y)$, as the limit
        \[
            \Coh^+(Y) \coloneqq \lim_{Z \in (\dAnk)_{/Y}}  \Coh^+(Z),
        \]
    computed in the \infcat $\Catst$. We define the \infcat of \emph{pro-coherent modules on $Y$}, denoted $\pro(\Coh^+(Y))$, as
        \[
            \pro(\Coh^+(Y)) \coloneqq \lim_{Z \in (\dAnk)_{/Y}} \pro(\Coh^+(Z)),  
        \]
    where the limit is computed in the \infcat $\Catst$.
\end{defin}

\begin{defin} Let $Y \in \anFMP_{/X}$, $Z \in \dAfd_k$ and let $\cF \in \Coh^+(Z)^{\ge 0}$. Suppose furthermore that we are given a morphism $f \colon Z \to Y$.
    We define the \emph{tangent space of $Y$ at $f$ twisted by $\cF$} as the fiber
        \[
            \bbT\an_{Y, Z, \cF,f} \coloneqq \mathrm{fib}_f\big( Y(Z[\cF]) \to Y(Z) \big) 
            \in \cS.  
        \]
    Whenever the morphism $f$ is clear from the context, we shall drop the subscript $f$ above and denote the tangent space of $Y$ at $f$ simply by $\bbT\an_{Y, Z, \cF}$.
\end{defin}


\begin{rem} Let $Y \in \anFMP_{/X}$.
    The equivalence of ind-objects
        \[
            Y \simeq \colim_{S \in (\anNil^\cl_{/X})_{/Y}} S,
        \]
    in the \infcat $\dAnk$, implies that, for any $Z \in \dAfd_k$, one has an equivalence of mapping spaces
        \[
            \Map_{\anPreStk}(Z, Y) \simeq \colim_{S \in (\anNil^\cl_{/X})_{/Y}}  \Map_{\anPreStk}(Z,S ).
        \]
    For this reason, given any morphism $f \colon Z \to Y$ and any
    $\cF \in \Coh^+(Z)^{\ge 0}$, we can identify the tangent space $\bbT\an_{Y, Z, \cF}$
    with the filtered colimit of spaces
        \begin{align} \label{eq:description_of_cotangent_complex_of_a_formal_moduli_problem_in_terms_of_its_tangent_space}
            \bbT\an_{Y, Z, \cF} & \simeq \colim_{S \in (\anNil^\cl_{/X})_{Z/ /Y}} \fib_f \big( S(Z[\cF]) \to S(Z) \big) \\
                                & \simeq \colim_{S \in (\anNil^\cl_{/X})_{Z/ /Y}} \bbT\an_{S, Z , \cF } \\
                                & \simeq \colim_{S \in (\anNil^\cl_{/X})_{Z/ /Y}} \Map_{\Coh^+(Z)} (f_{S,Z}^* (\bbL\an_S), \cF),
        \end{align}
    where we have denoted by $f_{S, Z} \colon Z \to S$ any morphism, in $(\dAnk)_{/X}$, factoring $f \colon Z \to Y$.
    Moreover, the latter equivalence follows readily from \cite[Lemma 7.7]{Porta_Yu_Representability}. Therefore, we deduce that
    the analytic formal moduli problem
    $Y \in \anFMP_{/X}$ admits an \emph{absolute pro-cotangent complex} given as
        \[
           \bbL\an_Y \coloneqq \{ f^*_{S,Z}( \bbL\an_S )  \}_{Z \in (\dAnk)_{/Y}, S \in (\anNil^\cl_{/X})_{/Y}} \in \pro(\Coh^+(Y)).
        \]
\end{rem}   

\begin{cor}
    Let $Y \in \anFMP_{/X}$. Then its absolute cotangent complex $\bbL\an_Y$ classifies analytic deformations on $Y$. More precisely, given $Z \to Y$ a morphism
    where $Z \in \dAfd_k$ and $\cF \in \Coh^+(Z)^{\ge 0}$ one has a natural equivalence of mapping spaces
        \[
            \bbT\an_{Y, Z, \cF} \simeq \Map_{\pro(\Coh^+(Y))}(\bbL\an_Y, \cF).  
        \]
\end{cor}

\begin{proof}
    It follows immediately from the natural equivalences displayed in \eqref{eq:description_of_cotangent_complex_of_a_formal_moduli_problem_in_terms_of_its_tangent_space}
    combined with the description of mapping spaces in \infcats of pro-objects.
\end{proof}


\subsection{Non-archimedean nil-descent for almost perfect complexes}
In this \S, we prove that the \infcat $\Coh^+(X)$, for $X \in \dAnk$ satisfies nil-descent with respect to morphims $Y \to X$, which exhibit
the former as an analytic formal moduli problem over $X$.

\begin{prop} \label{prop:nil_descent_for_Coh^+}
    Let $f \colon Y \to X$, where $X \in \dAnk$ and $Y \in \anFMP_{/X}$. Consider the \v{C}ech nerve
    $Y^\bullet \colon \mathbf \Delta \op \to \anPreStk$ associated to $f$.
    Then the natural functor
        \[
            f_\bullet^* \colon \Coh^+(X) \to \mathrm{Tot}(\Coh^+(Y^\bullet)),  
        \]
    is an equivalence of \infcats.
\end{prop}

\begin{proof}
    Consider the natural equivalence of $k$-analytic prestacks
        \[
            Y \simeq \colim_{Z \in (\anNil^\cl_{/X})_{/Y}}  Z.
        \]
    Then, by definition one has a natural equivalence
        \[
            \Coh^+(Y) \simeq \lim_{Z \in (\anNil^\cl_{/X})_{/Y}} \Coh^+(Z),  
        \]
    of \infcats. In particular, since totalizations commute with cofiltered limits in $\Catinf$, it follows that we can suppose from
    the beginning that $Y \simeq Z$ for some $Z \in \anNil_{/X}$. In this case, the morphism $f \colon Y \to X$ is affine. In particular, the fact that
    $\Coh^+(-)$ satisfies Zariski descent combined with \cref{lem:affine_morphisms_are_compatible_with_Zariski_localization_on_the_base} we further reduce ourselves
    to the case where we might assume both $X$ and $Y$ to be both equivalent to derived $k$-affinoid spaces. In this case, by Tate acyclicity
    theorem it follows that letting $A \coloneqq \Gamma(X, \cO_X \alg)$ and $B \coloneqq \Gamma(Y, \cO_Y\alg)$ the pullback functor $f^*$ can be identified with
    the base change functor
        \[
            \Coh^+(A) \to \Coh^+(B).  
        \]
    In this case, it follows that $B$ is nil-isomophic to $A$. Moreover, since the latter are derived noetherian rings
    the statement of the proposition follows due to \cite[Theorem 3.3.1]{preygel_Leistner_Mapping_stacks_properness}.
\end{proof}

\begin{cor}
    Let $X \in \dAnk$ and $f \colon Y \to X$ a morphism in $\anPreStk$ which exhibits $Y$ as an analytic formal moduli problem over $X$.
    Then the natural functor 
        \[
            f_\bullet^* \colon \pro(\Coh^+ (X)) \to \mathrm{Tot}(\pro(\Coh^+(Y^\bullet/X))),
        \]
    is an equivalence of \infcats, where $Y^\bullet$ denotes the \v{C}ech nerve of the morphism $f$.
\end{cor}

\begin{proof}
    By the very definition of the \infcat $\pro(\Coh^+(Y))$, we reduce ourselves as in
    \cref{prop:nil_descent_for_Coh^+} to the case where $Y = S$, for some $S \in \anNil_{/X}$. In this case, it follows readily from
    \cref{prop:nil_descent_for_Coh^+} that the natural functor
        \[
            f_\bullet^* \colon \pro(\Coh^+(X)) \to \mathrm{Tot}(\pro(\Coh^+(Y^\bullet/X))),  
        \]
    is fully faithful. \cref{lem:f^*_admits_a_right_adjoint_whenever_f_is_nil-iso} that we have a well defined right adjoint
        \[
            f_* \colon \Coh^+(S) \to \Coh^+(X),  
        \]
    to the usual pullback functor $f^* \colon \Coh^+(X) \to \Coh^+(S)$. We can extend the right adjoint $f_*$ to a well defined functor
        \[
            f_* \colon \pro(\Coh^+(S)) \to \pro(\Coh^+(X)),  
        \]
    which commutes with cofiltered limits. For this reason, we have a well defined functor
        \[
            \lim_{[n] \in \bDelta \op} f_{\bullet, *} \colon \mathrm{Tot}(\pro(\Coh^+(Y^\bullet/X))) \to \pro(\Coh^+(X)),  
        \]
    which further commutes with filtered limits. We claim that $\lim_{[n] \in \bDelta \op} f_{\bullet, *}$ is a right adjoint to $f_\bullet^*$ above. Indeed, given
    any $\{ \cF_i \}_{i \in I\op}  \in \pro(\Coh^+(X))$ and $\{ \cG_{j, [n]} \}_{j \in J_{[n]} \op, [n] \in \bDelta \op} \in \mathrm{Tot}(\pro(\Coh^+(Y^\bullet/X)))$,
    we compute
        \begin{align*}
            \Map_{\mathrm{Tot}(\pro(\Coh^+(Y^\bullet/X)))}( f_\bullet^*( \{ \cF_i \}_{i \in I\op}), \{ \cG_{j, [n]} \}_{j \in J\op_{[n]}, [n] \in \bDelta \op}) & \simeq \lim_{[n] \in \bDelta \op} \Map_{\pro(\Coh^+(Y^{[n]}))} ( \{f_{[n]}^\bullet(\cF_i) \}_{i \in I\op},  \{ \cG_{i, [n]} \}_{i \in I\op_{[n]}, [n]}) \\
            \lim_{[n] \in \bDelta \op} \lim_{j \in J\op_{[n]}} \mathrm{colim}_{i \in I} \Map_{\Coh^+(Y^{[n]})} ( f_{[n]}^\bullet(\cF_i),   \cG_{i, [n]}) & \simeq \lim_{[n] \in \bDelta \op} \lim_{j \in J\op_{[n]}} \colim_{i \in I} \Map_{\Coh^+(X)} (\cF_i, f_{[n], *}(\cG_{i, [n]})) \\
            \lim_{[n] \in \bDelta \op} \Map_{\pro(\Coh^+(X))}( \{\cF_i\}_{i \in I\op}, \{f_{[n], *}(\cG_{i, [n]}) \}_{i \in I_{[n]} \op}) & \simeq \Map_{\pro(\Coh^+(X))}( \{\cF_i\}_{i \in I\op}, \lim_{[n] \in \bDelta \op} \{f_{[n], *}(\cG_{i, [n]}) \}_{i \in I_{[n]} \op}),
        \end{align*}
    as desired. In order to conclude, we will show that the functor
        \[
            \lim_{[n] \in \bDelta \op} f_{\bullet, *} \colon \pro(\Coh^+(X)) \to \mathrm{Tot}(\pro(\Coh^+(Y^\bullet/X))),  
        \]
    is conservative. Since both the \infcats $\mathrm{Tot}(\pro(\Coh^+(X)))$ and $\mathrm{Tot}(\pro(\Coh^+(Y^\bullet/X)))$ are stable, we are reduced to prove that given any
        \[
            \{ \cG_{i, [n]} \}_{i \in I\op, [n]} \in \mathrm{Tot}(\pro(\Coh^+(Y^\bullet/X))),  
        \]
    such that 
        \begin{equation} \label{eq:conservativity_of_lim_f_bullet_*}
            \lim_{[n] \in \bDelta \op} f_{\bullet, *}( \{ \cG_{i, [n]} \}_{i \in I\op, [n]})  \simeq 0,
        \end{equation}
    then we necessarily have
        \[
            \{ \cG_{i, [n]} \}_{i \in I\op, [n]} \simeq 0  .
        \]
    Assume then \cref{eq:conservativity_of_lim_f_bullet_*}. Then, given any 
\end{proof}

\subsection{Non-archimedean formal groupoids}
Let $X \in \dAfd_k$ denote a derived $k$-affinoid space. We denote by
$\mathrm{AnFGrpd}(X)$ the full subcategory of the \infcat of simplicial objects
    \[
        \Fun( \bDelta \op, \anFMP_{/X}),
    \]
spanned by those objects $F \colon \mathbf \Delta \op \to \anFMP_{/X}$ satisfiying the following requirements:
    \begin{enumerate}
        \item $F([0]) \simeq X$ ;
        \item For each $n \ge 1$, the morphism
            \[
                F([n]) \to F([1]) \times_{F([0])} \dots \times_{F([0])} F([1])  ,
            \]
        induced by the morphisms $s^i \colon [1] \to [n]$ given by $(0,1) \mapsto (i, i+1)$, is an equivalence
        in $\anFMP_{/X}$.
    \end{enumerate}
\todo{Put the above as a definition + introduce analytic formal moduli problems over.}

\begin{lem} \label{formal_moduli_under_induce_formal_moduli_over_via_base_change}
    Let $X \in \dAnk$. Given any $Y \in \anFMP_{X/ }$, then for each $i= 0, 1$ the $i$-th projection morphism
        \[
            p_0 \colon X \times_Y X \to X,  
        \]
    computed in the \infcat $\anPreStk_k$ lies in the essential image of $\anFMP_{/X}$ via \cref{const:anFMP_as_ind_inf_schemes}.
\end{lem}

\begin{proof}
    Consider the pullback diagram
        \[
        \begin{tikzcd}
            X \times_Y X \ar{r}{p_1} \ar{d}{p_0} & X \ar{d} \\
            X \ar{r} & Y ,
        \end{tikzcd}
        \]
    computed in the \infcat $\anPreStk$. Thanks to \cref{prop:analytic_FMP_under_X_are_ind_inf_schemes} together with the fact that fiber products commute with filtered colimis in the \infcat $\anPreStk_k$,
    we deduce that
        \[
            X \times_Y X \simeq \colim_{Z \in \anNil^\cl_{X//Y}} X \times_Z X, 
        \]
    in $\anPreStk_k$. It is clear that $(p_i \colon X \times_Z X \to X)$ lies in the essential image of $ \anFMP_{/X}$, for $i = 0, 1$. Thus also the filtered colimit
        \[
            (p_i \colon X \times_Y X ) \in \anFMP_{/X}, \quad \mathrm{for \ } i = 0, 1,
        \]  
    as desired.
\end{proof}


\begin{construction} Thanks to \cref{formal_moduli_under_induce_formal_moduli_over_via_base_change},
there exists a well defined functor $\Phi \colon \anFMP_{X/} \to \anFGrpd(X)$ given by the formula
    \[
        (X \to Y ) \in \anFMP_{X/ } \mapsto Y^\wedge_X \in \anFGrpd(X),
    \]
where $Y^\wedge_X \in \anFGrpd(X)$ denotes the analytic formal groupoid over $X$ whose presentation is given by
    \[
    \begin{tikzcd}
      \dots \ar[r, shift left=2] \ar[r, shift left=0.75] 
      \ar[r, shift left=-0.75] \ar[r, shift left=-2]
      & X \times_Y X \times_Y X \ar[r, shift left=-1] \ar[r, shift left=1] \ar[r] 
      & X \times_Y X \ar[r, shift left=-0.25ex] \ar[r, shift left=0.25ex] 
      & X 
    \end{tikzcd}.
    \] 
Moreover, given any $\cG \in \anFGrpd(X)$, we can associate it an analytic formal moduli problem under $X$,
denoted $\rB_{X}(\cG)$, as follows: let $X \to S$ be an object
in $\anNil_{X/}$, then we let
    \[
        \rB_{X}(\cG) (S) \coloneqq  \big\{ (\tS \to S) \in \anFMP_{/S}, \tS \to X, \mathrm{a \ morphism \ of \ groupoid \ objects \ }\tS \times_S \tS \to \cG \mathrm{ \ satisfying \  (*) 
        } \big\}
    \]
where condition (*) is determined by requiring that the commutative squares
    \[
    \begin{tikzcd}
        \tS \times_S \tS \ar{r} \ar{d}{p_i} & \cG \ar{d}{p_i} \\
        \tS \ar{r}   & X
    \end{tikzcd}
    \]
for $i=0, 1$ are cartesian. Such association is functorial in $(X \to S) \in \anNil_{X/}$ and thus it defines a well defined functor
    \[
        \rB_{X}(\cG) \colon \anNil_{X/ }\op \to \cS.  
    \]
\end{construction}

\begin{rem}
    Let $X \in \dAnk$ and $\cG \in \anFGrpd(X)$. There exists a canonical morphism $X \to \rB_X(\cG)$ given by associating every 
        \[
            Z \in \dAfd_k
        \]
\end{rem}

\begin{lem}
    The functor $\rB_X(\cG) \colon \anNil_{X/}\op \to \cS$ is equivalent to an analytic formal moduli problem.
\end{lem}

\begin{proof}
    Thanks to \cref{prop:sufficient_conditions_for_a_prestack_to_be_equiv_to_an_analytic_FMP} it suffices to prove that $\rB_X(\cG)$ is
    infinitesimally cartesian and it admits furthermore a pro-cotagent complex. Infinitesimally cartesian follows from the modular description
    of $\rB_X(\cG)$ combined with the fact that $\cG$ is infinitesimally cartesian, as well. We are thus required to show that $\rB_X(\cG)$ admits
    a \emph{global} pro-cotangent complex. 
\end{proof}


\bibliographystyle{plain}
\bibliography{dahema}

\end{document}