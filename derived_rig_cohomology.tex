\pdfoutput=1
%The other issue is that some packages, such as microtype, produce different output under pdflatex. By default the arXiv goes from dvi to ps to pdf, so if you need pdflatex you have to set the \pdfoutput flag in the TeX file.
\newif\ifpersonal
\personaltrue % comment to remove personal notes
\RequirePackage[l2tabu,orthodox]{nag} %detect whether obsolete packages are used
\documentclass[10pt,a4paper,reqno]{amsart} %reqno places equation numbers on the right
\linespread{1.1}
\usepackage{amsmath,amsthm,amssymb,mathrsfs,mathtools,bm,eucal,tensor} % math related
\usepackage{microtype,fixltx2e,lmodern} % latex technical issues
\usepackage[utf8]{inputenc} % input encoding
\usepackage[T1]{fontenc} % font encoding
\usepackage{enumerate,comment,braket,xspace,tikz-cd,csquotes} % utilities
\usepackage[all,cmtip]{xy} % because the tikzcd options [shift left], [shift right] do not work on arXiv, we switched some diagrams to xymatrix
\usepackage[centering,vscale=0.7,hscale=0.8]{geometry}
%\usepackage[right]{showlabels}
\usepackage[hidelinks]{hyperref}
\usepackage[capitalize]{cleveref}

\theoremstyle{plain}
\newtheorem{thm-intro}{Theorem}
\newtheorem{thm}{Theorem}[section]
\newtheorem*{thm*}{Theorem}
\newtheorem{claim}[thm]{Claim}
\newtheorem{lem}[thm]{Lemma}
\newtheorem{prop}[thm]{Proposition}
\newtheorem{conj}[thm]{Conjecture}
\newtheorem{cor}[thm]{Corollary}
\newtheorem{assumption}[thm]{Assumption}
\theoremstyle{definition}
\newtheorem{defin}[thm]{Definition}
\newtheorem{notation}[thm]{Notation}
\newtheorem{eg}[thm]{Example}
\newtheorem{variant}[thm]{Variant}
\newtheorem{warning}[thm]{Warning}
\theoremstyle{remark}
\newtheorem{rem}[thm]{Remark}
\numberwithin{equation}{section}
\newtheorem{construction}[thm]{Construction}

% personal remarks

\ifpersonal
\newcommand{\personal}[1]{\textcolor[rgb]{0,0,1}{(Personal: #1)}}
\newcommand{\todo}[1]{\textcolor{red}{(Todo: #1)}}
\else
\newcommand*{\personal}[1]{\ignorespaces}
\newcommand*{\todo}[1]{\ignorespaces}
\fi

% Fonts
\newcommand{\C}{\mathbb C}
\newcommand{\CP}{\mathbb{CP}}
\newcommand{\F}{\mathbb F}
\newcommand{\Q}{\mathbb Q}
\newcommand{\R}{\mathbb R}
\newcommand{\Z}{\mathbb Z}

\newcommand{\rB}{\mathrm B}
\newcommand{\rH}{\mathrm H}
\newcommand{\rL}{\mathrm L}
\newcommand{\rR}{\mathrm R}
\newcommand{\rT}{\mathrm T}
\newcommand{\rb}{\mathrm b}
\newcommand{\rd}{\mathrm d}
\newcommand{\rI}{\mathrm I}
\newcommand{\rs}{\mathrm s}
\newcommand{\rt}{\mathrm t}


\newcommand{\fA}{\mathfrak A}
\newcommand{\fB}{\mathfrak B}
\newcommand{\fC}{\mathfrak C}
\newcommand{\fD}{\mathfrak D}
\newcommand{\fF}{\mathfrak F}
\newcommand{\fG}{\mathfrak G}
\newcommand{\fH}{\mathfrak H}
\newcommand{\fS}{\mathfrak S}
\newcommand{\fT}{\mathfrak T}
\newcommand{\fU}{\mathfrak U}
\newcommand{\fV}{\mathfrak V}
\newcommand{\fX}{\mathfrak X}
\newcommand{\fY}{\mathfrak Y}
\newcommand{\fW}{\mathfrak W}
\newcommand{\fZ}{\mathfrak Z}
\newcommand{\fa}{\mathfrak a}
\newcommand{\fb}{\mathfrak b}
\newcommand{\ff}{\mathfrak f}
\newcommand{\fm}{\mathfrak m}
\newcommand{\fs}{\mathfrak s}
\newcommand{\ft}{\mathfrak t}

\newcommand{\cA}{\mathcal A}
\newcommand{\cB}{\mathcal B}
\newcommand{\cC}{\mathcal C}
\newcommand{\cD}{\mathcal D}
\newcommand{\cE}{\mathcal E}
\newcommand{\cF}{\mathcal F}
\newcommand{\cH}{\mathcal H}
\newcommand{\cG}{\mathcal G}
\newcommand{\cI}{\mathcal I}
\newcommand{\cJ}{\mathcal J}
\newcommand{\cK}{\mathcal K}
\newcommand{\cL}{\mathcal L}
\newcommand{\cM}{\mathcal M}
\newcommand{\cN}{\mathcal N}
\newcommand{\cO}{\mathcal O}
\newcommand{\cP}{\mathcal P}
\newcommand{\cQ}{\mathcal Q}
\newcommand{\cR}{\mathcal R}
\newcommand{\cS}{\mathcal S}
\newcommand{\cT}{\mathcal T}
\newcommand{\cU}{\mathcal U}
\newcommand{\cV}{\mathcal V}
\newcommand{\cW}{\mathcal W}
\newcommand{\cX}{\mathcal X}
\newcommand{\cY}{\mathcal Y}
\newcommand{\cZ}{\mathcal Z}
\DeclareFontFamily{U}{BOONDOX-calo}{\skewchar\font=45 }
\DeclareFontShape{U}{BOONDOX-calo}{m}{n}{<-> s*[1.05] BOONDOX-r-calo}{}
\DeclareFontShape{U}{BOONDOX-calo}{b}{n}{<-> s*[1.05] BOONDOX-b-calo}{}
\DeclareMathAlphabet{\mathcalboondox}{U}{BOONDOX-calo}{m}{n}
%\DeclareMathAlphabet{\mathcalligra}{T1}{calligra}{m}{n}
\newcommand{\cf}{\mathcalboondox f}

\newcommand{\bbA}{\mathbb A}
\newcommand{\bbD}{\mathbb D}
\newcommand{\bbE}{\mathbb E}
\newcommand{\bbG}{\mathbb G}
\newcommand{\bbL}{\mathbb L}
\newcommand{\bbN}{\mathbb N}
\newcommand{\bbP}{\mathbb P}
\newcommand{\bbT}{\mathbb T}
\newcommand{\bbZ}{\mathbb Z}

\newcommand{\bA}{\mathbf A}
\newcommand{\bD}{\mathbf D}
\newcommand{\bP}{\mathbf P}
\newcommand{\bQ}{\mathbf Q}
\newcommand{\bT}{\mathbf T}
\newcommand{\bX}{\mathbf X}
\newcommand{\bY}{\mathbf Y}
\newcommand{\be}{\mathbf e}
\newcommand{\br}{\mathbf r}
\newcommand{\bu}{\mathbf u}
\newcommand{\balpha}{\bm{\alpha}}
\newcommand{\bDelta}{\bm{\Delta}}
\newcommand{\brho}{\bm{\rho}}

\newcommand{\sC}{\mathscr C}
\newcommand{\sX}{\mathscr X}
\newcommand{\sD}{\mathscr D}
\newcommand{\sU}{\mathscr U}


% Decorations

% Definition of \widebar from http://tex.stackexchange.com/questions/16337/can-i-get-a-widebar-without-using-the-mathabx-package/60253#60253
\makeatletter
\let\save@mathaccent\mathaccent
\newcommand*\if@single[3]{%
	\setbox0\hbox{${\mathaccent"0362{#1}}^H$}%
	\setbox2\hbox{${\mathaccent"0362{\kern0pt#1}}^H$}%
	\ifdim\ht0=\ht2 #3\else #2\fi
}
%The bar will be moved to the right by a half of \macc@kerna, which is computed by amsmath:
\newcommand*\rel@kern[1]{\kern#1\dimexpr\macc@kerna}
%If there's a superscript following the bar, then no negative kern may follow the bar;
%an additional {} makes sure that the superscript is high enough in this case:
\newcommand*\widebar[1]{\@ifnextchar^{{\wide@bar{#1}{0}}}{\wide@bar{#1}{1}}}
%Use a separate algorithm for single symbols:
\newcommand*\wide@bar[2]{\if@single{#1}{\wide@bar@{#1}{#2}{1}}{\wide@bar@{#1}{#2}{2}}}
\newcommand*\wide@bar@[3]{%
	\begingroup
	\def\mathaccent##1##2{%
		%Enable nesting of accents:
		\let\mathaccent\save@mathaccent
		%If there's more than a single symbol, use the first character instead (see below):
		\if#32 \let\macc@nucleus\first@char \fi
		%Determine the italic correction:
		\setbox\z@\hbox{$\macc@style{\macc@nucleus}_{}$}%
		\setbox\tw@\hbox{$\macc@style{\macc@nucleus}{}_{}$}%
		\dimen@\wd\tw@
		\advance\dimen@-\wd\z@
		%Now \dimen@ is the italic correction of the symbol.
		\divide\dimen@ 3
		\@tempdima\wd\tw@
		\advance\@tempdima-\scriptspace
		%Now \@tempdima is the width of the symbol.
		\divide\@tempdima 10
		\advance\dimen@-\@tempdima
		%Now \dimen@ = (italic correction / 3) - (Breite / 10)
		\ifdim\dimen@>\z@ \dimen@0pt\fi
		%The bar will be shortened in the case \dimen@<0 !
		\rel@kern{0.6}\kern-\dimen@
		\if#31
		\overline{\rel@kern{-0.6}\kern\dimen@\macc@nucleus\rel@kern{0.4}\kern\dimen@}%
		\advance\dimen@0.4\dimexpr\macc@kerna
		%Place the combined final kern (-\dimen@) if it is >0 or if a superscript follows:
		\let\final@kern#2%
		\ifdim\dimen@<\z@ \let\final@kern1\fi
		\if\final@kern1 \kern-\dimen@\fi
		\else
		\overline{\rel@kern{-0.6}\kern\dimen@#1}%
		\fi
	}%
	\macc@depth\@ne
	\let\math@bgroup\@empty \let\math@egroup\macc@set@skewchar
	\mathsurround\z@ \frozen@everymath{\mathgroup\macc@group\relax}%
	\macc@set@skewchar\relax
	\let\mathaccentV\macc@nested@a
	%The following initialises \macc@kerna and calls \mathaccent:
	\if#31
	\macc@nested@a\relax111{#1}%
	\else
	%If the argument consists of more than one symbol, and if the first token is
	%a letter, use that letter for the computations:
	\def\gobble@till@marker##1\endmarker{}%
	\futurelet\first@char\gobble@till@marker#1\endmarker
	\ifcat\noexpand\first@char A\else
	\def\first@char{}%
	\fi
	\macc@nested@a\relax111{\first@char}%
	\fi
	\endgroup
}
\makeatother


\newcommand{\oDelta}{\widebar\Delta}
\newcommand{\oGamma}{\widebar\Gamma}
\newcommand{\oSigma}{\widebar\Sigma}
\newcommand{\oalpha}{\widebar\alpha}
\newcommand{\obeta}{\widebar\beta}
\newcommand{\otau}{\widebar\tau}
\newcommand{\oC}{\widebar C}
\newcommand{\oD}{\widebar D}
\newcommand{\oE}{\widebar E}
\newcommand{\oG}{\widebar G}
\newcommand{\oM}{\widebar M}
\newcommand{\oS}{\widebar S}
\newcommand{\oU}{\widebar U}
\newcommand{\oW}{\widebar W}
\newcommand{\oX}{\widebar X}
\newcommand{\oY}{\widebar Y}


\newcommand{\ok}{\widebar k}
\newcommand{\ov}{\widebar v}
\newcommand{\ox}{\widebar x}
\newcommand{\oy}{\widebar y}
\newcommand{\oz}{\widebar z}

\newcommand{\hh}{\widehat h}
\newcommand{\hC}{\widehat C}
\newcommand{\hE}{\widehat E}
\newcommand{\hF}{\widehat F}
\newcommand{\hI}{\widehat I}
\newcommand{\hL}{\widehat L}
\newcommand{\hU}{\widehat U}
\newcommand{\hbeta}{\widehat\beta}
\newcommand{\hGamma}{\widehat\Gamma}


\newcommand{\ta}{\widetilde a}
\newcommand{\tb}{\widetilde b}
\newcommand{\tk}{\tilde k}
\newcommand{\tu}{\widetilde u}
\newcommand{\tv}{\tilde v}
\newcommand{\tw}{\widetilde w}
\newcommand{\tB}{\widetilde B}
\newcommand{\tD}{\widetilde D}
\newcommand{\tI}{\widetilde I}
\newcommand{\tP}{\widetilde P}
\newcommand{\tS}{\widetilde S}
\newcommand{\tU}{\widetilde U}
\newcommand{\tV}{\widetilde V}
\newcommand{\tW}{\widetilde W}
\newcommand{\tX}{\widetilde X}
\newcommand{\tfX}{\widetilde{\fX}}
\newcommand{\tfB}{\widetilde{\fB}}
\newcommand{\tsX}{\widetilde{\sX}}
\newcommand{\tH}{\widetilde H}
\newcommand{\tY}{\widetilde Y}
\newcommand{\talpha}{\widetilde{\alpha}}
\newcommand{\tbeta}{\widetilde{\beta}}
\newcommand{\tmu}{\widetilde{\mu}}
\newcommand{\tnu}{\widetilde{\nu}}
\newcommand{\tphi}{\widetilde{\phi}}
\newcommand{\ttau}{\widetilde{\tau}}


% Global tropicalization
\newcommand{\Ih}{I^\mathrm{h}}
\newcommand{\Iv}{I^\mathrm{v}}
\newcommand{\IX}{I_\fX}
\newcommand{\IY}{I_\fY}
\newcommand{\SD}{S_\fD}
\newcommand{\SX}{S_\fX}
\newcommand{\SsXH}{S_{(\sX,H)}}
\newcommand{\SY}{S_\fY}
\newcommand{\CsXH}{C_{(\sX,H)}}
\newcommand{\oSX}{\overline{\SX}}
\newcommand{\oIX}{\overline{\IX}}


% Vanishing cycles
\newcommand{\fXe}{\fX_\eta}
\newcommand{\fXs}{\fX_s}
\newcommand{\ofX}{\widebar{\fX}}
\newcommand{\ofXs}{\widebar{\fX}_s}
\newcommand{\fYe}{\fY_\eta}
\newcommand{\fYs}{\fY_s}
\newcommand{\fXbs}{\fX_{\bar s}}
\newcommand{\fXbe}{\fX_{\bar\eta}}
\newcommand{\fDe}{\fD_\eta}
\newcommand{\LX}{\Lambda_{\fX}}
\newcommand{\LXe}{\Lambda_{\fX_\eta}}
\newcommand{\LXs}{\Lambda_{\fXbs}}
\newcommand{\QXe}{\Q_{\ell,\fX_\eta}}
\newcommand{\QXbs}{\Q_{\ell,\fXbs}}
\newcommand{\sXe}{\sX_\eta}
\newcommand{\sXs}{\sX_s}
\newcommand{\LUe}{\Lambda_{\fU_\eta}}
\newcommand{\fCbs}{\fC_{\bar s}}
\newcommand{\QUe}{\Q_{\ell,\fU_\eta}}
\newcommand{\QCe}{\Q_{\ell,\fC_\eta}}
\newcommand{\QCs}{\Q_{\ell,\fCbs}}

% stacks

\newcommand{\hcC}{\mathrm h\cC}
\newcommand{\hcD}{\mathrm h\cD}
\newcommand{\PSh}{\mathrm{PSh}}
\newcommand{\Sh}{\mathrm{Sh}}
\newcommand{\Shv}{\mathrm{Shv}}
\newcommand{\Tuupperp}{\tensor*[^\cT]{u}{^p}}
\newcommand{\Tulowerp}{\tensor*[^\cT]{u}{_p}}
\newcommand{\Tuuppers}{\tensor*[^\cT]{u}{^s}}
\newcommand{\Tulowers}{\tensor*[^\cT]{u}{_s}}
\newcommand{\pu}{\tensor*[_p]{u}{}}
\newcommand{\su}{\tensor*[_s]{u}{}}
\newcommand{\Tpu}{\tensor*[^\cT_p]{u}{}}
\newcommand{\Tsu}{\tensor*[^\cT_s]{u}{}}
\newcommand{\Dfpull}{\tensor*[^\cD]{f}{^{-1}}}
\newcommand{\Dfpush}{\tensor*[^\cD]{f}{_*}}
\newcommand{\Duuppers}{\tensor*[^\cD]{u}{^s}}
\newcommand{\Dulowers}{\tensor*[^\cD]{u}{_s}}
\newcommand{\Geom}{\mathrm{Geom}}
\newcommand{\LPr}{\mathcal{P}\mathrm{r}^\rL}
\newcommand{\RPr}{\mathcal{P}\mathrm{r}^\rR}
\newcommand{\CX}{\cC_{/X}}
\newcommand{\CY}{\cC_{/Y}}
\newcommand{\CXP}{(\cC_{/X})_{\bP}}
\newcommand{\GeomXP}{(\mathrm{Geom}_{/X})_\bP}
\newcommand{\GeomYP}{(\mathrm{Geom}_{/Y})_\bP}
\newcommand{\infcat}{$\infty$-category\xspace}
\newcommand{\infcats}{$\infty$-categories\xspace}
\newcommand{\Span}{\mathrm{Span}}
\newcommand{\infsite}{$\infty$-site\xspace}
\newcommand{\infsites}{$\infty$-sites\xspace}
\newcommand{\inftopos}{$\infty$-topos\xspace}
\newcommand{\inftopoi}{$\infty$-topoi}
\newcommand{\pres}{{}^{\mathrm L} \mathcal P \mathrm{res}}
\newcommand{\Grpd}{\mathrm{Grpd}}
\newcommand{\sSet}{\mathrm{sSet}}
\newcommand{\rSet}{\mathrm{Set}}
\newcommand{\Ab}{\mathrm{Ab}}
\newcommand{\DAb}{\cD(\Ab)}
\newcommand{\tauan}{\tau_\mathrm{an}}
\newcommand{\qet}{\mathrm{q\acute{e}t}}
\newcommand{\tauet}{\tau_\mathrm{\acute{e}t}}
\newcommand{\tauqet}{\tau_\mathrm{q\acute{e}t}}
\newcommand{\bPsm}{\bP_\mathrm{sm}}
\newcommand{\bPqsm}{\bP_\mathrm{qsm}}
\newcommand{\Modh}{\mathrm{Mod}^\heartsuit}
\newcommand{\Mod}{\mathrm{Mod}}
\newcommand{\sMod}{\mathrm{sMod}}
\newcommand{\Coh}{\mathrm{Coh}}
\newcommand{\Cohb}{\mathrm{Coh}^{\mathsf b}}
\newcommand{\Cohh}{\mathrm{Coh}^\heartsuit}
\newcommand{\RcHom}{\rR\!\mathcal H\!\mathit{om}}
\newcommand{\kfiltered}{$\kappa$-filtered\xspace}
\newcommand{\St}{\mathrm{St}}
\newcommand{\Stn}{\mathrm{Stn}}
\newcommand{\Sch}{\mathrm{Sch}}
\newcommand{\FSch}{\mathrm{FSch}}
\newcommand{\Aff}{\mathrm{Aff}}
\newcommand{\Afflfp}{\mathrm{Aff}^{\mathrm{lfp}}}
\newcommand{\An}{\mathrm{An}}
\newcommand{\Afd}{\mathrm{Afd}}
\newcommand{\Top}{\mathcal T\mathrm{op}}
\newcommand{\Cohty}{\mathrm{Coh}^+(\fY) / \Coh^+(\fY)_\tors}
\newcommand{\Cohtu}{\mathrm{Coh}^+(\fU_n) / \Coh^+(\fU_n)_\tors}
\newcommand{\Nil}{\mathrm{Nil}}


%DFmlG

\newcommand{\ad}{\mathrm{adic}}
\newcommand{\dfDM}{\mathrm{dfDM}}
\newcommand{\dfSch}{\mathrm{dfSch}}
\newcommand{\rig}{\mathrm{rig}}
\newcommand{\rigg}{(-)^{\mathrm{rig}}}
\newcommand{\loc}{\mathrm{loc}}
\newcommand{\fXzar}{\fX^{\mathrm{dfSch}}_{\mathrm{dfZar}}}
\newcommand{\cTad}{\cT_{\mathrm{adic}}(k^\circ)}
\newcommand{\taft}{\mathrm{taft}}


% DAnG

\newcommand{\dAn}{\mathrm{dAn}}
\newcommand{\dAnc}{\mathrm{dAn}_{\mathbb C}}
\newcommand{\dAnk}{\mathrm{dAn}_k}
\newcommand{\Ank}{\mathrm{An}_k}
\newcommand{\cTan}{\cT_{\mathrm{an}}}
\newcommand{\cTank}{\cT_{\mathrm{an}}(k)}
\newcommand{\cTdisc}{\cT_{\mathrm{disc}}}
\newcommand{\cTdisck}{\cT_{\mathrm{disc}}(k)}
\newcommand{\cTet}{\cT_{\mathrm{\acute{e}t}}}
\newcommand{\cTetk}{\cT_{\mathrm{\acute{e}t}}(k)}
\newcommand{\Str}{\mathrm{Str}}
\newcommand{\wStr}[2]{\Fun'(#1, #2)} %wStr stands for weak structures, i.e. structures where we dropped the third conditions.
\newcommand{\Strloc}{\mathrm{Str}^\mathrm{loc}}
\newcommand{\RTop}{\tensor*[^\rR]{\Top}{}}
\newcommand{\LTop}{\tensor*[^\rL]{\Top}{}}
\newcommand{\RHTop}{\tensor*[^\rR]{\mathcal{H}\Top}{}}
\newcommand{\LRT}{\mathrm{LRT}}
\newcommand{\Tor}{\mathrm{Tor}}
\newcommand{\dAfd}{\mathrm{dAfd}}
\newcommand{\dAfdk}{\mathrm{dAfd}_k}
\newcommand{\dStn}{\mathrm{dStn}}
\newcommand{\biget}{\mathrm{big,\acute{e}t}}
\newcommand{\trunc}{\mathrm{t}_0}
\newcommand{\Hyp}{\mathrm{Hyp}}
\newcommand{\HSpec}{\mathrm{HSpec}}
\newcommand{\CAlg}{\mathrm{CAlg}}
\newcommand{\CAlgad}{\mathrm{CAlg}^{\mathrm{ad}}}
\newcommand{\Ring}{\mathrm{Ring}}
\newcommand{\CRing}{\mathrm{CRing}}
\newcommand{\sCRing}{\mathrm{sCRing}}
\newcommand{\trunctopoi}{\Spec^{\cG_{\mathrm{an}}^{\le 0}(k)}_{\cG_{\mathrm{an}(k)}}}
\newcommand{\Cat}{\mathrm{Cat}}
\newcommand{\Catinf}{\mathrm{Cat}_\infty}
\newcommand{\Catst}{\mathrm{Cat}_{\infty}^{\mathrm{st}}}
\newcommand{\final}{\mathrm{final}}
\newcommand{\initial}{\mathrm{initial}}
\newcommand{\anPreStk}{\mathrm{AnPreStk}}


% Analytic deformation theory

\newcommand{\fib}{\mathrm{fib}}
\newcommand{\DerAn}{\mathrm{Der}\an}
\newcommand{\anL}{\mathbb L\an}
\newcommand{\fCAlg}{\mathrm{fCAlg}}
\newcommand{\adL}{\mathbb L^{\mathrm{ad}}}
\newcommand{\AnRing}{\mathrm{AnRing}}
\newcommand{\SpecEtAn}{\mathrm{Spec}^{\cTan}_{\cTet}}
\newcommand{\cTanc}{\cTan(\mathbb C)}
\newcommand{\Zar}{_{\mathrm{Zar}}}
\newcommand{\dAff}{\mathrm{dAff}}
\newcommand{\afp}{^{\mathrm{afp}}}
\newcommand{\bfMap}{\mathbf{Map}}
\newcommand{\cHom}{\cH \mathrm{om}}
\newcommand{\dAnSt}{\mathrm{dAnSt}}
\newcommand{\PrL}{\mathcal P \mathrm{r}^{\mathrm{L}}}
\newcommand{\PrR}{\mathcal P \mathrm{r}^{\mathrm{R}}}
\newcommand{\cTAb}{\cT_{\Ab}}
\newcommand{\Catlex}{\Cat_\infty^{\mathrm{lex}}}
\newcommand{\underover}[1]{#1//#1}
\newcommand{\Perf}{\mathrm{Perf}}
\newcommand{\anNil}{\mathrm{AnNil}}
\newcommand{\anFMP}{\mathrm{AnFMP}}
\newcommand{\anFGrpd}{\mathrm{AnFGrpd}}


% Hilbert and Map

\newcommand{\fAfflfp}[1]{\mathrm{fAff}_{#1}^{\mathrm{lfp}}}
\newcommand{\fAff}[1]{\mathrm{fAff}_{#1}}
\newcommand{\Hilb}{\mathrm{Hilb}}
\newcommand{\dSch}{\mathrm{dSch}}
\newcommand{\FormalModels}{\mathrm{FM}}
\newcommand{\Ind}{\mathrm{Ind}}

% Non-archimedean Quantum K-theory

\newcommand{\vdim}{\mathrm{vdim}}
\newcommand{\cOvir}{\cO^{\mathrm{vir}}}
\newcommand{\bsM}{\widebar{\mathscr M}}
\newcommand{\fM}{\mathfrak M}
\newcommand{\tsigma}{\widetilde{\sigma}}

% Special symbols
\newcommand{\bcM}{\widebar{\mathcal M}}
\newcommand{\bcC}{\widebar{\mathcal C}}
\newcommand{\bcMgn}{\widebar{\mathcal M}_{g,n}}
\newcommand{\bcMol}{\widebar{\mathcal M}_{0,1}}
\newcommand{\bcMot}{\widebar{\mathcal M}_{0,3}}
\newcommand{\bcMof}{\widebar{\mathcal M}_{0,4}}
\newcommand{\bcMon}{\widebar{\mathcal M}_{0,n}}
\newcommand{\bcMgnprime}{\widebar{\mathcal M}_{g,n'}}
\newcommand{\bcMgnijprime}{\widebar{\mathcal M}_{g_{ij},n'_{ij}}}
\newcommand{\bMgnt}{\widebar{M}^\mathrm{trop}_{g,n}}
\newcommand{\Mmdisc}{M_{m\textrm{-disc}}}
\newcommand{\Gm}{\mathbb G_{\mathrm m}}
\newcommand{\Gmk}{\mathbb G_{\mathrm m/k}}
\newcommand{\Gmkprime}{\mathbb G_{\mathrm m/k'}}
\newcommand{\Gmnan}{(\Gm^n)\an}
\newcommand{\Gmknan}{(\Gmk^n)\an}
\newcommand{\Lin}{\mathit{Lin}}
\newcommand{\Simp}{\mathit{Simp}}
\newcommand{\vol}{\mathit{vol}}
\newcommand{\LanD}{\mathcal L_{an}^D}

% Categories

\newcommand{\cart}{\mathrm{cart}}
\DeclareMathOperator{\Tw}{Tw}
\DeclareMathOperator{\Exc}{Exc}
\newcommand{\fin}{\mathrm{fin}}
\newcommand{\lex}{\mathrm{lex}}
\DeclareMathOperator{\Ob}{Ob}
\newcommand{\ind}{\mathrm{Ind}}
\newcommand{\pro}{\mathrm{Pro}}
\newcommand{\cofib}{\mathrm{cofib}}
\newcommand{\QCoh}{\mathrm{QCoh}}
\newcommand{\QCohcpl}{\mathrm{QCoh}_{\mathrm{cpl}}}

% Shorthands
\newcommand{\kc}{k^\circ}
\newcommand{\llb}{[\![}
\newcommand{\rrb}{]\!]}
\newcommand{\llp}{(\!(}
\newcommand{\rrp}{)\!)}
\newcommand{\an}{^\mathrm{an}}
\newcommand{\alg}{^\mathrm{alg}}
\newcommand{\loweralg}{_\mathrm{alg}}
\newcommand{\bad}{^\mathrm{bad}}
\newcommand{\ess}{^\mathrm{ess}}
\newcommand{\ness}{^\mathrm{ness}}
\newcommand{\et}{_\mathrm{\acute{e}t}}
\newcommand{\fet}{_\mathrm{f\acute{e}t}}
\newcommand{\ev}{\mathit{ev}}
%\newcommand{\eistar}{\mathbf e_i^*}
%\newcommand{\ejstar}{\mathbf e_j^*}
%\newcommand{\ekstar}{\mathbf e_k^*}
\newcommand{\mult}{\mathit{mult}}
\newcommand{\inv}{^{-1}}
\newcommand{\id}{\mathrm{id}}
\newcommand{\gn}{$n$-pointed genus $g$ }
\newcommand{\gnprime}{$n'$-pointed genus $g$ }
\newcommand{\GW}{\mathrm{GW}}
\newcommand{\GWon}{\GW_{0,n}}
\newcommand{\canal}{$\mathbb C$-analytic\xspace}
\newcommand{\nanal}{non-archimedean analytic\xspace}
\newcommand{\kanal}{$k$-analytic\xspace}
\newcommand{\ddim}{$d$-dimensional\xspace}
\newcommand{\ndim}{$n$-dimensional\xspace}
\newcommand{\narch}{non-archimedean\xspace}
\newcommand{\nminusone}{$(n\!-\!1)$}
\newcommand{\nminustwo}{$(n\!-\!2)$}
\newcommand{\red}{\mathrm{red}}
\renewcommand{\th}{^\mathrm{\tiny th}}
\newcommand{\Wall}{\mathit{Wall}}
\newcommand{\vlb}{virtual line bundle\xspace}
\newcommand{\mvlb}{metrized \vlb}
\newcommand{\wrt}{with respect to\xspace}
\newcommand{\Zaffine}{$\mathbb Z$-affine\xspace}
\newcommand{\sw}{^\mathrm{sw}}
\newcommand{\Trop}{\mathrm{Trop}}
\newcommand{\trop}{^\mathrm{trop}}
\newcommand{\op}{^\mathrm{op}}
\newcommand{\Cech}{\check{\mathcal C}}
\newcommand{\DM}{Deligne-Mumford\xspace}
\providecommand{\abs}[1]{\lvert#1\rvert}
\providecommand{\norm}[1]{\lVert#1\rVert}
\newcommand{\tors}{\mathrm{tors}}
\newcommand{\cpl}{\mathrm{cpl}}
\newcommand{\cl}{\mathrm{cl}}


% Arrows
\newcommand*{\longhookrightarrow}{\ensuremath{\lhook\joinrel\relbar\joinrel\rightarrow}}
\newcommand*{\DashedArrow}[1][]{\mathbin{\tikz [baseline=-0.25ex,-latex, dashed,#1] \draw [#1] (0pt,0.5ex) -- (1.3em,0.5ex);}}

\usetikzlibrary{decorations.markings} %arrows for open immersions and closed immersions
\tikzset{
  closed/.style = {decoration = {markings, mark = at position 0.5 with { \node[transform shape, xscale = .8, yscale=.4] {/}; } }, postaction = {decorate} },
  open/.style = {decoration = {markings, mark = at position 0.5 with { \node[transform shape, scale = .7] {$\circ$}; } }, postaction = {decorate} }
}


%Operators
\DeclareMathOperator{\Anc}{Anc}
\DeclareMathOperator{\Area}{Area}
\DeclareMathOperator{\Aut}{Aut}
\DeclareMathOperator{\Bl}{Bl}
\DeclareMathOperator{\CH}{CH}
\DeclareMathOperator{\Coker}{Coker}
\DeclareMathOperator{\codim}{codim}
\DeclareMathOperator{\cosk}{cosk}
\DeclareMathOperator{\Div}{Div}
\DeclareMathOperator{\dist}{dist}
\DeclareMathOperator{\Ext}{Ext}
\DeclareMathOperator{\Fun}{Fun}
\DeclareMathOperator{\FunR}{Fun^R}
\DeclareMathOperator{\FunL}{Fun^L}
\DeclareMathOperator{\Gal}{Gal}
\DeclareMathOperator{\Hom}{Hom}
\DeclareMathOperator{\Image}{Im}
\DeclareMathOperator{\Int}{Int}
\DeclareMathOperator{\Isom}{Isom}
\DeclareMathOperator{\Ker}{Ker}
\DeclareMathOperator{\Map}{Map}
\DeclareMathOperator{\Mor}{Mor}
\DeclareMathOperator{\NE}{NE}
\DeclareMathOperator{\oStar}{\widebar{\Star}}
\DeclareMathOperator{\Pic}{Pic}
\DeclareMathOperator{\Proj}{Proj}
\DeclareMathOperator{\rank}{rank}
\DeclareMathOperator{\RHom}{RHom}
\DeclareMathOperator{\Sp}{Sp}
\DeclareMathOperator{\Spa}{Spa}
\DeclareMathOperator{\SpB}{Sp_\mathrm{B}}
\DeclareMathOperator{\Spec}{Spec}
\DeclareMathOperator{\Spf}{Spf}
\DeclareMathOperator{\Star}{Star}
\DeclareMathOperator{\supp}{supp}
\DeclareMathOperator{\Sym}{Sym}
\DeclareMathOperator{\val}{val}

\DeclareMathOperator*{\colim}{colim}
\DeclareMathOperator*{\holim}{holim}
\DeclareMathOperator*{\hocolim}{hocolim}
\DeclareMathOperator*{\cotimes}{\widehat{\otimes}}


\begin{document}

\title{Spreading out the Hodge filtration in rigid analytic geometry}

\author{Jorge ANT\'ONIO}
\address{Jorge ANT\'ONIO, IRMA, UMR 7501
7 rue René-Descartes
67084 Strasbourg Cedex}
\email{jorgeantonio@unistra.fr}


\begin{abstract}

\end{abstract}

\maketitle

\tableofcontents

\section{Introduction}

In this paper, we will provide a rigid analytic construction of the deformation to the normal cone, studied in \cite{Gaitsgory_Study_II}.
Our goal is to use this geometric construction to deduce certain important results concerning both \emph{rigid analytic} and
\emph{over-convergent} (Hodge complete)
\emph{derived de Rham cohomology} of rigid analytic spaces over a non-archimedean field of characteristic zero.
We will then exploit this ideas to come up with analogues concerning \emph{derived rigid cohomology} of finite type schemes over a perfect field
in characteristic zero. In particular, our main goal is to extrapolate the main result of \cite{Bhatt_Derived_Completions} to the setting of
derived rigid cohomology.

\subsection{Notations and Conventions} We shall denote the \emph{analytic mapping stack} as $\bfMap_{\dAnSt_k}(X, Y)$.


\subsection{Preliminaries}
Let $\cX$ be an \inftopos. The notion of a \emph{local $\cTank$-structure on $\cX$} was first introduced in \cite[Definition 2.4]{Porta_Yu_Derived_non-archimedean_analytic_spaces},
see also \cite[\S 2]{antonio2018p}.

Let  $\cO \in \Strloc_{\cTank}(\cX)$ be a local $\cTank$-structure on $\cX$. Since the pregeometry $\cTank$ is compatible
with $n$-truncations, cf. \cite[Theorem 3.23]{Porta_Yu_Derived_non-archimedean_analytic_spaces}, it follows that
$\pi_0(\cO) \in \Strloc_{\cTank}(\cX)$, as well.

Denote by $\cJ \subseteq \pi_0(\cO)$, the \emph{Jacobson ideal} of $\pi_0(\cO \alg)$, which can be naturally regarded as an object
in the \infcat
    \[
        \Mod_{\pi_0(\cO \alg)} \simeq \Mod_{\pi_0(\cO)},  
    \]
for a justification of the latter equivalence, see for instance \cite[Theorem 4.5]{Porta_Yu_Representability} .
Since the \infcat $\Str_{\cTank}(\cX)$ is a presentable \infcat we can consider the quotient
    \[
        \pi_0(\cO)_\red \coloneqq \pi_0(\cO)/\cJ \in \Str_{\cTank}(\cX)  ,
    \]
which we refer to the \emph{reduced $\cTank$-structure on $\cX$ associated to $\pi_0(\cO)$}. Moreover, the corresponding \emph{underlying algebra} satisfies
    \[(\pi_0(\cO)_{\red})\alg \simeq \pi_0(\cO)\alg/ \cJ \in \Str_{\cTdisck}(\cX).\]
One can further prove that
$\pi_0(\cO)_\red \in \Str_{\cTank}(\cX)$ actually lies in the full subcategory $\Strloc_{\cTank}(\cX)$.

\begin{defin}
    Let $Z = (\cZ, \cO_Z) \in \RTop(\cTan(k))$ denote a $\cTank$-structured \inftopos. We define the \emph{reduced $\cTank$-structure \inftopos} as
        \[
            Z_\red \coloneqq (\cZ, \pi_0(\cO_Z)_\red) \in \RTop(\cTank).
        \]
    We shall denote by $\Afd_k^\red$ (resp., $\Ank^\red$) the full subcategory of $\dAfd_k$ (resp., $\dAnk$)
    spanned by reduced $k$-affinoid (resp., $k$-analytic spaces).
\end{defin}

\begin{notation}
    Let $(-)^\red \colon \dAnk \to \Ank^\red$ denote the functor obtained by the formula
        \[
            Z = (\cZ, \cO_Z) \in \dAnk\mapsto Z_\red = (\cZ, \pi_0(\cZ)_\red) \in \Ank^\red.
        \]
    We shall refer to it as the \emph{underlying reduced $k$-analytic space}.
\end{notation}

\begin{defin}
    Let $f \colon X \to Y$ be a morphism in $\dAnk$. We shall say that $f$ is an \emph{admissible open immersion} if the induced morphism on $0$-th truncations
        \[
            \trunc(f) \colon \trunc(X) \to \trunc(Y),  
        \]
    is an admissible open immersion in the sense of \cite[\S 1.3]{Berkovich_Etale_1993}.
\end{defin}


\begin{lem}
    Let $f \colon X \to Y$ be an admissible open immersion of derived $k$-analytic spaces. Then $f^\red \colon X^\red \to Y^\red$ is also
    an admissible open immersion.
\end{lem}

\begin{proof}
    By the definitions, it is clear that the truncation
        \[
            \trunc(f) \colon \trunc(X) \to \trunc(Y),  
        \]
    is an admissible open immersion of ordinary $k$-analytic spaces. In the case of ordinary $k$-analytic spaces it is clear from the construction
    that the reduction of Zariski open immersions is again a Zariski open immersion.
\end{proof}

\begin{defin}
    In \cite[Definition 5.41]{Porta_Yu_Representability} the authors introduced the notion of a square-zero extension between $\cTank$-structured
    \inftopoi. In particular, given a morphism $f \colon Z \to Z'$ in $\RTop(\cTank)$, we shall say that $f$ \emph{has the structure of
    a square-zero extension} if $f$ exhibits $Z'$ as a square-zero extension of $Z$.
\end{defin}

Recall the definition of the \infcats of derived $k$-affinoid and derived $k$-analytic spaces given
in \cite[Definition 7.3 and Definition 2.5.]{Porta_Yu_Derived_non-archimedean_analytic_spaces}, respectively.

\begin{rem} \label{rem:construction_of_nilpotent_extensions_as_square_zero_extensions}
    Let $X \in \Ank$. Let $\cJ \subseteq \cO_X$ be an ideal satisfying $\cJ^2 = 0$. Consider the fiber sequence
        \[
            \cJ \to \cO_X \to \cO_X/\cJ,  
        \]
    in the \infcat $\Coh^+(X)$. We have a fiber sequence of the form
        \[
            \bbL_{\cO_X}\an \to \bbL_{\cO_X / \cJ}\an \to \bbL_{\cO_X / \cJ / \cO_X}\an,
        \]
    and we have a further identification $\tau_{\le 1} (\bbL_{\cO_X/ \cJ / \cO_X}) \simeq \cJ[1]$. For this reason, we obtain a well defined morphism
        \[
            d \colon \bbL_{\cO_X/ \cJ} \to \cJ[1],
        \]
    in the derived \infcat $\Mod_{\cO_X/ \cJ}$. This derivation classifies a square-zero extension of $\cO_X/ \cJ$ by $\cJ[1]$ which can be identified with the
    object $\cO_X$ itself. In particular, we deduce that $X$ is a square-zero extension of $X_\red$.
\end{rem}

\begin{lem} \label{lem:derived_k_analytic_space_whose_reduction_is_affinoid_is_also_affinoid}
    Let $Z \coloneqq (\cZ, \cO_Z) \in \RTop(\cTank)$ denote a $\cTank$-structure \inftopos such that $\pi_0(\cO_Z\alg) $ is Noetherian derived $k$-algebra on $\cZ$.
    Suppose that the reduction
    $Z_\red$ is equivalent to a derived $k$-affinoid space. Then the truncation $\trunc(Z)$ is isomorphic to an ordinary $k$-affinoid space.
    If we assume further that for every $i>0$, the homotopy sheaves $\pi_i(\cO_Z)$ are
    coherent $\pi_0(\cO_Z)$-modules, then $Z$ itself is equivalent to a derived $k$-affinoid space.
\end{lem}

\begin{proof} The second claim of the Lemma follows readily from the first assertion together with the definitions.
    We are thus reduced to prove that $\trunc(Z)$ is isomorphic to an ordinary $k$-affinoid space.
    Let $\cJ \subseteq \pi_0(\cO_Z)$, denote the coherent ideal sheaf associated to the closed immersion $Z_\red \hookrightarrow Z$. Notice that the ideal $\cJ$
    agrees with the Jacobson ideal of $\pi_0(\cO_Z)$. Thanks to our assumption that $\pi_0(\cO_Z)$ is a Noetherian derived $k$-algebra on $\cZ$, it follows that there exists
    a sufficiently large integer $n \ge 2$ such that
        \[
            \cJ^n = 0.  
        \]
    Arguing by induction, we can suppose that $n = 2$, that is to say that
        \[\cJ^2 = 0.\]
    In particular, \cref{rem:construction_of_nilpotent_extensions_as_square_zero_extensions} implies that the
    the natural morphism $Z_\red \to Z$ has the structure of a square zero extension.
    The assertion now follows from \cite[Proposition 6.1]{Porta_Yu_Representability}
    and its proof.
\end{proof}

\begin{rem}
    We observe that the converse of \cref{lem:derived_k_analytic_space_whose_reduction_is_affinoid_is_also_affinoid} holds true.
    Indeed, the natural morphism $Z_\red \to Z$ is a closed immersion. In particular, if $Z \in \dAfd_k$ we deduce readily from
    that $Z_\red \in \dAfd_k$, as well.
\end{rem}

\begin{defin}
    Let $f \colon X \to Y$ be a morphism in the \infcat $\dAnk$. We shall say that $f$ is an \emph{affine morphism} if
    for every morphism $Z \to Y$ in $\dAnk$ such that $Z$ is equivalent to a derived $k$-affinoid space, the pullback
        \[
            Z' \coloneqq Z \times_Y X \in \dAnk,  
        \]
    is also equivalent to a derived $k$-affinoid space.
\end{defin}

\begin{lem} \label{lem:affine_morphisms_are_compatible_with_Zariski_localization_on_the_base}
    Let $f \colon X \to Y$ be an affine morphism in $\dAn_k$. Suppose we are given an admissible open covering
        \[
            g \colon \coprod_{j \in J} U_j \to Y,  
        \]
    where for each $j \in J$, $U_j \in \dAfd_k$. For each $j \in J$, let 
        \[
            V_j \coloneqq U_j \times_X Y \in \dAfd_k,  
        \]
    then $\coprod_{j \in J} V_j \to Y$ is an admissible open covering.
\end{lem}

\begin{proof} It is clear from our assumption that $f$ is an affine morphism that for every index $j \in J$, the objects $V_j \in \dAfd_k$. The claim of the Lemma
    follows immediately from the observation that
    both the classes of effective epimorphisms of \inftopoi and admissible open immersions of derived $k$-analytic spaces are stable under pullbacks,
    cf. \cite[Proposition 6.2.3.15]{HTT} and \cite[Corollary 5.11, Proposition 5.12]{Porta_Yu_Representability}, respectively.
\end{proof}


\section{Non-archimedean differential geometry}

\subsection{Analytic formal moduli problems under a base}
In this \S, we will study the notion of \emph{analytic formal moduli problems under} a fixed derived $k$-analytic space. The
results presented here will prove to be crucial for the study of the deformation to the normal cone in the $k$-analytic
setting, presented in the next section.
We start with the following definition:

\begin{defin}
    \begin{enumerate}
        \item Let $f \colon X \to Y$ be a morphism in $\dAnk$. We say that $f$ is a \emph{nil-isomorphism} if $f_\red \colon X_\red \to
        Y_\red$ is an isomorphism of $k$-analytic spaces. 
        \item Let $X = (\cX, \cO_X)$ be a derived $k$-analytic space. We shall say that a nil-isomorphism $f \colon X \to Y$ is \emph{of finite type} if for every
            \[
                \cO_Y \to \cO_X,   
            \]
        induces an equivalence $\tau_{\ge m }(\cO_Y) \to \tau_{\ge m}(\cO_X)$ in the \infcat $\cD_\mathrm{Ab}(\cX)$.
        \item We will denote by $\anNil_{/ X}$ the full subcategory of $(\dAnk)_{X/}$
        spanned by nil-isomorphisms $X \to Y$ of finite type.
    \end{enumerate}
\end{defin}

\begin{lem} \label{lem:nil-isos_are_affine_morphisms}
    Let $f \colon X \to Y$ be a nil-isomorphism of derived $k$-analytic spaces. Then:
    \begin{enumerate}
        \item Given any morphism $Z \to Y$ in $\dAnk$, the induced morphism
            \[
                Z \times_X Y \to Z,  
            \]
        is again an nil-isomoprhism.
        \item $f$ is an affine morphism.
        \item $f$ is a finite morphism.
    \end{enumerate}
\end{lem}

\begin{proof} To prove (i), it suffices to prove that
    the functor $(\textrm{-})^\red \colon \dAnk \to \Ank^{\red}$ commutes with finite limits. The truncation functor
        \[
            \trunc \colon \dAnk \to \Ank,  
        \]
    commutes with finite limits, c.f. \cite[Proposition 6.2 (v)]{Porta_Yu_Derived_non-archimedean_analytic_spaces}. It suffices then to prove that the usual underlying reduced functor
        \[
            (-)^\red \colon \Ank \to \Ank^\red,
        \]
    commutes with finite limits. By construction,
    the latter assertion is equivalent to the claim that
    the usual complete tensor product of ordinary $k$-affinoid algebras commutes with the operation of taking the quotient by the Jacobson radical, which is immediate.

    We now prove (ii). Let $Z \to Y$ be an admissible open immersion such that $Z$ is a derived $k$-affinoid space. We claim that the pullback
    $Z \times_X Y$ is again a derived $k$-affinoid space. Thanks to \cref{lem:derived_k_analytic_space_whose_reduction_is_affinoid_is_also_affinoid}
    we reduced to prove that $(Z \times_X Y)_\red$ is equivalent to an
    ordinary $k$-affinoid space. Thanks to (i), we deduce that the induced morphism
        \[
            (Z \times_X Y)_\red \to Z_\red,  
        \]
    is an isomorphism of ordinary $k$-analytic spaces. In particular, $(Z \times_X Y)_\red$ is a $k$-affinoid space. The result now follows from
    \cref{lem:derived_k_analytic_space_whose_reduction_is_affinoid_is_also_affinoid}.

    To prove (iii), we shall show that the induced morphism on the $0$-th truncations $\trunc(X) \to \trunc(Y)$ is a finite morphism of ordinary $k$-affinoid spaces.
    But this follows immeaditely from the fact that both $\trunc(X)$ and $\trunc(Y)$ can be obtained from the reduced $X_\red$ by means of a finite sequence of square-zero extensions
    as in \cref{rem:construction_of_nilpotent_extensions_as_square_zero_extensions}.
\end{proof}


\begin{defin}
    A morphism $X \to Y$ be a morphism in $\dAnk$ is called a \emph{nil-embedding} if the induced morphism of ordinary $k$-analytic spaces
    $\trunc(X) \to \trunc(Y)$ is a closed immersion whose ideal of definition assumed to be nilpotent. 
\end{defin}

\begin{prop} \label{prop:filtered_colimit_for_nil-embeddings}
    Let $f \colon X \to Y$ be a nil-embedding of derived $k$-analytic spaces. Then there exists a sequence of morphisms
        \[X = X_0^0 \hookrightarrow X_0^1 \hookrightarrow \dots \hookrightarrow X_0^n = X_0 
        \hookrightarrow X_1 \dots X_n \hookrightarrow \dots \hookrightarrow Y,\]
    such that for each $0 \le i \le n$ and $j \ge 0$ the morphisms $X_0^i \hookrightarrow X_0^{i+1}$ and $X_j \to X_{j+1}$ have the structure of square-zero extensions.
    Furthermore, the induced morphisms $\mathrm{t}_{\le j}(X_j) \to \mathrm{t}_{\le j}(Y)$ are equivalences of derived
    $k$-analytic spaces, for every $j \ge 0$.
\end{prop}

\begin{proof}
    The proof follows the same scheme of reasoning as of \cite[Proposition 5.5.3]{Gaitsgory_Study_II}. For the sake of completeness we present the complete argument here.
    Consider the induced morphism on the underlying truncations
        \[
            \trunc(f) \colon \trunc(X) \to \trunc(Y).   
        \]
    By construction, there exists a sufficiently large integer $n \ge 0$ such that
        \[
            \cJ^{n+1} = 0,  
        \]
    where $\cJ \subseteq \pi_0(\cO_Y)$ denotes the ideal associated to the nil-embedding $\trunc(f)$.
    Therefore, we can factor the latter as a finite sequence of square-zero extensions of ordinary $k$-analytic spaces
        \[
            \trunc(X) \hookrightarrow X_0^{\mathrm{ord}, 0} \hookrightarrow \dots \hookrightarrow X^{\mathrm{ord}, n}_0 = \trunc(Y),
        \]
    as in the proof of \cref{lem:derived_k_analytic_space_whose_reduction_is_affinoid_is_also_affinoid}. For each $0 \le i \le n$, we set
        \[
            X_0^i \coloneqq X \bigsqcup_{\trunc(X)} X_0^{\mathrm{ord}, i}.
        \]
    By construction, we have that the natural morphism $\trunc(X_0^n) \to \trunc(Y)$ is an isomorphism of ordinary $k$-analytic spaces.
    We now argue by induction on the Postnikov tower associated to the morphism $f \colon X \to Y$.
    Suppose that for a certain integer $i \ge 0$, we have constructed a derived $k$-analytic space $X_i$ together with morphisms $g_i \colon
    X \to X_i$ and $h_i \colon X_i \to Y$ such that $f \simeq h_i \circ g_i$
    and the induced morphism
        \[
            \mathrm{t}_{\le i}(X_i) \to \mathrm{t}_{\le i}(Y)
        \]
    is an equivalence of derived $k$-analytic spaces. We shall proceed as follows: by the assumption that $h_i$ is $(i+1)$-connective, we deduce from
    \cite[Proposition 5.34]{Porta_Yu_Representability} the existence of a natural equivalence
        \[
            \tau_{\le i}(\bbL_{X_i/Y}\an) \simeq 0,
        \]
    in $\Mod_{\cO_{X_i}}$. Consider the natural fiber sequence
        \[
            h_i^* \bbL\an_{Y} \to \bbL\an_{X_i} \to \bbL\an_{X_i/Y},
        \]
    in $\Mod_{\cO_{X_i}}$. The natural morphism
        \[
            \bbL\an_{X_i/ Y} \to \pi_{i+1}(\bbL\an_{X_i/Y})[i+1],  
        \]
    induces a morphism $\bbL\an_{X_i} \to \pi_{i+1}(\bbL\an_{X_i/Y})[i+1]$, such that the composite
        \begin{equation} \label{eq:fiber_sequence_of_cotangent_complexes_to_produce_the_existence_of_the_desired_square_zero_extension_approximating_Y_in_degree_i+1}
            h_i^* \bbL\an_{Y} \to \bbL\an_{X_i} \to \pi_{i+1}(\bbL\an_{X_i/Y})[i+1],  
        \end{equation}
    is null-homotopic, in $\Mod_{\cO_{X_i}}$. By the universal property of the relative analytic cotangent complex,
    \eqref{eq:fiber_sequence_of_cotangent_complexes_to_produce_the_existence_of_the_desired_square_zero_extension_approximating_Y_in_degree_i+1}
    produces a square-zero extension
        \[
            X_i  \to X_{i+1},
        \]
    together with a morphism $h_{i+1} \colon X_{i+1} \to Y$, factoring $h_i \colon X_i \to Y$. We  are reduced to show that the morphism
        \[
            \cO_Y \to h_{i+1, *}(\cO_{X_{i+1}}), 
        \]
    is $(i+2)$-connective. Consider the commutative diagram
        \begin{equation} \label{eq:commutative_diagram_of_fiber_sequences_exhibiting_O_X_i+1_as_an_approximation_of_level_i+1_of_Y}
        \begin{tikzcd}
            h_{i, *}(\pi_{i+1}(\bbL\an_{X_i/Y}))[i] \ar{r} & h_{i+1, *}(\cO_{X_{i+1}}) \ar{r} & h_{i, *}(\cO_{X_i}) \\
            \cI \ar{r} \ar{u}{s_i} & \cO_Y \ar{r} \ar{u} & h_*(\cO_{X_i}) \ar{u} \\
            \cJ \ar{r}{=} \ar{u} & \cJ \ar{r} \ar{u} & 0 \ar{u}
        \end{tikzcd},
        \end{equation}
    in $\mathrm{Mod}_{\cO_Y}$, where both the vertical and horizontal composites are fiber sequences. By our inductive hypothesis, $\cJ$ is $(i+1)$-connective.
    Moreover, thanks to \cite[Proposition 5.34]{Porta_Yu_Representability} we can identify the natural
    morphism    
        \[
           s_i \colon \cI \to h_{i, *}(\pi_{i+1}(\bbL\an_{X_i/Y}))[i]
        \]
    with the natural morphism $\cI \to \tau_{\le i}(\cI)$. We deduce that the fiber of the morphism $s_i$ must be necessarily $(i+2)$-connective. The latter observation
    combined with the structure of \eqref{eq:commutative_diagram_of_fiber_sequences_exhibiting_O_X_i+1_as_an_approximation_of_level_i+1_of_Y}
    implies that $h_{i+1} \colon X_{i+1} \to Y$ induces an equivalence of derived $k$-analytic spaces
        \[
            \rt_{\le i+1}(X_{i+1}) \to \rt_{\le i+1}(Y),  
        \]
    as desired.
\end{proof}

\begin{cor} The following assertions hold:
    \begin{enumerate}
        \item Let $X \in \dAnk$. Then the natural morphism
          \[
                X_\red \to X,  
            \]
        in $\dAnk$, can be \emph{approximated} by successive square zero extensions.
        \item If $f \colon X \to Y$ is a nil-embedding of finite type then it can be approximated by a finite sequence of square-zero extensions.
    \end{enumerate}
\end{cor}

\begin{proof}
    Both the assertions of the Corollary follow readily from \cref{prop:filtered_colimit_for_nil-embeddings} by observing that the
    canonical morphism $X_\red \to X$ has the structure of a nil-embedding and that the finiteness assumption on $f$ forces the finiteness of the approximation
    sequence.
\end{proof}

\begin{lem} \label{lem:pullbacks_of_derived_affinoid_spaces_along_finite_morphisms_are_algebraic}
    Let $f \colon X \to Y$ be a finite morphism of derived $k$-affinoid spaces. Let $Z \to Y$ be an admissible open immersion and denote by
        \[
            A \coloneqq \Gamma(X, \cO_X\alg), \quad B \coloneqq \Gamma(Y, \cO_Y\alg), \quad C \coloneqq \Gamma(Z, \cO_Z\alg),  
        \]
    the corresponding derived $k$-algebras of derived global sections. Denote by
        \[
            Z' \coloneqq Z \times_Y X \in \dAfd_k,   
        \]
    then one has a natural equivalence
        \[
            \Gamma(Z', \cO_{Z'}\alg) \simeq A \otimes_B C,  
        \]
    in the \infcat $\CAlg_k$.
\end{lem}

\begin{proof}
    Consider the natural morphism of derived $k$-algebras
        \[
            \theta \colon A \otimes_B \to C \to \Gamma(Z', \cO_{Z'} \alg).
        \]
    Our goal is to show that $\theta$ is an equivalence. We start by observing that
    since the morphism $f \colon X \to Y$ is finite,
    the induced morphism $B \to A$ in $\CAlg_k$ is finite as well. In particular, the (ordinary) complete tensor product coincides with the (ordinary) algebraic tensor product, i.e.,
    the natural morphism
        \[
            \pi_0(A) \otimes_{\pi_0(B)} \pi_0(C) 
            \to \pi_0(A) \widehat{\otimes}_{\pi_0(B)} \pi_0(C)   ,
        \]
    is an equivalence of ordinary rings. In particular, thanks to \cite[Proposition 6.2 (v) and Theorem 6.5]{Porta_Yu_Derived_non-archimedean_analytic_spaces} we deduce that
        \[
            \pi_0(A \otimes_B C) \to \pi_0(\Gamma(Z', \cO_{Z'}\alg)),  
        \]
    is an equivalence of derived rings, as well. In order to conclude, we shall prove that for every $i> 0$, the natural morphism
        \[
            \pi_i(A \otimes_B C) \to \pi_i(\Gamma(Z', \cO_{Z'})),   
        \]
    is an equivalence as well. Since the morphism $f \colon X \to Y$ is a finite morphism we deduce that
    $\pi_i(\cO_X)$ is a coherent $\pi_0(\cO_Y)$-module, for every $i \ge 0$. By unwinding the definitions, we deduce then that we have a natural identification
        \[
            \pi_i(\cO_{Z'}) \simeq \pi_i0(\cO_Z) \otimes_{\pi_0(\cO_Y)} \pi_i(\cO_X),  
        \]
    in the \infcat $\Mod_{\pi_0(\cO_Z')}$. The result now follows by taking global sections and analyzing the spectral sequence
        \[
            \pi_i(\Gamma(Z', \pi_j(\cO_{Z'})\alg)) \Rightarrow \pi_{i+j}(\Gamma(Z', \cO_{Z'}\alg)), 
        \]
    whose existence is guaranteed by \cite[1.2.2.14]{Lurie_Higher_algebra}.
\end{proof}

\begin{lem} \label{lem:Beck_Chevalley_natural_transformation_equivalence}
    Let $f \colon X \to Y$ be a finite morphism of derived $k$-affinoid spaces and $g \colon Z \to Y$ an admissible open immersion in $\dAfd_k$. Form the pullback diagram
        \[
        \begin{tikzcd}
            Z' \ar{r}{g'}  \ar{d}{f'} & X \ar{d}{f} \\
            Z \ar{r}{g} & Y,
        \end{tikzcd}
        \]
    in the \infcat $\dAfd_k$.
    Then the commutative diagram
        \[
        \begin{tikzcd}
            \Coh^+(Y) \ar{r}{f^*} \ar{d}{g^*} & \Coh^+(X) \ar{d}{(g')^*} \\
            \Coh^+(Z) \ar{r}{(f')^*} & \Coh^+(Z'),
        \end{tikzcd}
        \]
    is right adjointable. In other words, the Beck-Chevalley natural transformation
        \[
            \alpha \colon g^* \circ f_* \to (f')_* \circ g'_* 
        \]
    is an equivalence of functors.
\end{lem}

\begin{proof}
    Since $f$ is assumed to be a finite morphism of derived $k$-affinoid spaces, it follows from the derive Tate aciclicity theorem, c.f. \cite[Theorem 3.1]{Porta_Yu_Derived_Hom_spaces}
    that the right adjoint $f_* \colon \Coh^+(X) \to \Coh^+(Y)$ is well defined. The assertion of the Lemma is now an immediate consequence of \cref{lem:pullbacks_of_derived_affinoid_spaces_along_finite_morphisms_are_algebraic}
    together with \cite[Proposition 2.5.4.5]{Lurie_SAG} and the derived Tate aciclicity theorem.
\end{proof}

\begin{prop} \label{lem:f^*_admits_a_right_adjoint_whenever_f_is_nil-iso}
    Let $f \colon S \to S'$ be a nil-isomorphism between derived $k$-analytic spaces. Then the pullback functor
        \[
            f^* \colon \Coh^+(S') \to \Coh^+(S),  
        \]
    admits a well defined right adjoint, $f_*$.
\end{prop}

\begin{proof}
    Since $f \colon S \to S'$ is a nil-isomorphism, we conclude from \cref{lem:nil-isos_are_affine_morphisms} that $f$ is an affine morphism
    between derived $k$-analytic spaces. By admissible descent of $\Coh^+$, cf. \cite[Theorem 3.7]{Antonio_Porta_Nonarchimedean_Hilbert},
    together with \cref{lem:affine_morphisms_are_compatible_with_Zariski_localization_on_the_base} and \cref{lem:Beck_Chevalley_natural_transformation_equivalence} we reduce the statement of the Lemma to the case
    where both $S$ and $S'$ are equivalent to derived $k$-affinoid spaces. In this case, the result follows by our assumptions on $f$ and \cref{lem:nil-isos_are_affine_morphisms}.
\end{proof}

\begin{cor} \label{lem:pushouts_of_square_zero_extensions_have_the_structure_of_a_square_zero_extension}
    Let $f \colon S \to S'$ be a square-zero extension and $g \colon S \to T$ a nil-isomorphism in $\dAnk$. Suppose we are given a pushout diagram
        \[
        \begin{tikzcd}
            S \ar{r}{f} \ar{d} & S' \ar{d} \\
            T \ar{r} & T'  
        \end{tikzcd},
        \]
    in $\dAnk$. Then the induced morphism $T \to T'$ is a square-zero extension.
\end{cor}

\begin{proof}
    Since $g$ is a nil-isomorphism of derived $k$-analytic spaces, \cref{lem:f^*_admits_a_right_adjoint_whenever_f_is_nil-iso}
    implies that the pullback functor $g^* \colon \Coh^+(T) \to \Coh^+(S)$ admits a well defined right adjoint
        \[
            g_* \colon \Coh^+(S ) \to \Coh^+(T) .
        \]
    Let $\cF \in \Coh^+(S)^{\ge 0}$ and $d \colon \bbL\an_S \to \cF[1]$ be a derivation
    associated with the morphism $f \colon S \to S'$. Consider now the natural composite
        \[
            d ' \colon \bbL\an_T \to g_* (\bbL\an_S) \xrightarrow{g_*(d)} g_* (\cF)[1],  
        \]
    in the \infcat $\Coh^+(T)$. By the universal property of the relative analytic cotangent complex, we deduce the existence of a square-zero extension
        \[
            T \to T',  
        \]
    in the \infcat $\dAnk$. Let $X \in \dAnk$ together with morphisms $S' \to X$ and $T \to X$ compatible with both $f$ and $g$. By the universal property of
    the relative analytic cotangent complex, the morphism $S' \to X$ induces a uniquely defined (up to a contractible indeterminacy space) morphism
        \[
            \bbL\an_{S/X} \to \cF[1],
        \]
    in $\Coh^+(S)$, such that the compositve $\bbL\an_S \to \bbL\an_{S/X} \to \cF[1]$ agrees with $d$. By applying the right adjoint $g_*$ above we obtain a
    commutative diagram
        \[
        \begin{tikzcd}
            \bbL\an_T \ar{r}{\mathrm{can}} \ar{d} & \bbL\an_{T/X} \ar{d} \ar{rd}{d''} & \\
            g_*(\bbL\an_S) \ar{r}{g_*(\mathrm{can})} & g_*(\bbL\an_{S/X}) \ar{r} & g_*(\cF)[1],
        \end{tikzcd}
        \]
    in the \infcat $\Coh^+(T)$. From this, we conclude again by the universal property of the relative analytic cotangent complex the existence
    of a uniquely defined natural morphism $T' \to X$ extending both $T \to X$ and $S' \to X$ and compatible with the restriction to $S$. The latter assertion is equivalent to state
    that the commutative square
        \[
        \begin{tikzcd}
            S \ar{r} \ar{d} & S' \ar{d} \\
            T \ar{r} & T',
        \end{tikzcd}
        \]
    is a pushout diagram in $\dAnk$. The proof is thus concluded.
\end{proof}

\begin{prop} \label{prop:existence_of_pushouts_along_closed_nil-isomorphisms}
    Let $f \colon X \to Y$ be a nil-embedding of derived $k$-analytic spaces. Let
    $g \colon X \to Z$ be a finite morphism in $\dAnk$. The the diagram
        \[
        \begin{tikzcd}
            X \ar{r}{f} \ar{d}{g} & Y \\
            Z
        \end{tikzcd}  
        \]
    admits a pushout in $\dAnk$, denoted $Z'$. Moreover, the natural morphism
    $Z \to Z'$ is also a nil-embedding.
\end{prop}


\begin{proof} The \infcat of $\cTank$-structured \inftopoi \ $\RTop(\cTank)$ is a presentable \infcat. Consider the pushout diagram
        \[
        \begin{tikzcd}
            X \ar{r}{f} \ar{d}{g} & Y \ar{d} \\
            Z \ar{r} & Z',
        \end{tikzcd}
        \]
    in the \infcat $\RTop(\cTank)$. By construction, the underlying \inftopos of $Z'$ can be computed as the pushout in the \infcat $\RTop$ of
    the induced diagram on the underlying \inftopoi \ of $X$, $Z$ and $Y$. Moreover, since $g$ is a nil-isomorphism it induces an equivalence on underlying \inftopoi \
    of both $X$ and $Y$. It follows that the induced morphism $Z \to Z'$ in $\RTop(\cTank)$ induces an equivalence on the underlying \inftopoi.
    Moreover, it follows essentially by construction that we have a natural equivalence
        \begin{align*}
            \cO_{Z'}  & \simeq g_*(\cO_Y) \times_{g_*(\cO_Y)} \cO_Z \\
                      & \in \Str_{\cTank}^\loc(Z).
        \end{align*}
    As the morphism $g_*(\cO_Y) \to g_*(\cO_X)$ is an effective epimorphism and the latter are preserved under fiber products in an \inftopos, c.f. \cite[Proposition 6.2.3.15]{HTT},
    it follows that the natural morphism
        \[
            \cO_{Z'} \to \cO_Z,  
        \]
    is an effective epimorphism, as well.
    Consider now the commutative diagram of fiber sequences
        \[
        \begin{tikzcd}
            \cJ ' \ar{r} \ar{d} & \cO_{Z'} \ar{r} \ar{d}  & \cO_Z \ar{d} \\
            \cJ \ar{r} & g_*(\cO_Y) \ar{r} & g_*(\cO_X),
        \end{tikzcd}
        \]
    in the stable \infcat $\Mod_{\cO_Z'}$. Since the right commutative square is a pullback square it follows that the morphism
        \[
            \cJ' \to \cJ,  
        \]
    is an equivalence. In particular, $\pi_0(\cJ')$ is a finitely generated
    nilpotent ideal of $\pi_0(\cO\alg_{\cJ'})$. Indeed, finitely generation follows from our assumption that $g$ is a finite morphism.
    Thanks to \cref{lem:derived_k_analytic_space_whose_reduction_is_affinoid_is_also_affinoid},
    it follows that $\trunc(Z')$ is an ordinary $k$-analytic space and the morphism $\trunc(Z') \to \trunc(Z)$ is a nil-embedding. We are thus reduced to show that
    for every $i>0$, the homotopy sheaf $\pi_i(\cO_{Z'}) \in \Coh^+(\trunc(Z'))$. But this follows immediately from the existence of a fiber sequence
        \[
            \cO_{Z'} \to g_*(\cO_Y) \oplus \cO_Z \to g_*(\cO_X),  
        \]
    in the \infcat $\Mod_{\cO_{Z'}}$ together with the fact that $g_*(\cO_Y)$ and $g_*(\cO_Z)$ have coherent homotopy sheaves, by our assumption that $g$ is a
    finite morphism combined with \cref{lem:nil-isos_are_affine_morphisms}.
\end{proof}

\begin{defin} \label{defin:analytic_formal_moduli_problems_under}
    An \emph{analytic formal moduli problem under $X$} corresponds to the datum of a functor
        \[F \colon (\anNil_{X/})\op \to \cS,\]
    satisfying the following two conditions:
    \begin{enumerate}
        \item $F(X) \simeq *$ in $\cS$;
        \item Given any pushout diagram
            \[\begin{tikzcd}
                S \ar{r}{f} \ar{d} & S' \ar{d} \\
                T \ar{r} & T',
            \end{tikzcd}\]
        in the \infcat $\anNil_{X/}$, such that $f$ is has the structure of a square zero extension, the induced morphism
            \[F(T') \to F(T) \times_{F(S)}F(S),\]
        is an equivalence in $\cS$.
        \item For each $n \ge 0$, consider the well defined functor
            \[
                 \anNil_{\mathrm{t}_{\le n}(X)/ } \to \anNil_{X/ },  
            \]
        given by the formula
            \[
                (\rt_{\le n}(X) \to S) \in \anNil_{\rt_{\le n}(X)/ } \mapsto ( X \to S \bigsqcup_{\rt_{\le n}(X)} X ).
            \]
        Define $F^{\le n} \colon \anNil_{\rt_{\le n}(X)/ } \to \cS$ as the functor given on objects by the association
            \[
                (rt_{\le n}(X) \to S) \in \anNil_{\rt_{\le n}(X)/ } \mapsto  F^{\le n}(S \bigsqcup_{\rt_{\le n}(X)} X) \in \cS  
            \]
        Given any $S \in \anNil\op_{X/ }$, we then require that the natural morphism
            \[
                F(S) \to \lim_{n \ge 0} F^{\le n}(\rt_{\le n}(S)),  
            \]
        to be an equivalence in $\cS$
    \end{enumerate}
    We shall denote by $\anFMP_{X/}$ the full subcategory of $\Fun((\anNil_{X/ })\op, \cS)$ spanned by analytic formal moduli problems
    under $X$.
\end{defin}

\begin{rem}
    In the previous definition we remark that the functor
        \[
            F^{\le n} \colon \anNil\op_{\rt_{\le n}(X)/ } \to \cS,  
        \]
    is itself an analytic formal moduli problem under $\rt_{\le n}(X)$.
\end{rem}

We shall give some important examples of formal moduli problems under $X$:

\begin{eg}
    \begin{enumerate}
        \item Let $X \in \dAnk$. As in the algebraic case, we can consider the \emph{de Rham pre-stack associated to $X$}, $X_\mathrm{dR} \colon \dAfd_k\op \to \cS$,
        determined by the formula
            \[
                X_{\mathrm{dR}}(Z) \coloneqq X(Z_\red), \quad Z \in \dAfd_k.  
            \]
        We have a natural morphism $X \to X_\mathrm{dR}$ induced from the natural morphism $Z_\red \to Z$.
        We claim that $h_*(X_\mathrm{dR}) \in \Fun(\anNil_{X/}\op, \cS)$ belongs to the full subcategory $\anFMP_{X/}$. Indeed, in this case it is clear that
        $h_*(X_\red)$ is the final object in $\anFMP_{X/}$ which clearly satisfies conditions i) and ii) in \cref{defin:analytic_formal_moduli_problems_under}.
        \item Let $f \colon X \to Y$ be a morphism in the \infcat $\dAnk$. We define the \emph{formal completion of $X$ in $Y$ along $f$} as the pullback 
            \begin{align*}
                Y^\wedge_X  & \coloneqq Y \times_{Y_\mathrm{dR}} X_\mathrm{dR} \\
                            & \in \dAnSt_k.
            \end{align*}
        By construction we have a natural factorization $X \to Y^\wedge_X \to Y$ in $\dAnSt_k$, and moreover the restriction of $X \to Y^\wedge_X$ to the \infcat
        $\anNil\op_{X/}$ along the natural functor
            \[
                \anNil\op_{X/ } \to (\dAnSt_k)_{X/ } \op,  
            \]
        exhibits $Y^\wedge_X$ as a formal moduli problem under $X$.
        \item Let $f \colon X \to Y$ be a closed immersion in the \infcat $\dAnk$. Consider the shifted tangent bundle associated to $f$ together with the zero section
            \[
            \begin{tikzcd}
                X \ar{r}{s_0} \ar{rd}{=} & \bT\an_{X/ Y}[-1] \ar{d}{p} \\
                    & X.
            \end{tikzcd}
            \]
        The completion $\bT\an_{X/Y} [-1]^\wedge_X$ will play an important role in what follows.
    \end{enumerate}
\end{eg}

\begin{notation}
    We set $\anNil_{X/}^\cl \subseteq \anNil_{X/}$ to be the full subcategory spanned by those objects corresponding to nil-embeddings of the form
        \[
            X \to S,  
        \]
    in $\dAnk$.
\end{notation}

\begin{prop} \label{prop:analytic_FMP_under_X_are_ind_inf_schemes}
    Let $Y \in \anNil_{X/}$. The following assertions hold:
    \begin{enumerate}
        \item Then the inclusion functor
            \[
              \anNil^{\cl}_{X//Y} \hookrightarrow \anNil_{X//Y} , 
            \]
        is cofinal.
        \item The natural morphism
            \[
               \colim_{Z \in \anNil_{X//Y}^\cl} Z \to Y,  
            \]
        is an equivalence in $\Fun((\anNil_{X//Y})\op, \cS)$.
        \item The \infcat $\anNil_{X//Y}^\cl$ is filtered.
    \end{enumerate}
\end{prop}

\begin{proof}
    We start by proving claim (i). Let $n \ge 0$, and consider the usual restriction along the natural morphism $X_\red \to \rt_{\le n}(X)$ functor 
        \[\mathbf{res}^{\le n} \colon \anNil_{\rt_{\le n}(X)/} \to \anNil_{X_\red/}.\]
    Such functor admits a well defined left adjoint
        \[\mathbf{push}^{\le n} \colon \anNil_{X_\red/} \to \anNil_{\rt_{\le n}(X)/},\]
    which is determined by the formula
        \[
            (X_\red \to T) \in \anNil_{X_\red /} \mapsto (\rt_{\le n}(X) \to T') \in \anNil_{X/},   
        \]
    where we set
        \begin{equation} \label{eq:set_definition_of_T'_as_pushout_of_T_along_the_inclusion_X_red_to_X}
            T' \coloneqq \rt_{\le n}(X) \bigsqcup_{X_\red} T \in \anNil_{X/}.  
        \end{equation}
    We claim that $T' \in \anNil_{\rt_{\le n}(X)/}$ belongs to the full subcategory $\anNil^\cl_{\rt_{\le n}(X)/} \subseteq \anNil_{\rt_{\le n}(X)/}$.
    Indeed, since the structural morphism
        $X_\red \to T$,
    is necessarily a nil-embedding we deduce the claim from \cref{prop:existence_of_pushouts_along_closed_nil-isomorphisms}.
    We shall denote by
        \[
            \mathbf{res}^{\le n}_!(Y) \colon \anNil\op_{X_\red /} \to \cS,
        \]
    the left Kan extension of $Y$ along the functor $\mathbf{res}^{\le n}$ above. By the colimit formula
    for left Kan extensions, c.f. \cite[Lemma 4.3.2.13]{HTT}, it follows that $\mathbf{res}^{\le n}_!(Y)$ is given by the formula
        \[
            (X_\red \to T) \in  \anNil_{X_\red /} \mapsto Y^{\le n}(T') \in \cS,
        \]
    where $T'$ is as in \eqref{eq:set_definition_of_T'_as_pushout_of_T_along_the_inclusion_X_red_to_X}. 
    We thus have a diagram of functors
        \[
            \mathbf{res}^{\le n} \colon \anNil_{\rt_{\le n}(X)/ /Y^{\le n}} \rightleftarrows \anNil_{X_\red // \mathbf{res}^{\le n}_!(Y)} \colon \mathbf{push}^{\le n},
        \]
    where $\mathbf{res}^{\le n}$ is given on objects by the formula
        \[
            (\rt_{\le n}(X) \to S \to Y^{\le n}) \in \anNil_{\rt_{\le n}(X)/ /Y^{\le n}} \mapsto (X_\red \to S \to \mathbf{res}_!^{\le n}(Y) ) 
        \]
    and the functor $\mathbf{push}^{\le n}$ is given by the association
        \[
            (X_\red \to T \to Y^{\le n}) \in \anNil_{X_\red/ /Y^{\le n}} \mapsto (\rt_{\le n}(X) \to T \sqcup_{X_\red} \rt_{\le n}(X) \to Y^{\le n}) \in \anNil_{\rt_{\le n}(X)/ /Y^{\le n}}.  
        \]
    We claim that the pair $(\mathbf{res}^{\le n}, \mathbf{push}^{\le n})$ form an adjunction. Indeed, a morphism
        \[
            (X_\red \to S \to \mathbf{res}^{\le n}_!(Y)) \to \mathbf{res}_!^{\le n}(\rt_{\le n}(X) \to T \to Y^{\le n}),  
        \]
    corresponds to a commutative diagram
        \[
        \begin{tikzcd}
            X_\red \ar{r} \ar{d}{=} & S \ar{r} \ar{d} & \mathbf{res}_!^{\le n}(Y) \ar{d}{=} \\
            X_\red \ar{r} & T \ar{r} & \mathbf{res}_!^{\le n}(Y),
        \end{tikzcd}
        \]
    in the \infcat $\Fun(\anNil_{X_\red/ }\op, \cS)$. The latter datum is equivalent to the datum of a commutative diagram
        \begin{equation} \label{eq:com_diagram_expliciting_datum_of_res_push_adjunctions}
        \begin{tikzcd}
                                 &                 & \rt_{\le n}(X) \ar{d}   \ar{rd}    &   \\
            X_\red \ar{r} \ar{d} & S \ar{r} \ar{d} & S' \ar{r} \ar{d}                   & Y^{\le n} \ar{d}{=} \\
            X_\red \ar{r}        & T \ar{r}        & T' \ar{r}                          & Y^{\le n}  
        \end{tikzcd}.
        \end{equation}
    Since the morphism $\rt_{\le n}(X) \to T'$ factors through the structural map
        \[
            \rt_{\le n}(X) \to T,  
        \]
    we deduce that the datum of \eqref{eq:com_diagram_expliciting_datum_of_res_push_adjunctions} is equivalent to the datum of a commutative diagram
        \[
        \begin{tikzcd}
            \rt_{\le n}(X) \ar{r} \ar{d} & S' \ar{r} \ar{d} & Y^{\le n} \ar{d}{=} \\
            \rt_{\le n}(X) \ar{r}        & T \ar{r}        & Y^{\le n}, 
        \end{tikzcd}
        \]
    which corresponds to a uniquelly well defined morphism
        \[
            \mathbf{push}^{\le n}(X_\red \to S \to \mathbf{res}^{\le n}_!(Y)) \to (\rt_{\le n}(Y) \to T \to Y^{\le n}),
        \]
    in the \infcat $\anNil_{\rt_{\le n}(X)/ /Y^{\le n}}$. We further observe that for every $n \ge m \ge 0$, the objects
        \[
            \mathbf{res}^{\le n}_!(Y) \quad \mathrm{and} \quad \mathbf{res}^{\le m}_!(Y),  
        \]
    are equivalent as functors $\anNil_{X_\red/ } \op \to \cS$, we shall denote this functor simply by $\mathbf{res}_!(Y)$.
    
    Passing to the limit over $n \ge 0$ we obtain a commutative diagram of the form
        \[
        \begin{tikzcd}
            \anNil_{X_\red/ /\mathbf{res}_!(Y)} \ar{rr} \ar{rd} & & \lim_{n \ge 0} \anNil_{\rt_{\le n}(X)/ /Y^{\le n}} \\
                                                & \anNil_{X/ /Y} \ar{ur}.
        \end{tikzcd}
        \]  
    The horizontal morphism is cofinal since it fits into an adjunction, by our previous considerations. Thanks to \cite[]{HTT} in order to show that the natural morphism
        \[
            \anNil_{X_\red/ /\mathbf{res}_!(Y)} \to \anNil_{X/ /Y},
        \]
    is cofinal, it suffices to prove that $\anNil_{X/ /Y} \to \lim_{n \ge 0} \anNil_{\rt_{\le n}(X)/ /Y^{\le n}}$ is itself cofinal. But the latter is an immediate consequence of
    the fact that derived $k$-analytic spaces are nilcomplete, c.f. \cite[Lemma 7.7]{Porta_Yu_Representability}, together with our assumption (iii) in \cref{defin:analytic_formal_moduli_problems_under}.
    Assertion (i) of the Proposition now follows from the observation that the functor
        \[
            \anNil_{X_\red/ /\mathbf{res}_!(Y)} \to \anNil_{X/ /Y},
        \]
    factors through the full subcategory $\anNil_{X/ /Y}^{\mathrm{cl}} \subseteq \anNil_{X/ /Y}$.
    
    Claim (ii) follows immediately from (i) combined with Yoneda Lemma. To prove (iii) we shall make use of \cite[Lemma 5.3.1.12]{HTT}. Let
        \[
            F \colon \partial \Delta^n \to \anNil^\cl_{X//Y}.  
        \]
    For each $[m] \in \Delta^{n}$, denote by $S_m \coloneqq F([m]) $ in $\anNil^\cl_{X//Y}$. We then have that the pushout
        \[
            S_n \bigsqcup_X S_{n-1},  
        \]
    exists in $\anNil^\cl_{X/}$. We wish to show that $S_n \bigsqcup_X S_{n-1}$ admits a morphism
        \[S_n \bigsqcup_X S_{n-1} \to Y,\]
    compatible with the diagram $F$. In order to prove the latter assertion, we observe that \cref{prop:filtered_colimit_for_nil-embeddings} can filter the diagram $F$ by diagrams $F_i \to F$ such that $X \to F_0$ is formed by square-zero
    extensions and so are each $F_i \to F_{i+1}$. Moreover, by the fact that $Y$ satisfies condition (ii) in \cref{defin:analytic_formal_moduli_problems_under}
    it follows that we can find a well defined morphism
        \[
            S_n \bigsqcup_X S_{n-1} \to Y,  
        \]
    which is compatible with $F$, as desired.
\end{proof}


\begin{defin}
    Let $Y \in \anFMP_{X/}$ denote an analytic formal moduli problem under $X$. The \emph{relative pro-analytic cotangent complex of $Y$ under $X$} is defined as the pro-object
        \begin{align*}
            \bbL\an_{X/Y} &  \coloneqq \{ \bbL\an_{X/Z} \}_{Z \in \anNil^\cl_{X//Y}} \\
                          &  \in \pro(\Coh^+(X)),
        \end{align*}
    where, for each $Z \in \anNil^\cl_{X//Y}$
        \[\bbL\an_{X/Z} \in \Coh^+(X),\]
    denotes the usual analytic relative cotangent complex associated to
    the structural morphism $X \to Z$ in $\anNil^\cl_{X//Y}$.
\end{defin}




\begin{rem}
    Let $Y \in \anFMP_{X/}$. For a general $Z \in \dAnk$, there exists a natural morphism
        \[
            \bbL\an_X \to \bbL\an_{X/Z} ,  
        \]
    in the infcat $\Coh^+(X)$. Passing to the limit over $Z \in \anNil^\cl_{X//Z}$, we obtain a natural map
        \[
            \bbL\an_X \to \bbL\an_{X/Y},  
        \]
    in $\pro(\Coh^+(X))$, as well.
\end{rem}

The following result provides justifies our choice of terminology for the object $\bbL\an_{X/Y} \in \pro(\Coh^+(X))$:

\begin{lem} \label{lem:pro_cot_complex_classifies_nil_extensions_for_analytic_moduli_problems}
    Let $Y \in \anFMP_{X/}$. Let $X \hookrightarrow S$ be a square zero extension associated to an analytic derivation
        \[
            d \colon \bbL\an_S \to \cF [1] ,  
        \]
    where $\cF \in \Coh^+(X)^{\ge 0}$. Then there exists a natural morphism
        \[
            \Map_{\anFMP_{X/}}(S, Y) \to \Map_{\pro(\Coh^+(X))}(\bbL\an_{X/Y}, \cF) \times_{\Map_{\Coh^+(X)}(\bbL\an, \cF)} \{ d \}
        \]
    which is furthermore an equivalence in the \infcat $\cS$.
\end{lem}

\begin{proof}
    Thanks to \cref{prop:analytic_FMP_under_X_are_ind_inf_schemes} combined with the Yoneda Lemma
    we can identify the space of liftings of the map $X \to Y$ along $X \to S$ with the mapping space
        \[
            \Map_{\anFMP_{X/}}(S, Y) \simeq \colim_{Z \in \anNil_{X//Y}} \Map_{\anNil_{X/}} (S, Z).  
        \]
    Fix $Z \in \anNil_{X//Y}^\cl$. Then we have a natural identification of mapping spaces
        \begin{align} \label{eq:identification_of_universal_property_of_rel_an_cotagent_complex}
            \Map_{\anNil_{X/}} (S, Z) & \simeq \Map_{(\dAnk)_{X/}} (S, Z) \\
                                      & \simeq \Map_{\Coh^+(X)}(\bbL\an_{X/Z}, \cF) \times_{\Map_{\Coh^+(X)}(\bbL\an_X, \cF)} \{ d \},
        \end{align}
    see \cite[\S 5.4]{Porta_Yu_Representability} for a justification of the latter assertion.
    Passing to the colimit over $Z \in \anNil_{X//Y}^\cl$, we conclude thanks to the formula for mapping spaces in pro-\infcats that we have
    a natural equivalence
        \[
            \Map_{\anFMP_{X/}}(S, Y) \simeq \Map_{\pro(\Coh^+(X))}(\bbL\an_{X/Y}, \cF) \times_{\Map_{\Coh^+(X)}(\bbL\an, \cF)} \{ d \},
        \]
    as desired.
\end{proof}

\begin{construction} \label{rem:morphisms_of_AnFMP_induce_transition_morphisms_on_relative_analytic_cot_complexes}
    Let $f \colon Y \to Z$ denote a morphism in $\anFMP_{X/}$. Then, for every $S \in \anNil_{X/ /Y}^\cl$ the induced morphism
        \[
            S \to Z,   
        \]
    in $\anFMP_{X/}$ factors necessarily through some $S' \in \anNil^\cl_{X/ /Z}$. For this reason, we obtain a natural morphism
        \[
            \bbL\an_{X/S'} \to \bbL\an_{X/S},  
        \]
    in the \infcat $\Coh^+(X)$. Passing to the limit over $S \in \anNil^\cl_{X//Y}$ we obtain a canonically defined morphism
        \[
            \theta (f) \colon \bbL\an_{X/Z} \to \bbL\an_{X/Y},  
        \]
    in $\pro(\Coh^+(X))$. Moreover, this association is functorial and thus we obtain a well defined functor
        \[
            \bbL\an_{X/ \bullet} \colon \anFMP_{X/ } \to \pro(\Coh^+(X)),  
        \]
    given by the formula
        \[
            (X \to Y) \in \anFMP_{X/ } \mapsto \bbL\an_{X/ Y} \in \pro(\Coh^+(X)).  
        \]
\end{construction}

\begin{prop} \label{prop:conservativity_of_relative_an_cot_complex}
    Let $X \in \dAnk$ be a derived $k$-analytic space. Then the functor
    \[
        \bbL\an_{X/ \bullet} \colon \anFMP_{X/} \to \pro(\Coh^+(X)),
    \]
    obtained via \cref{rem:morphisms_of_AnFMP_induce_transition_morphisms_on_relative_analytic_cot_complexes,rem:morphisms_of_AnFMP_induce_transition_morphisms_on_relative_analytic_cot_complexes},
    is conservative.
\end{prop}

\begin{proof} Let $f \colon Y \to Z $ be a morphism in $\anFMP_{X/}$.
    Thanks to \cref{prop:analytic_FMP_under_X_are_ind_inf_schemes} we are reduced to show that given any
        \[
            S \in \anNil_{X//Z}^\cl,  
        \]
    the structural morphism $g_S \colon X \to S$ admits a unique extension $S \to Y$ which factors the structural morphism $X \to Y$. Thanks to
    \cref{prop:filtered_colimit_for_nil-embeddings} we can reduce ourselves to the case where $X \to S$ has the structure of
    a square zero extension. In this case, the result follows from \cref{lem:pro_cot_complex_classifies_nil_extensions_for_analytic_moduli_problems}
    combined with our hypothesis.
\end{proof}

Our goal now is to give an alternative description of analytic formal moduli problems under $X \in \dAnk$, in terms of derived $k$-analytic stacks: 

\begin{construction} \label{const:anFMP_as_ind_inf_schemes} Consider the \infcat of derived $k$-analytic stacks, $\dAnSt_k$.
    We have a natural functor
        \[
            h \colon \anNil_{X/} \to \dAnk \hookrightarrow \dAnSt_k.
        \]
    Therefore, given any derived $k$-analytic stack $Y$ equipped with a morphism $X \to Y$, one can consider its restriction to the \infcat
    $\anNil_{X/}$:
        \[
            Y \circ h \colon \anNil_{X/} \op \to \cS.      
        \]
    We have thus a natural restriction functor
        \[
            h_* \colon \dAnSt_k \to \Fun(\anNil_{X/} \op, \cS).  
        \]
On the other hand, \cref{prop:analytic_FMP_under_X_are_ind_inf_schemes} allows us to define a natural functor
    \[
        F \colon \anNil_{X/} \op \to \dAnSt_k
    \]
via the formula
    \[
        (X \to Y) \in \anFMP_{X/ } \mapsto \colim_{S \in \anNil_{X/ /Y}^\cl} S,  
    \]
the colimit being computed in the \infcat $\dAnSt_k$. The latter agrees with the left Kan extension of the functor 
    \[
        h \colon \anNil_{X/ } \to \dAnSt_k,  
    \]
along the natural inclusion functor $\anNil_{X/ } \hookrightarrow \anFMP_{X/ }$.
In particular, any analytic formal moduli under $X$ when regarded as a derived $k$-analytic stack can be realized
as an \emph{ind}-\emph{inf}-object, i.e. it can be written as a filtered colimit of nil-embeddings $X \to Z$.
We refer the reader to \cite[\S1]{Gaitsgory_Study_II} for a precise meaning
of the latter notion in the algebraic setting.
\end{construction}





\begin{defin}
    Let $Y \in \dAnSt_k$. We shall say that $Y$ has a deformation theory if it satisfies the following conditions:
    \begin{enumerate}
        \item $Y$ is \emph{nilcomplete}, c.f. \cite[Definition 7.4]{Porta_Yu_Representability};
        \item $Y$ is \emph{infinitesimally cartesian} if it satisfies \cite[Definition 7.3]{Porta_Yu_Representability};
        \item $Y$ admits a \emph{pro-cotangent complex}, i.e., if it satisfies \cite[Definition 7.6]{Porta_Yu_Representability} under the weaker assumption
        that the corresponding derivation functor is pro-corepresentable.
    \end{enumerate}
\end{defin}

\begin{prop} \label{prop:sufficient_conditions_for_a_prestack_to_be_equiv_to_an_analytic_FMP}
    Let $Y \in (\dAnSt_k)_{X/}$. Assume further that $Y$ admits a deformation theory.
    Then $Y$ is equivalent to an analytic formal moduli problem under $X$.
\end{prop}

\begin{proof}
    We must prove that given a pushout diagram
        \[
        \begin{tikzcd}
            S \ar{r}{f} \ar{d}{g} & S' \ar{d} \\
            T \ar{r} & T'  
        \end{tikzcd}
        \]
    in the \infcat $\anNil_{X/}$, where $f$ has the structure of a square-zero extension, then the natural morphism
        \[
            Y(T') \to Y(T) \times_{Y(S)} Y(S'),  
        \]
    is an equivalence in the \infcat $\cS$. Suppose further that $S \hookrightarrow S'$ is associated to some analytic derivation
        \[
            d \colon \bbL\an_S \to \cF[1],
        \]
    for some $\cF \in \Coh^+(S)^{\ge 0}$.
    Thanks to \cref{lem:pushouts_of_square_zero_extensions_have_the_structure_of_a_square_zero_extension} we deduce
    that the induced morphism $T \to T'$ admits a structure of a square-zero extension, as well.
    Then, by our assumptions of $Y$ being infinitesimally cartesian and admitting a relative pro-cotangent complex, we have a chain of natural equivalences of the form.
        \begin{align*}
            Y(T') & \simeq \bigsqcup_{f \colon T \to Y} \Map_{T/ }(T', Y) \\
                  & \simeq \bigsqcup_{f \colon T \to Y} \Map_{\pro(\Coh^+(T))_{\bbL\an_T/ }}(\bbL\an_{T/ Y}, g_*(\cF)[1]) \\
                  & \simeq \bigsqcup_{f \colon T \to Y} \Map_{\pro(\Coh^+(S))_{g^* \bbL\an_T/ }}(g^* \bbL\an_{T/ Y}, \cF[1]) \\
                  & \simeq \bigsqcup_{f \colon T \to Y} \Map_{\pro(\Coh^+(S))_{\bbL\an_S/ }}(\bbL\an_{S/ Y}, \cF[1]) \\
                  & \simeq \bigsqcup_{f \colon t \to Y} \Map_{S/ }(S', Y) \\
                  & \simeq Y(T) \times_{Y(S)} Y(S'),
        \end{align*}
    where the third equivalence follows from the existence of a commutative diagram between fiber sequences
        \[
        \begin{tikzcd}    
            g^* f^*\bbL\an_Y \ar{r} \ar{d}{=} & g^* \bbL\an_{T} \ar{r} \ar{d} &  g^* \bbL\an_{T/Y} \ar{d} \\
            (f\circ g)^*\bbL\an_Y \ar{r} & \bbL\an_{S} \ar{r} & \bbL\an_{S/Y},
        \end{tikzcd}
        \]
    in the \infcat $\pro(\Coh^+(S))$ combined with the fact that the derivation $d_T \colon \bbL\an_T \to g_*(\cF)[1]$ is induced from
        \[
            d \colon \bbL\an_S \to \cF[1],
        \]
    as in the proof of \cref{lem:pushouts_of_square_zero_extensions_have_the_structure_of_a_square_zero_extension}. The result now follows.
\end{proof}

\begin{prop} \label{prop:stacks_under_X_with_deformation_theory_are_analytic_FMP}
    Let $Z \in (\dAnSt_k)_{X/ }$. Suppose further that $Z$ admits a deformation theory. Then the natural morphism
        \[
            \colim_{S \in \anNil_{X/ /Z}^\cl} S \to Z,  
        \]
    in the \infcat $(\dAnSt_k)_{/X}$, is an equivalence.
\end{prop}

\begin{proof}
    We shall prove that for every derived $k$-analytic space $T \in \dAnk$, any diagram of the form
        \[
        \begin{tikzcd}
            X \ar{rr}{f} \ar{rd}{g} & & Z \\
                & T \ar{ru}{h} &  
        \end{tikzcd},
        \]
    factors through an object
        \[
            X \to S \to Z,  
        \]
    in $\anNil_{X/ /Z}^\cl$. Consider the commutative diagram
        \begin{equation} \label{eq:comm_diagram_T_T_red_X_to_Z}
        \begin{tikzcd}
            X_{\red} \ar{r} \ar{d} & T_\red \ar{d} \ar{dr} & \\
            X \ar{r} & T \ar{r} & Z,
        \end{tikzcd}
        \end{equation}
    in the \infcat $\dAnSt_k$. Since $Z_\red \simeq X_\red$ we obtain that $T_\red \to Z$ factors necessarily through the colimit
        \[
            \colim_{S \in \anNil^\cl_{X/ /Z}} S \in \anPreStk.  
        \]
    Assume first that $T$ is bounded, i.e. $T \in \dAnk^{< \infty}$. Then we can construct $T$ out of $T_\red$ via a finite sequence of square-zero extensions,
    as in \cref{prop:filtered_colimit_for_nil-embeddings}. Therefore, in order to construct
    a factorization
        \[
            T \to S \to Z,  
        \]
    in $\anPreStk_{X/}$, we reduce ourselves to the case where the morphism $T_\red \to T$ is itself a square-zero extension.
    In this case, let
        \[
            d \colon \bbL\an_{T_\red} \to \cF[1],  
        \]
    where $\cF \in \Coh^+(T_\red)^{\ge 0}$ be the associated derivation. The existence of the diagram \eqref{eq:comm_diagram_T_T_red_X_to_Z} implies that
    we have a commutative diagram of the form
        \[
        \begin{tikzcd}
                g^*\bbL\an_{T_\red} \ar{r} & \bbL\an_{X_\red} \ar{r} & \bbL\an_{X_\red/ T_\red} \\
                f^* \bbL\an_{Z} \ar{r} \ar{u} & \bbL\an_{X_\red} \ar{u} \ar{r}& \bbL\an_{X_\red/ Z} \ar{u},
        \end{tikzcd}
        \]
    in the \infcat $\pro(\Coh^+(X))$. For this reason, the limit-colimit formula for mapping spaces in pro-\infcats implies that the natural morphism
        \[
            \bbL\an_{X_\red/ Z} \to \bbL\an_{X_\red/ T_\red},
        \]
    in the \infcat $\pro(\Coh^+(X))$ factors necessarily via a morphism of the form
        \[
            \bbL\an_{X_\red/ S} \to \bbL\an_{X_\red/ T_\red},  
        \]
    for some suitable $S \in \anNil^\cl_{X/ /Z}$. Thus the existence problem
        \[
        \begin{tikzcd}
            T_\red \ar{r} \ar{d} & T \arrow[d, dashrightarrow] \ar{rd} & \\
            X_\red = S_\red \ar{r} & S  \ar{r} & Z,
        \end{tikzcd}
        \]
    admits a solution $T \to S \to Z$, as desired. If $T \in \dAnk$ is a general derived $k$-analytic space, we reduce ourselves to the bounded case
    using the fact that $Y$ is nilcomplete.
\end{proof}


\begin{cor}
    The functor
        \[
            F \colon \anFMP_{X/ } \to (\dAnSt_k)_{X/ },  
        \]
    is fully faithful. Moreover, its essential image coincides with those $Z \in (\dAnSt_k)_{X/ }$ which admit deformation theory.
\end{cor}

\begin{proof}
    Let $(\dAnSt_k)^{\mathrm{def}}_{X/ } \subseteq (\dAnSt_k)_{X/ }$ denote the full subcategory spanned by derived $k$-analytic stacks under $X$ admitting
    a deformation theory. It is clear that the natural functor
        \[
            F \colon \anFMP_{X/ } \to (\dAnSt_k)_{X/ },  
        \]
    factors through the full subcategory $(\dAnSt_k)_{X/ }^\mathrm{def}$. Moreover, the restriction functor
        \[
            \mathrm{res} \colon (\dAnSt_k)_{X/ }^\mathrm{def} \to \Fun(\anNil_{X/ }\op, \cS),  
        \]
    factors through $\anFMP_{X/ } \subseteq \Fun(\anNil_{X/ }\op, \cS)$. Moreover,
    \cref{prop:stacks_under_X_with_deformation_theory_are_analytic_FMP} implies that the restriction functor that $F$ and $\mathrm{res}$ are mutually inverse functors,
    proving the claim.
\end{proof}



\subsection{Analytic formal moduli problems over a base}
Let $X \in \dAnk$ denote a derived $k$-analytic space. In \cite[Definition 6.11]{Porta_Yu_NQK}
the authors introduced the \infcat of \emph{analytic formal moduli problems over $X$}, which we shall denote by $\anFMP_{/X}$.

\begin{notation}
    Let $X \in \dAnk$. We shall denote by $\anNil_{/X}$ the full subcategory of $(\dAnk)_{/X}$ spanned by nil-isomorphisms
        \[
            Z \to X,
        \]
    in the \infcat $\dAnk$.
\end{notation}

\begin{defin}
    We shall denote by $\anNil_{/X}^\cl \subseteq \anNil_{/X}$ the faithful subcategory in wich we allow morphisms
        \[
            i \colon S \to S'  ,
        \]
    where $i$ is a nil-embedding in $\dAnk$.
\end{defin}

We start with the analogue of \cref{prop:analytic_FMP_under_X_are_ind_inf_schemes} in the setting of analytic formal moduli problems over $X$:

\begin{prop} \label{prop:required_conditions_for_formal_moduli_problems}
    Let $Y \in \anFMP_{/X}$. The following assertions hold:
    \begin{enumerate}
        \item The inclusion functor
            \[
                (\anNil^{\cl}_{/X})_{/Y}  \to (\anNil_{/X})_{/Y},
            \]
        is cofinal.
        \item The natural morphism
            \[
                \colim_{Z \in (\anNil^\cl_{/X})_{/Y}}  Z \to Y,
            \]
        is an equivalence in the \infcat $\anFMP_{/X}$.
        \item The \infcat $\anNil_{/X}^\cl$ is filtered.
    \end{enumerate}
\end{prop}

\begin{proof}
    We first prove assertion (i). Let $S \to Z$ be a morphism in $(\anNil^\cl_{/X})_{/Y}$. Consider the pushout diagram
        \begin{equation} \label{eq:diagram_pushout_of_nil_isomorphisms_with_ltv_being_the_reduced_subspace}
        \begin{tikzcd}
            S_\red \ar{r} \ar{d} & S \ar{d} \\
            Z \ar{r} & Z',
        \end{tikzcd}
        \end{equation}
    in the \infcat $\anNil_{/X}$ whose existence is guaranteed by \cref{prop:existence_of_pushouts_along_closed_nil-isomorphisms}. Since
    the upper horizontal morphism in \eqref{eq:diagram_pushout_of_nil_isomorphisms_with_ltv_being_the_reduced_subspace} is a nil-embedding, we can reduce
    ourselves via \cref{prop:filtered_colimit_for_nil-embeddings} to the case where the latter is an actual square-zero extension.
    Since $Y$ is assumed to be an analytic formal moduli problem over $X$ we then deduce that the canonical morphism
        \begin{align*}
            Y(Z') & \to Y(Z) \times_{Y(S_\red)} Y(S) \\
            &\simeq Y(Z) \times Y(S),  
        \end{align*}
    is an equivalence (we implicitly used above the fact that $S_\red \simeq X_\red$). As a consequence the object $(Z' \to X)$ in $\anNil_{/X}$ admits an induced morphism
    $Z' \to Y$ making the required diagram commute.
    Thanks \cref{prop:existence_of_pushouts_along_closed_nil-isomorphisms} we deduce that both
    $S \to Z'$ and $Z \to Z'$ are nil-embedding. Therefore, we can factor the diagram
        \[
        \begin{tikzcd}
            S \ar{rr} \ar{rd} & & Z \ar{dl} \\
                &               Y       &  
        \end{tikzcd}
        \]
    via a closed nil-isomorphism $Z \to Z'$. As a consequence, we deduce readily that the inclusion functor
        \[(\anNil^\cl_{/X})_{/Y} \to (\anNil_{/X})_{/Y},\]
    is cofinal.
    It is clear that assertion (ii) follows immediately from (i). We now prove (iii). Let 
        \[\theta \colon K \to (\anNil^\cl_{/X})_{/Y},\]
    be a functor where
    $K$ is a finite \infcat. We must show that $\theta$ can be extended to a functor
        \[\theta^{\rhd} \colon K^{\rhd} \to (\anNil^\cl_{/X})_{/Y}.\]
    Thanks to \cref{prop:filtered_colimit_for_nil-embeddings} we are allowed to reduce ourselves to the case where morphisms indexed by $K$
    are square-zero extensions. The result now follows from the fact that $Y$ being an analytic moduli problem sends finite colimits along square-zero extensions
    to finite limits.
\end{proof}

\begin{lem} \label{formal_moduli_under_induce_formal_moduli_over_via_base_change}
    Let $X \in \dAnk$. Given any $Y \in \anFMP_{X/ }$, then for each $i= 0, 1$ the $i$-th projection morphism
        \[
            p_0 \colon X \times_Y X \to X,  
        \]
    computed in the \infcat $\dAnSt_k$ lies in the essential image of $\anFMP_{/X}$ via \cref{const:anFMP_as_ind_inf_schemes}.
\end{lem}

\begin{proof}
    Consider the pullback diagram
        \[
        \begin{tikzcd}
            X \times_Y X \ar{r}{p_1} \ar{d}{p_0} & X \ar{d} \\
            X \ar{r} & Y ,
        \end{tikzcd}
        \]
    computed in the \infcat $\dAnSt_k$. Thanks to \cref{prop:analytic_FMP_under_X_are_ind_inf_schemes} together with the fact that fiber products commute with filtered colimis in the \infcat $\dAnSt_k$,
    we deduce that
        \[
            X \times_Y X \simeq \colim_{Z \in \anNil^\cl_{X//Y}} X \times_Z X, 
        \]
    in $\dAnSt_k$. It is clear that $(p_i \colon X \times_Z X \to X)$ lies in the essential image of $ \anFMP_{/X}$, for $i = 0, 1$. Thus also the filtered colimit
        \[
            (p_i \colon X \times_Y X \to X) \in \anFMP_{/X}, \quad \mathrm{for \ } i = 0, 1,
        \]  
    as desired.
\end{proof}


Just as in the previous section we deduce that every analytic formal moduli problem over $X$ admits the structure of an \emph{ind}-\emph{inf}-object
in $\anPreStk_k$:

\begin{cor} \label{cor:formal_moduli_problems_over_X_are_ind_inf_objects}
    Let $Y \in (\dAnSt_k)_{/X}$. Then $Y$ is equivalent to an analytic formal moduli problem over $X$ if and only if there exists
    a presentation 
        \[Y \simeq \colim_{i \in I} Z_i,\]
    where $I$ is a filtered \infcat and for every $i \to j$ in $I$, the induced morphism
        \[
          Z_i \to Z_j,  
        \]
    is a closed embedding of derived $k$-affinoid spaces that are nil-isomorphic to $X$.
\end{cor}

\begin{proof}
    It follows immediately from \cref{prop:required_conditions_for_formal_moduli_problems} (ii).
\end{proof}

\begin{defin}
    Let $Y \in \anFMP_{/X}$. We define the \infcat of \emph{coherent modules on $Y$}, denoted $\Coh^+(Y)$, as the limit
        \[
            \Coh^+(Y) \coloneqq \lim_{Z \in (\dAnk)_{/Y}}  \Coh^+(Z),
        \]
    computed in the \infcat $\Catst$. We define the \infcat of \emph{pseudo-pro-coherent modules on $Y$}, denoted $\pro^{\mathrm{ps}}(\Coh^+(Y))$, as
        \[
            \pro^\mathrm{ps}(\Coh^+(Y)) \coloneqq \lim_{Z \in (\anNil_{/X}^\cl)_{/Y}} \pro(\Coh^+(Z)),  
        \]
    where the limit is computed in the \infcat $\Catst$.
\end{defin}

\begin{defin} Let $Y \in \anFMP_{/X}$, $Z \in \dAfd_k$ and let $\cF \in \Coh^+(Z)^{\ge 0}$. Suppose furthermore that we are given a morphism $f \colon Z \to Y$.
    We define the \emph{tangent space of $Y$ at $f$ twisted by $\cF$} as the fiber
        \[
            \bbT\an_{Y, Z, \cF,f} \coloneqq \mathrm{fib}_f\big( Y(Z[\cF]) \to Y(Z) \big) 
            \in \cS.  
        \]
    Whenever the morphism $f$ is clear from the context, we shall drop the subscript $f$ above and denote the tangent space of $Y$ at $f$ simply by $\bbT\an_{Y, Z, \cF}$.
\end{defin}


\begin{rem} \label{rem:construction_of_analytic_cotangent_complex_of_analytic_FMP_over_X} Let $Y \in \anFMP_{/X}$.
    The equivalence of ind-objects
        \[
            Y \simeq \colim_{S \in (\anNil^\cl_{/X})_{/Y}} S,
        \]
    in the \infcat $\dAnSt_k$, implies that, for any $Z \in \dAfd_k$, one has an equivalence of mapping spaces
        \[
            \Map_{\dAnSt_k}(Z, Y) \simeq \colim_{S \in (\anNil^\cl_{/X})_{/Y}}  \Map_{\anPreStk}(Z,S ).
        \]
    For this reason, given any morphism $f \colon Z \to Y$ and any
    $\cF \in \Coh^+(Z)^{\ge 0}$, we can identify the tangent space $\bbT\an_{Y, Z, \cF}$
    with the filtered colimit of spaces
        \begin{align} \label{eq:description_of_cotangent_complex_of_a_formal_moduli_problem_in_terms_of_its_tangent_space}
            \bbT\an_{Y, Z, \cF} & \simeq \colim_{S \in (\anNil^\cl_{/X})_{Z/ /Y}} \fib_f \big( S(Z[\cF]) \to S(Z) \big) \\
                                & \simeq \colim_{S \in (\anNil^\cl_{/X})_{Z/ /Y}} \bbT\an_{S, Z , \cF } \\
                                & \simeq \colim_{S \in (\anNil^\cl_{/X})_{Z/ /Y}} \Map_{\Coh^+(Z)} (f_{S,Z}^* (\bbL\an_S), \cF),
        \end{align}
    where we have denoted by $f_{S, Z} \colon Z \to S$ any morphism, in $(\dAnk)_{/X}$, factoring $f \colon Z \to Y$ such that
        \[(S \to X) \in \anNil^\cl_{/X}.\]
    The final equivalence in \eqref{eq:description_of_cotangent_complex_of_a_formal_moduli_problem_in_terms_of_its_tangent_space}, follows from \cite[Lemma 7.7]{Porta_Yu_Representability}.
    Therefore, we deduce that
    the analytic formal moduli problem
    $Y \in \anFMP_{/X}$ admits an \emph{absolute pro-cotangent complex} given as
        \[
           \bbL\an_Y \coloneqq \{ f^*_{S,Z}( \bbL\an_S )  \}_{Z, S \in (\anNil^\cl_{/X})_{/Y}} \in \pro(\Coh^+(Y)).
        \]
\end{rem}   

\begin{cor}
    Let $Y \in \anFMP_{/X}$. Then its absolute cotangent complex $\bbL\an_Y$ classifies analytic deformations on $Y$. More precisely, given any morphism $Z \to Y$
    where $Z \in \dAfd_k$ and $\cF \in \Coh^+(Z)^{\ge 0}$ one has a natural equivalence of mapping spaces
        \[
            \bbT\an_{Y, Z, \cF} \simeq \Map_{\pro(\Coh^+(Y))}(\bbL\an_Y, \cF).  
        \]
\end{cor}

\begin{proof}
    It follows immediately from the natural equivalences displayed in \eqref{eq:description_of_cotangent_complex_of_a_formal_moduli_problem_in_terms_of_its_tangent_space}
    combined with the description of mapping spaces in \infcats of pro-objects.
\end{proof}

We now introduce the notion of square-zero extensions of analytic formal moduli problems over $X$:

\begin{construction} \label{const:square_zero_extensions_of_analytic_FMP_over_X}
    Let $(f \colon Y \to X) \in \anFMP_{/X}$. Let $d \colon \bbL\an_Y \to \cF[1]$ be an \emph{analytic derivation} in $\pro(\Coh^+(Y))$, where
    $\cF \in \Coh^+(Y)^{\ge 0}$, such that
        \[  
            \cF \simeq f^*(\cF'),
        \]
    for some suitable object $\cF' \in \Coh^+(X)^{\ge 0}$. Thanks to \cref{rem:construction_of_analytic_cotangent_complex_of_analytic_FMP_over_X}
    one has the following natural equivalences of mapping spaces
        \begin{align*}
            \Map_{\pro(\Coh^+(Y))} (\bbL\an_Y , \cF[1])  & \simeq 
            \lim_{S \in (\anNil^\cl_{/X})_{/Y}} \colim_{S' \in (\anNil^\cl_{/X})_{S//Y}} \Map_{\pro(\Coh^+(S))}(f_{S, S'}^*(\bbL\an_{S'}), g_{S}^*(\cF')[1]) \\
                                                         & \simeq
            \lim_{S \in (\anNil^\cl_{/X})_{/Y}} \colim_{S' \in (\anNil^\cl_{/X})_{S//Y}} \Map_{\pro(\Coh^+(S))}(\bbL\an_{S'}, (f_{S, S'})_{ *} g_S^*(\cF')[1]),
        \end{align*}
    where $g_{S} \colon S \to X$ denotes the structural morphism in $\anNil^\cl_{/X}$ and $f_{S, S'} \colon S \to S'$ a given transition morphism
    in the \infcat $(\anNil^\cl_{/X})_{/Y}$. For this reason, we can form the filtered colimit
        \[
            Y' \coloneqq \colim_{S \in (\anNil^\cl_{/X})_{/Y}}  \colim_{S' \in (\anNil^\cl_{/X})_{S//Y}} \oS' \in \dAnSt_k.
        \]
    By construction, one has a natural morphism $Y \hookrightarrow Y'$ in the \infcat $\anPreStk$.
    Moreover, thanks to \cref{prop:sufficient_conditions_for_a_prestack_to_be_equiv_to_an_analytic_FMP} it follows that $Y' \in \anFMP_{/X}$.
\end{construction}

\begin{defin}
    Let $Y \in \anFMP_{/X}$. Suppose we are given an analytic derivation
        \[
            d \colon \bbL\an_Y \to \cF [1],  
        \]
    in $\pro(\Coh^+(Y))$ where $\cF \in \Coh^+(Y)^{\ge 0}$ is such that $\cF \simeq f^*(\cF')$, for some $\cF' \in \Coh^+(X)^{\ge 0}$. We shall
    say that the induced morphism
        \[
            h \colon Y \to Y',  
        \] 
    defined in \cref{const:square_zero_extensions_of_analytic_FMP_over_X}, is a \emph{square-zero extension} of $Y$
    associated to the analytic derivation $d$. 
\end{defin}

\begin{cor} \label{cor:construction_of_square_zero_extensions_for_analytic_FMP_using_univ_property_of_cotangent_complex}
    Let $Y \in \anFMP_{/X}$. Given any square-zero extension $h \colon X \hookrightarrow S$ in $\dAnk$. Then the space of cartesian squares
        \[
        \begin{tikzcd}
            Y \ar{r}{h'} \ar{d}{f} & Y' \ar{d}{g} \\
            X \ar{r}{h} & S,
        \end{tikzcd}
        \]
    such that $h' \colon Y \to Y'$ is a square-zero extension and $g \colon Y' \to S$ exhibits the former
    as an analytic formal moduli problem over $S$ is naturally equivalent to the space of factorizations
        \[
            f^* \bbL\an_X \to \bbL\an_Y \to f^*( \cF')[1],
        \]
    in $\pro(\Coh^+(Y))$, of the analytic derivation $d \colon \bbL\an_X \to \cF'[1]$ associated to the morphism $h$ above.
\end{cor}

\begin{proof}
    By the universal property of filtered colimits together with the fact that these preserve fiber products we reduce the statement to the case where $Y \in \anNil_{/X}$ and thus $Y' \in \anNil_{/S}$, in which
    case the statement follows immediately by the universal property of the relative analytic cotangent complex.
\end{proof}

\begin{cor} \label{cor:universal_property_of_relative_cotangent_complex_for_morphisms_between_analytic_FMP}
    Let $f \colon Z \to X$ be a morphism in the \infcat $\dAnk$. Suppose we are given analytic formal moduli problems
        \[
            f \colon Y \to X \quad \mathrm{and} \quad g \colon \tZ \to Z
        \]
    together with a commutative diagram
        \[
        \begin{tikzcd}
            \tZ \ar{r}{s} \ar{d} & Y \ar{d} \\
            Z \ar{r}{f} & X,  
        \end{tikzcd}
        \]
    in the \infcat $\dAnSt_k$. Let $d \colon \bbL\an_Z \to \cF[1]$, where $\cF \in \Coh^+(Z)^{\ge 0}$, be an analytic derivation corresponding to a square-zero extension morphism
    $Z \to Z'$ in the \infcat $\dAnk$. Denote by
    $\widetilde{d} \colon \bbL\an_{\tZ} \to \cF[1]$ the induced analytic derivation as in \cref{const:square_zero_extensions_of_analytic_FMP_over_X}
    and let $h \colon \tZ \hookrightarrow \tZ'$ be the induced square-zero extension in $\dAnSt_k$ such that we have a cartesian diagram
        \[
        \begin{tikzcd}
            \tZ \ar{r} \ar{d} & \tZ' \ar{d} \\
            Z \ar{r} & Z'  
        \end{tikzcd}
        \]
    in the \infcat $\dAnSt_k$. Then the space of factorizations
        \[
            s \colon \tZ \to \tZ' \to Y,  
        \]
    is naturally equivalent to the space of factorizations
        \[
            \widetilde{d} \colon \bbL\an_{\tZ} \to  \bbL\an_{\tZ/ Y} \to \cF[1]  ,
        \]
    in the \infcat $\pro(\Coh^+(\tZ))$.
\end{cor}

\begin{proof}
    The statement holds true in the case where $\tZ \in \anNil_{/Z}$ and $Y \in \anNil_{/X}$, by the universal property of the relative cotangent complex.
    The general case is reduced to the previous one by a standard argument with ind-objects in $\dAnSt_k$.
\end{proof}


\subsection{Non-archimedean nil-descent for almost perfect complexes}
In this \S, we prove that the \infcat $\Coh^+(X)$, for $X \in \dAnk$ satisfies nil-descent with respect to morphims $Y \to X$, which exhibit
the former as an analytic formal moduli problem over $X$.

\begin{prop} \label{prop:nil_descent_for_Coh^+}
    Let $f \colon Y \to X$, where $X \in \dAnk$ and $Y \in \anFMP_{/X}$. Consider the \v{C}ech nerve
    $Y^\bullet \colon \mathbf \Delta \op \to \dAnSt_k$ associated to $f$.
    Then the natural functor
        \[
            f_\bullet^* \colon \Coh^+(X) \to \lim_{\bDelta \op}(\Coh^+(Y^\bullet)),  
        \]
    is an equivalence of \infcats.
\end{prop}

\begin{proof}
    Consider the natural equivalence of derived $k$-analytic stacks
        \[
            Y \simeq \colim_{Z \in (\anNil^\cl_{/X})_{/Y}}  Z.
        \]
    Then, by definition one has a natural equivalence
        \[
            \Coh^+(Y) \simeq \lim_{Z \in (\anNil^\cl_{/X})_{/Y}} \Coh^+(Z),  
        \]
    of \infcats. In particular, since totalizations commute with cofiltered limits in $\Catinf$, it follows that we can suppose from
    the beginning that $Y \simeq Z$ for some $Z \in \anNil_{/X}$. In this case, the morphism $f \colon Y \to X$ is affine. In particular, the fact that
    $\Coh^+(-)$ satisfies descent along admissible open immersions, combined with \cref{lem:affine_morphisms_are_compatible_with_Zariski_localization_on_the_base} we further reduce ourselves
    to the case where both $X$ and $Y$ are derived $k$-affinoid spaces. In this case, by Tate acyclicity
    theorem it follows that letting $A \coloneqq \Gamma(X, \cO_X \alg)$ and $B \coloneqq \Gamma(Y, \cO_Y\alg),$ the pullback functor $f^*$ can be identified with
    the usual base change functor
        \[
            \Coh^+(A) \to \Coh^+(B).  
        \]
    In this case, it follows that $B$ is nil-isomophic to $A$. Moreover, since the latter are derived noetherian rings
    the statement of the proposition follows due to \cite[Theorem 3.3.1]{preygel_Leistner_Mapping_stacks_properness}.
\end{proof}

We now deduce \emph{pseudo-pro-nil-descent} for moprhisms of the form $Y \to X$, which exhibit $Y$ as an analytic formal moduli problem over $X$:

\begin{cor} \label{cor:pseudo_nil-descent_for_pro_Coh^+}
    Let $X \in \dAnk$ and $f \colon Y \to X$ a morphism in $\dAnSt_k$ which exhibits $Y$ as an analytic formal moduli problem over $X$.
    Then the natural functor 
        \[
            f_\bullet^* \colon \pro(\Coh^+ (X)) \to \lim_{\bDelta \op}(\pro^\mathrm{ps}\Coh^+(Y^\bullet/X)),
        \]
    is fully faithful, where $Y^\bullet$ denotes the \v{C}ech nerve associated to the morphism $f$. Moreover, the essential image of the functor $f_\bullet^*$ identifies canonically with the full subcategory
        \[
            \lim_{\bDelta \op}'\pro^\mathrm{ps}(\Coh^+(X)) \subseteq \lim_{\bDelta \op} \pro^\mathrm{ps}(\Coh^+(Y^\bullet/ X)),
        \]
    spanned by those $\{\cF_{i, [n]} \}_{i \in I\op, [n]} \in \lim_{\bDelta}(\pro^\mathrm{ps}(\Coh^+(Y^\bullet/X)))$, for some filtered \infcat $I$, which belong to the essential image of the natural
    functor
        \[
            \lim_{\bDelta \op} \Fun(I\op, \Coh^+(Y^\bullet/X)) \to \lim_{\bDelta \op} \pro^\mathrm{ps}(\Coh^+(Y^\bullet/X)) . 
        \]
\end{cor}

\begin{proof}
    By the very definition of the \infcat $\pro^\mathrm{ps}(\Coh^+(Y))$, we reduce ourselves as in
    \cref{prop:nil_descent_for_Coh^+} to the case where $Y = S$, for some $S \in \anNil_{/X}$. In this case, it follows readily from
    \cref{prop:nil_descent_for_Coh^+} that the natural functor
        \[
            f_\bullet^* \colon \pro(\Coh^+(X)) \to \lim_{\bDelta \op} \pro^\mathrm{ps}(\Coh^+(Y^\bullet/X)),  
        \]
    is fully faithful. We now proceed to prove the second claim of the corollary. Notice that,
    \cref{lem:f^*_admits_a_right_adjoint_whenever_f_is_nil-iso} implies that there exists a well defined right adjoint
        \[
            f_* \colon \Coh^+(S) \to \Coh^+(X),  
        \]
    to the usual pullback functor $f^* \colon \Coh^+(X) \to \Coh^+(S)$. We can extend the right adjoint $f_*$ to a well defined functor
        \[
            f_* \colon \pro(\Coh^+(S)) \to \pro(\Coh^+(X)),  
        \]
    which commutes with cofiltered limits. For this reason, we have a well defined functor
        \[
            f_{\bullet, *} \colon \lim_{\bDelta\op}(\pro(\Coh^+(Y^\bullet/X))) \to \pro(\Coh^+(X)),  
        \]
    which further commutes with cofiltered limits. We claim that $ f_{\bullet, *}$ is a right adjoint to $f_\bullet^*$ above. Indeed, given
    any $\{ \cF_i \}_{i \in I\op}  \in \pro(\Coh^+(X))$ and $\{ \cG_{j, [n]} \}_{j \in J_{[n]} \op, [n] \in \bDelta \op} \in \lim_{\bDelta \op}(\pro(\Coh^+(Y^\bullet/X)))$,
    we compute
        \begin{align*}
            \Map_{\lim_{\bDelta \op}(\pro^\mathrm{ps}(\Coh^+(Y^\bullet/X)))}( f_\bullet^*( \{ \cF_i \}_{i \in I\op}), \{ \cG_{j, [n]} \}_{j \in J\op_{[n]}, [n] \in \bDelta \op}) & \simeq \lim_{[n] \in \bDelta \op} \Map_{\pro^\mathrm{ps}(\Coh^+(Y^{[n]}))} ( \{f_{[n]}^\bullet(\cF_i) \}_{i \in I\op},  \{ \cG_{i, [n]} \}_{i \in I\op_{[n]}}) \\
            \lim_{[n] \in \bDelta \op} \lim_{j \in J\op_{[n]}} \mathrm{colim}_{i \in I} \Map_{\Coh^+(Y^{[n]})} ( f_{[n]}^*(\cF_i),   \cG_{i, [n]}) & \simeq \lim_{[n] \in \bDelta \op} \lim_{j \in J\op_{[n]}} \colim_{i \in I} \Map_{\Coh^+(X)} (\cF_i, f_{[n], *}(\cG_{i, [n]})) \\
            \lim_{[n] \in \bDelta \op} \Map_{\pro(\Coh^+(X))}( \{\cF_i\}_{i \in I\op}, \{f_{[n], *}(\cG_{i, [n]}) \}_{i \in I_{[n]} \op}) & \simeq \Map_{\pro(\Coh^+(X))}( \{\cF_i\}_{i \in I\op}, \lim_{[n] \in \bDelta \op} \{f_{[n], *}(\cG_{i, [n]}) \}_{i \in I_{[n]} \op}),
        \end{align*}
    as desired. 
    It is clear that the functor $f_{\bullet
    }^*$ above factors through the full subcategory
        \[
            \lim_{\bDelta \op}'(\pro^\mathrm{ps}( \Coh^+(Y^\bullet / X)))  \subseteq \lim_{\bDelta \op}(\pro(\Coh^+(Y^\bullet/X))).
        \]
    For this reason, the pair $(f_\bullet^*, f_{\bullet, *})$ restricts to a well defined adjunction
        \[
            f^*_{\bullet} \colon \pro(\Coh^+(X))  \rightleftarrows  \mathrm{Tot}'(\pro(\Coh^+(Y^\bullet/ X))) \colon f_{\bullet, *}.  
        \]
    In order to conclude, we will show that the functor
        \[
            f_{\bullet, *} \colon  \mathrm{Tot}'(\pro(\Coh^+(Y^\bullet/X))) \to \pro(\Coh^+(X)) ,  
        \]
    is conservative. Since both the \infcats $\pro(\Coh^+(X))$ and $\lim_{\bDelta \op}'(\pro^\mathrm{ps}(\Coh^+(Y^\bullet/X)))$ are stable, we are reduced to prove that given any
        \[
            \{ \cG_{i, [n]} \}_{i \in I\op} \in \lim_{\bDelta \op}'(\pro^\mathrm{ps}(\Coh^+(Y^\bullet/X))),  
        \]
    such that 
        \begin{equation} \label{eq:conservativity_of_lim_f_bullet_*}
            \lim_{[n] \in \bDelta \op} f_{\bullet, *}( \{ \cG_{i, [n]} \}_{i \in I\op})  \simeq 0,
        \end{equation}
    we necessarily have
        \[
            \{ \cG_{i, [n]} \}_{i \in I\op} \simeq 0  ,
        \]
    in $\lim_{\bDelta \op}'(\pro^\mathrm{ps}(\Coh^+(Y^\bullet / X)))$.
    Assume then \eqref{eq:conservativity_of_lim_f_bullet_*}. Under our hypothesis, for each index $i \in I$, the object
    $\{ \cG_{i, [n]} \}_{ [n] \in \bDelta \op}$ satisfies descent datum and thanks to \cref{prop:nil_descent_for_Coh^+} it produces a uniquely well defined object
        \[
            \cG_i \in \pro(\Coh^+(X)),  
        \]
    such that for every $[n] \in \bDelta \op$, one has a natural equivalence of the form
        \begin{align*}
            f_{[n]}^* (\cG_i) & \simeq \cG_{i , [n]} \\
                              & \in \Coh^+(Y^{[n]}).
        \end{align*}
    We deduce then that  
        \begin{align*}
            f^*_{\bullet}(\{ \cG_i \}_{i \in I\op}) & \simeq \{\cG_{i, [n]} \}_{i \in I\op, [n]} \\
                                                    & \simeq 0,  
        \end{align*}
    in $\lim'_{\bDelta \op} \pro^\mathrm{ps}(\Coh^+(Y^\bullet/ X))$, as desired.
\end{proof}

We now use the pseudo-pro-nil-descent for $\pro(\Coh^+(X))$ to compute relative analytic cotangent complexes of
analytic formal moduli problems over $X$:

\begin{cor} \label{cor:analytic_relative_cotangent_complex_defines_cartesian_sections_in_the_totalization}
    Let $f \colon Z \to X$ be a morphism in $\dAnk$. Suppose we are given a pullback square
        \[
        \begin{tikzcd}
            \tZ \ar{r}{h} \ar{d} & Y \ar{d} \\
            Z \ar{r} & X,
        \end{tikzcd}
        \]
    in the \infcat $\dAnSt_k$, where $( Y \to X) \in \anFMP_{/X}$ and $(g \colon\tZ \to Z) \in \anFMP_{/Z}$. Then the lax-limit object
        \[
            \{ \bbL\an_{\tZ^{[n]} / Y^{[n]}} \} \in \lim^\mathrm{lax}(\pro^\mathrm{ps}(\Coh^+(\tZ^\bullet/Z))),  
        \]
    defines an actual cartesian section
        \[
            \{ \bbL\an_{\tZ^{[n]}/ Y^{[n]}} \} \in \lim_{\bDelta \op}\pro(\Coh^+((\tZ)^\bullet /Z)),
        \]
    which belongs to the essential image of the natural functor
        \[
            g^*_\bullet \colon \pro(\Coh^+(Z)) \to \lim_{\bDelta \op}\pro(\Coh^+(\tZ^\bullet / Z)).
        \]
\end{cor}

\begin{proof}
    We first show that the object
        \[\{ \bbL\an_{(\tZ)^{[n]}/ Y^{[n]}} \}  \in \lim_{\bDelta \op}^\mathrm{lax}\pro^\mathrm{ps}(\Coh^+((\tZ)^\bullet/ Z)),\]
    defines a cartesian section in
        \[
            \lim_{\bDelta \op} \pro(\Coh^+((\tZ)^\bullet / Z)).  
        \]
    In order to show this assertion, it is sufficient to prove for every $[n ] \in \bDelta \op$, that we have a natural equivalence
        \[
            h^*(\bbL\an_{\tZ^{[n]} / Y^{[n]}}) \simeq \bbL\an_{\tZ^{[n+1]}/ Y^{[n+1]}},  
        \]
    in the \infcat $\pro^\mathrm{ps}(\Coh^+((\tZ)^{[n+1]}))$. The latter claim is an immediate consequence of the base change property for the analytic
    cotangent complex in the case where $Y \in \anNil_{/X}$ (and thus so do $\tZ \in \anNil_{/Z})$), which follows readily from
    \cite[Proposition 5.12]{Porta_Yu_Representability}. In the general case where $Y \in \anFMP_{/X}$, we reduce to the previous case
    by combining \cref{prop:required_conditions_for_formal_moduli_problems} with the observation that filtered colimits
    commute with finite limits in the \infcat $\dAnSt_k$. 

    We now prove the second assertion of the Corollary. Thanks to the characterization of the essential image of natural functor
        \[
            g^*_{\bullet} \colon \pro(\Coh^+(Z)) \to \lim_{\bDelta \op} \pro^\mathrm{ps}(\Coh^+((\tZ)^\bullet /Z)),  
        \]
    provided in \cref{cor:pseudo_nil-descent_for_pro_Coh^+}, we are reduced to show that for each $[n] \in \bDelta \op$, we have a natural
    equivalence of pro-objects
        \[
            \bbL\an_{\tZ^{[n]} / Y^{[n]}} \simeq \{ \bbL\an_{\tS^{[n]}/ S^{[n]}} \}_{\tS \in (\anNil_{/ Z})_{/\tZ}, \ S \in (\anNil_{/X})_{/ Y}}.
        \]
    The latter statement follows readily from the first part of the proof by a direct inductive argument.
\end{proof}


\subsection{Non-archimedean formal groupoids} Let $X \in \dAn_k$.
We start with the definition of the notion of \emph{analytic formal groupoids over $X$}:

\begin{defin}We denote by
    $\mathrm{AnFGrpd}(X)$ the full subcategory of the \infcat of simplicial objects
        \[
            \Fun( \bDelta \op, \anFMP_{/X}),
        \]
    spanned by those objects $F \colon \mathbf \Delta \op \to \anFMP_{/X}$ satisfiying the following requirements:
        \begin{enumerate}
            \item $F([0]) \simeq X$ ;
            \item For each $n \ge 1$, the morphism
                \[
                    F([n]) \to F([1]) \times_{F([0])} \dots \times_{F([0])} F([1])  ,
                \]
            induced by the morphisms $s^i \colon [1] \to [n]$ given by $(0,1) \mapsto (i, i+1)$, is an equivalence
            in $\anFMP_{/X}$.
        \end{enumerate}
    We shall refer to objects in $\anFGrpd(X)$ as \emph{analytic formal groupoids over $X$}.
\end{defin}

\begin{rem}
    Note that \cref{prop:required_conditions_for_formal_moduli_problems} implies that fiber products exist in $\anFMP_{/X}$. Therefore, the previous
    definition is reasonable.
\end{rem}



\begin{construction} \label{const:formal_completion_construction_Phi} Thanks to \cref{formal_moduli_under_induce_formal_moduli_over_via_base_change},
there exists a well defined functor $\Phi \colon \anFMP_{X/} \to \anFGrpd(X)$ given by the formula
    \[
        (X \to Y ) \in \anFMP_{X/ } \mapsto Y^\wedge_X \in \anFGrpd(X),
    \]
where $Y^\wedge_X \in \anFGrpd(X)$ denotes the analytic formal groupoid over $X$ admitting
    \[
    \begin{tikzcd}
      \dots \ar[r, shift left=2] \ar[r, shift left=0.75] 
      \ar[r, shift left=-0.75] \ar[r, shift left=-2]
      & X \times_Y X \times_Y X \ar[r, shift left=-1] \ar[r, shift left=1] \ar[r] 
      & X \times_Y X \ar[r, shift left=-0.5ex] \ar[r, shift left=0.5ex] 
      & X 
    \end{tikzcd},
    \] 
as simplicial presentation.
\end{construction}

\begin{construction} \label{const:formal_classifying_stack_construction}
    Let $\cG \in \anFGrpd(X)$. Consider the \emph{$k$-analytic classifying pre-stack}, $\rB_X(\cG)^{\mathrm{pre}} \in \dAnSt_k$, obtained as the geometric realization
    of the simplicial object $\cG$, regarded naturally as a functor
        \[
            \cG \colon \bDelta \op \to \dAnSt_k.  
        \]
    Given any $Z \in \dAfd_k$, the \emph{space of $Z$-points of $\rB_X(\cG)^{\mathrm{pre}}$},
        \[
            \rB_X(\cG)^\mathrm{pre}(Z),  
        \]
    can be identified with the space whose objects correspond to the datum of:
        \begin{enumerate}
            \item A morphism $\tZ \to X$, where $\tZ \in \dAnSt_k$, such that
                \[
                    \tZ \simeq Z \times_{\rB_X(\cG)^\mathrm{pre}} X;  
                \]
            \item A morphism of groupoid-objects
                \[\tZ \times_Z \tZ \to \cG,\]
                in the \infcat $\dAnSt_k$.
        \end{enumerate}
    We now define $\rB_X(\cG) \to \rB^\mathrm{pre}_X(\cG)$ as the sub-object spanned by those connected components of $\rB_X(\cG)^\mathrm{pre}$ corresponding to
    morphisms $\tZ \to Z$ in \cref{const:formal_classifying_stack_construction} (i)
    which exhibit $\tZ \in \anFMP_{/Z}$. Denote by
        \[
            \mathrm{can} \colon \rB_X(\cG) \to \rB_X(\cG)^\mathrm{pre},  
        \]
    the canonical morphism. It follows from the constructions that the natural morphism    
        \[
            X \to \rB_X(\cG)^\mathrm{pre},  
        \]
    factors as $X \to \rB_X(\cG) \xrightarrow{\mathrm{can}} \rB_X(\cG)^\mathrm{pre}$.
\end{construction}

\begin{lem}
    The natural morphism $X \to \rB_X(\cG)$ exhibits the latter as an object in the \infcat $\anFMP_{X/}$ of analytic formal moduli problems
    under $X$.
\end{lem}

\begin{proof}
    Thanks to \cref{prop:sufficient_conditions_for_a_prestack_to_be_equiv_to_an_analytic_FMP} it suffices to prove that $\rB_X(\cG)$ is
    infinitesimally cartesian and it admits furthermore a pro-cotagent complex. The fact that $\rB_X(\cG)$ is infinitesimally cartesian follows from the
    modular description
    of $\rB_X(\cG)$ combined with the fact that $\cG$ is infinitesimally cartesian, as well. Similarly, $\rB_X(\cG)$ being nilcomplete follows again from its
    modular description combined with the fact that analytic formal moduli problems are nilcomplete.
    
    
    We are thus required to show that $\rB_X(\cG)$ admits
    a \emph{global} pro-cotangent complex. Let $Z \in \dAnk$ and suppose we are given an arbitrary morphism
        \[
            q \colon Z \to \rB_X(\cG),    
        \]
    in the \infcat $\dAnSt_k$.
    Thanks to \cref{cor:pseudo_nil-descent_for_pro_Coh^+} combined with \cref{cor:analytic_relative_cotangent_complex_defines_cartesian_sections_in_the_totalization}
    it follows that the object
        \[
            \{ \bbL\an_{\tZ^{[n]} / \cG^{[n]}} \}_{[n] \in \bDelta \op} \in \lim_{\bDelta \op} \pro^\mathrm{ps}(\Coh^+(\tZ^\bullet / Z)), 
        \]
    defines a well defined object $ \bbL^{\mathrm{an}'}_{Z/ \rB_X(\cG)} \in \pro(\Coh^+(Z))$. Moreover, it is clear that there exists a natural morphism
        \[
            \theta \colon \bbL\an_Z \to \bbL^{\mathrm{an} '}_{Z /  \rB_X(\cG)},
        \]
    in the \infcat $\pro(\Coh^+(Z))$. Let
        \[
            q^* \bbL^{\mathrm{an} '}_{\rB_X(\cG)}  \coloneqq \mathrm{fib}(\theta),
        \]
    computed in the \infcat $\pro(\Coh^+(Z))$. We claim that $q^* \bbL^{\mathrm{an} '}_{\rB_X(\cG)}$ identifies with the analytic cotangent
    complex of $\rB_X(\cG)$ at the point $q \colon Z \to \rB_X(\cG)$. Let
        \[
            Z \hookrightarrow Z',  
        \]
    denote a square-zero extension which corresponds to a certain analytic derivation
        \[
            d \colon \bbL\an_Z \to \cF[1],  
        \]
    for some $\cF \in \Coh^+(Z)^{\ge 0}$. Using \cref{cor:construction_of_square_zero_extensions_for_analytic_FMP_using_univ_property_of_cotangent_complex}
    we deduce that the space of cartesian squares of the form 
        \[
        \begin{tikzcd}
            \tZ \ar{r} \ar{d} & \tZ' \ar{d} \\
            Z \ar{r} & Z'
        \end{tikzcd}
        \]
    where the morphism $\tZ \to \tZ'$ is a square-zero extension in the \infcat $\dAnSt_k$ is equivalent to the space of factorizations
        \[
            d \colon g^*  \bbL\an_Z \to \bbL\an_{\tZ} \xrightarrow{{d'}} g^*(\cF)[1],
        \]
    in the \infcat $\pro(\Coh^+(\tZ))$. Apply the same reasoning to the each object in the \v{C}ech nerve
        \[
            \tZ^\bullet \to Z.  
        \]
    Furthermore, \cref{cor:universal_property_of_relative_cotangent_complex_for_morphisms_between_analytic_FMP}
    implies that the space of factorizations
        \[
            \tZ^\bullet \to (\tZ')^\bullet \to \cG^\bullet,  
        \]
    identifies with the space of factorizations
        \[
            d' \colon \bbL\an_{\tZ} \to \bbL\an_{\tZ/ \cG} \to g^*(\cF)[1],
        \]  
    in the \infcat $\pro(\Coh^+(\tZ))$. By pseudo-pro-nil-descent we then deduce that the above factorization space identified with the space
    of factorizations
        \[
            d \colon \bbL\an_Z \to \bbL^{\mathrm{an}, '}_{Z/ \rB_X(\cG)} \to \cF[1].
        \]
    This implies that $\bbL^{\mathrm{an}, '}_{Z/ \rB_X(\cG)}$ satisfies the universal property of the relative analytic pro-cotangent complex, as desired.
\end{proof}

\begin{thm} \label{thm:Phi_is_an_equivalence}
    The functor $\Phi \colon \anFMP_{/X} \to \anFGrpd(X)$ of \cref{const:formal_completion_construction_Phi} is an equivalence of \infcats.
\end{thm}

\begin{proof} Let $\cG \in \anFGrpd(X)$.
    Thanks to (1) in \cref{const:formal_classifying_stack_construction} it follows that one has a canonical equivalence
        \[
            X \times_{\rB_X(\cG)} X \simeq \cG,
        \]
    in $\dAnSt_k$. This shows that the construction
        \[
            \rB_X(\cG) \colon \anFGrpd(X) \to \anFMP_{X/},  
        \]
    is a right inverse to $\Phi$. As a consequence the functor $\Phi$ is essentially surjective. By the same reasoning we deduce that
    given $X \to Y $ in $\anFMP_{X/}$ the natural morphism
        \[
              Y \to \rB_X(Y \times_X Y),
        \]
    is also an equivalence in $\dAnSt_kDERIVED NON-ARCHIMEDEAN ANALYTIC HILBERT SPAC$.
\end{proof}

\subsection{The affinoid case} Let $X \in \dAfdk$ denote a derived $k$-affinoid space.
Thanks to derived Tate aciclycity theorem, cf. \cite[Theorem 3.1]{Porta_Yu_Derived_Hom_spaces}
the \emph{global sections functor}
    \[
        \Gamma \colon \Coh^+(X) \to  \Coh^+(A),
    \]
where $A \coloneqq \Gamma(X, \cO_X \alg) \in \CAlg_k$, is an equivalence of \infcats. Since ordinary $k$-affinoid algebras are Noetherian, we deduce that
$A \in \CAlg_k$ is a Noetherian derived $k$-algebra.

\begin{notation}
    Let $X \in \dAfd_k$ and $A \coloneqq \Gamma(X, \cO_X)$. We denote by
        \[
            \mathrm{FMP}_{/\Spec A} \in \Catinf,  
        \]
    the \infcat of \emph{algebraic formal moduli problems over $\Spec A$}, (c.f. \cite[Definition 6.11]{Porta_Yu_NQK}).
\end{notation}

\begin{thm}$\mathrm{(}$\cite[Theorem 6.12]{Porta_Yu_NQK}$\mathrm )$ \label{thm:equivalence_between_analytic_and_algebraic_formal_moduli_problems_over}
    Let $X \in \dAfdk$ and $A \coloneqq \Gamma(X, \cO_X\alg) \in \CAlg_k$. Then the induced functor
        \[
            (-) \an \colon \mathrm{FMP}_{/ \Spec A} \to \anFMP_{/X},
        \]
    is an equivalence of \infcats.
\end{thm}

As an immediate consequence, we obtain the following result:

\begin{cor} \label{cor:equivalence_between_pointed_analytic_and_algebraic_formal_moduli_problems_over}
    Let $X \in \dAfd_k$ and $A \coloneqq \Gamma(X, \cO_X\alg)$. Then one has an equivalence of \infcats
        \[
            \mathrm{FMP}_{\Spec A/ /\Spec A} \to \anFMP_{X/ /X} ,
        \]
    of pointed algebraic formal moduli problems over $\Spec A$ and pointed analytic formal moduli problems over $X$, respectively.
\end{cor}

\begin{proof}
    It is an immediate consequence of \cref{thm:equivalence_between_analytic_and_algebraic_formal_moduli_problems_over}. Indeed, equivalences of \infcats with final objects
    induce natural equivalences on the associated \infcats of pointed objects.
\end{proof}


\begin{cor}
    Let $X \in \dAfd_k$. Then both the \infcats $\anFGrpd(X)$ and $\anFMP_{X/}$ admit sifted colimits.
\end{cor}

\begin{proof}
    Thanks to \cref{thm:Phi_is_an_equivalence} we are reduced to prove solely that $\anFGrpd(X)$ admits sifted colimits. Let $F \colon I \to \anFGrpd(X)$
    denote a functor, where $I$ is a sifted \infcat. Then, for each $i \in I$ and $[n] \in \bDelta \op$ we have that
        \[
            F(i)_{[n]} \in \mathrm{Ptd}(\anFMP_{/X}).  
        \]
    Since the \infcat $\mathrm{Ptd}(\mathrm{FMP}_{/ \Spec A})$ admits sifted colimits, see for instance \cite[\S 5, Corollary1.6.6]{Gaitsgory_Study_II}, we deduce
    thanks to \cref{cor:equivalence_between_pointed_analytic_and_algebraic_formal_moduli_problems_over} that so does the \infcat $\mathrm{Ptd}(\anFMP_{/ X})$.
    For this reason, we deduce that for each $[n] \in \bDelta \op$, the object
        \[
            Y_{[n]} \coloneqq \colim_{i \in I} F(i)_{[n]} \in \mathrm{Ptd}(\anFMP_{/X}),  
        \]
    is well defined. Moreover, the simplicial object
        \[
            \{ Y_{[n]} \}_{[n] \in \bDelta \op},  
        \]
    in the \infcat $\mathrm{Ptd}(\anFMP_{/X})$ forms an analytic formal groupoid over $X$, since sifted colimits commute with finite products. Moreover, it follows from the
    definitions that the object $\{Y_{[n]} \}_{[n]}$ is a sifted colimit of the diagram $F \colon I \to \anFGrpd(X)$, as desired.
\end{proof}

Let $Y \in \anFMP_{X/}$, we can thus identify
the associated relative pro-cotangent complex
    \[
        \bbL\an_{X/Y} \in \pro(\Coh^+(A)).  
    \]
Consider the diagram
    \[
    \begin{tikzcd}
        X \ar{r}{\Delta}  \ar{rd}[swap]{=} & X \times_Y X \arrow[d, "p_0", shift left=2] \arrow{d}[swap]{p_1} \\
        &   X  
    \end{tikzcd}
    \]
in the \infcat $\anPreStk$, where $\Delta \colon X \to X \times_Y X$ denotes the usual \emph{diagonal embedding}. We then obtain a natural
fiber sequence associated to the above diagram of the form
    \begin{equation} \label{eq:fiber_sequence_for_rel_cot_complexes_along_diagonal_and_projections}
        \Delta^* \bbL\an_{X \times_Y X/ X} \to \bbL\an_{X/X} \to \bbL\an_{X/ X \times_Y X}.
    \end{equation}
Notice further that by \cite[Proposition 5.12]{Porta_Yu_Representability} one has an equivalence
    \[
        \bbL\an_{X \times_Y X/ X} \simeq p_i^* \bbL\an_{X/Y},  
    \]
for $i =0, 1$. We further deduce that
    \[\bbL\an_{X \times_Y X/ X} \simeq \bbL\an_{X/Y},\]
in the \infcat $\pro(\Coh^+(A))$. Moreover, since $\bbL\an_{X/ X} \simeq 0$, we obtain from the fiber sequence \eqref{eq:fiber_sequence_for_rel_cot_complexes_along_diagonal_and_projections}
a natural equivalence
    \[
        \bbL\an_{X/ X \times_Y X} \simeq \bbL\an_{X / Y} [1],    
    \]
in $\pro(\Coh^+(A))$. Moreover, we can identify the $\bbL\an_{X \times_Y X / X}$ with the pro-object  
    \[
        \bbL\an_{X/ X \times_Y X } \simeq \{ \bbL\an_{X/ X \times_S X} \}_{S \in \anNil_{X//Y}^\cl} ,
    \]
where $\bbL\an_{X/ X\times_S X} \in \Coh^+(A)$ denotes the relative analytic cotangent complex associated to the closed embedding
    \[
        X \to X \times_S X,  
    \]
for $S \in \anNil_{X/Y}^\cl$. Thanks to \cite[Corollary 5.33]{Porta_Yu_Representability} we deduce that
    \[
        \bbL\an_{X/X\times_SX} \simeq \bbL_{A/ A \widehat{\otimes}_{B_S} A} ,
    \]
in $\Coh^+(A)$, where $B_S = \Gamma(S, \cO\alg_S)$. \personal{Notice that we can remove the hat in the previous tensor product
since under our assumptions $A \in \Coh^+(B_S)$, for every such considered $S$.}

\begin{notation}
    Let $X \in \dAfd_k$. We shall denote by
        \[
            \QCoh(X) \coloneqq \Mod_A,  
        \]
    where $A \coloneqq \Gamma(X, \cO_X\alg)$. The latter can be naturally identified with the stable \infcat $\Ind(\Perf(A))$.
\end{notation}

\begin{notation}
    Let $X \in \dAfd_k$. We shall denote the \emph{plain duality functor} as
        \[
            (-)^\vee \colon \Coh^+(X) \op  \to \QCoh(X), 
        \]
    which is given on objects by the formula
        \[
            \cF \in \Coh(X)\op \mapsto \Map_{\QCoh(X)}(\cF, \cO_X\alg) \in \Perf(X).  
        \]
    We can extend the latter functor via filtered colimits to a functor on ind-completions:
        \begin{align*}
            (-)^\vee \colon \pro(\Coh^+(X)) \op & \simeq \Ind(\Coh^+(X)\op) \\
                                                & \simeq \QCoh(X).
        \end{align*}
    The latter associates to each ind-colimit
        \begin{align*}
            \cF     & \simeq \colim_{i \in I } \cF_i \\
                    & \in \Ind(\Coh^+(X)\op),  
        \end{align*}
    where $I$ is a filtered \infcat and for each $i \in I$, $\cF_i \in \Coh^+(X)$, the filtered colimit object
        \[
            \colim_{i \in I} \cF_i^\vee   
        \]
    computed in the presentable \infcat $\QCoh(X)$.
\end{notation}

\begin{defin}[Tangent complex] Let $Y \in \anFMP_{X/}$. We define the \emph{relative analytic tangent complex} of $X \to Y$ as the object
    \begin{align*}
        \bbT\an_{X/ Y}  & \coloneqq (\bbL\an_{X/Y})^\vee \\
                        & \simeq \colim_{S \in \anNil^\cl_{X/ /Y}} (\bbL\an_{X/S})^\vee,
    \end{align*}
    in $\QCoh(X)$.
\end{defin}

The following result will play a major role in the study of the deformation to the normal bundle:

\begin{prop} \label{prop:conservativity_and_preservation_of_sifted_colimits_of_tangent_complex}
    The functor $\bbT\an_{X/ \bullet} \colon \anFMP_{X/} \to \QCoh(X)$ given on objects by the formula
        \[
            (X \to Y) \in \anFMP_{X/} \mapsto \bbT\an_{X/Y} \in \QCoh(X), 
        \]
    is conservative and commutes with sifted colimits.
\end{prop}

\begin{proof}
    We start by first proving that $\bbT\an_{X/ \bullet}$ is conservative. Let $Y \in \anFMP_{X/}$ and let $F \in \Perf(X)$ be a perfect complex on $X$
    (as an immediate consequence
    of derived Tate aciclycity one has that $\Perf(X) \simeq \Perf(A)$). In this case, we have a chain of natural equivalences
        \begin{align*}
            \Map_{\QCoh(X)}(F, \bbT\an_{X/Y}) & \simeq \colim_{S \in \anNil^\cl_{X/ /Y}} \Map_{\QCoh(X)}(F, \bbT\an_{X/S} ) \\
                                              & \simeq \colim_{S \in \anNil^\cl_{X/ /Y}} \Map_{\QCoh(X)}(F \otimes_{\cO_X} \bbL\an_{X/S}, \cO_X) \\
                                              & \simeq \colim_{S \in \anNil^\cl_{X/ /Y}} \Map_{\QCoh(X)}(\bbL\an_{X/S}, F^\vee) \\
                                              & \simeq \colim_{S \in \anNil^\cl_{X/ /Y}} \Map_{\Coh^+(X)}(\bbL\an_{X/S}, F^\vee) \\
                                              & \simeq \Map_{\pro(\Coh^+(X))}(\bbL\an_{X/ /Y}, F^\vee), 
        \end{align*}
    in the \infcat $\cS$. Let $f \colon Y \to Z$ be a morphism in the \infcat $\anFMP_{X/}$ such that $f$ induces an equivalence at the level of the relative
    analytic tangent complexes
        \[
            f_* \colon \bbT\an_{X/Y} \to \bbT\an_{X/Z}.  
        \]
    Then, from the previous considerations we deduce that for every $F \in \Perf(X)$ the induced morphism of mapping spaces
        \[
              \Map_{\pro(\Coh^+(X))}(\bbL\an_{X/ Y}, F) \simeq \Map_{\pro(\Coh^+(X))}(\bbL\an_{X/Z}, F)
        \]
    is an equivalence. Moreover, thanks to \cite[Corollary 5.5.6.22]{HTT} we deduce that for every $m \in \bbZ$ and every $F \in \Perf(X)$ we have a chain
    of natural equivalences
        \begin{align*}
            \Map_{\pro(\Coh^+(X))}(\bbL\an_{X/Y}, \tau_{\le m}(F)) & \simeq \colim_{S \in \anNil^\cl_{X/ /Y}} \Map_{\Coh^+(X)}(\bbL\an_{X/S}, \tau_{\le m}(F)) \\
                                                                   & \simeq \colim_{S \in \anNil^\cl_{X/ /Y}} \tau_{\le m} \Map_{\pro(\Coh^+(X))}(\bbL\an_{X/S}, F) \\
                                                                   & \simeq \tau_{\le m} \big( \colim_{S \in \anNil^\cl_{X/ /Y}} \Map_{\pro(\Coh^+(X))}(\bbL\an_{X/ S}, F) \big) \\
                                                                   & \simeq \tau_{\le m} \big( \Map_{\pro(\Coh^+(X))}(\bbL\an_{X/Y}, F) \big) \\
                                                                   & \simeq \tau_{\le m} \big( \Map_{\pro(\Coh^+(X))}(\bbL\an_{X/Z}, F) \big) \\
                                                                   & \simeq \Map_{\pro(\Coh^+(X))}(\bbL\an_{X/Z}, F),
        \end{align*}
    of mapping spaces. Since every bounded almost perfect $\cO_X$-module $\cF \in \Coh^+(X)$ is a retract of a truncation of a perfect $\cO_X$-module we deduce that
    one has an equivalence of mapping spaces
        \[
            \Map_{\pro(\Coh^+(X))}(\bbL\an_{X/ Y}, \cF) \simeq \Map_{\pro(\Coh^+(X))}(\bbL\an_{X/Z}, \cF),
        \]
    for every $\cF \in \Coh^\mathrm{b}(X)$. Conservativity of the functor $\bbT\an_{X/ \bullet}$ now follows immediately from \cref{cor:universal_property_of_relative_cotangent_complex_for_morphisms_between_analytic_FMP}
    combined with \cite[Corollary 5.40]{Porta_Yu_Representability} and the fact that both $Y, Z \in \anFMP_{X/}$ are nilcomplete.
    We shall now prove that $\bbT\an_{X/ \bullet}$ commutes with sifted colimits.
    Thanks to \cref{thm:Phi_is_an_equivalence} we are reduced to show that
        \[
            \bbT\an_{X/ X \times_\bullet X} \colon \anFMP_{X/} \to \Ind(\Coh^+(A)),  
        \]
    preserves sifted colimits. Moreover, given any $Y \in \anFMP_{X/}$, each projection morphism
        \[
            p_i \colon X \times_Y X \to X,  
        \]
    exhibits the latter as an object in $\anFMP_{/X}$. Since $X \in \dAfd_k$, \cite[Theorem 6.12]{Porta_Yu_NQK} implies that one has an equivalence of \infcats
        \[
            (-)\an \colon \mathrm{FMP}_{/\Spec A} \simeq \anFMP_{/X} .
        \]
    By \eqref{eq:fiber_sequence_for_rel_cot_complexes_along_diagonal_and_projections} we are reduced to show that
        \[
            \Delta^! \bbT_{X \times_{\bullet} X/ X}  ,  
        \]
    commutes with sifted colimits. functor $\bbT\an_{X/ \bullet}$ can be identified with the functor
        \[
            \bbT\an_{X/ \bullet}  
        \]
    But $\bbT\an_{X/ X \times_Y X}$ is defined as the Serre dual of
        \[
            \bbL\an_{X/ X\times_\bullet X } \simeq \bbL_{A/ A \otimes_\bullet A},  
        \]
    in $\Ind(\Coh^+(A))$. It follows from \cite[\S 5, Corollary 2.2.4]{Gaitsgory_Study_II} that the latter preserves sifted colimits. The assertion
    now follows from the fact that the analytification functor
        \[
            (-)\an \colon \Ind(\Coh^+(A)) \to \Ind(\Coh^+(A)),
        \]
    is a left adjoint, and thus commutes with sifted colimits.
\end{proof}




\section{Non-archimedean Deformation to the normal bundle}

In this \S \ we introduce the construction of the deformation to the normal cone in the setting of derived $k$-analytic geometry. This has
already been studied in \cite{Porta_Yu_NQK} in the case where we consider the natural morphism $f \colon X_\red \to X$, for some $X \in \dAnk$.
The situation in the algebraic case was extensively studied in \cite{Gaitsgory_Study_II}.

\subsection{General construction in the algebraic case} Consider the object
    \[
        \mathrm{B}^\bullet_{\mathrm{scaled}} \in \Fun(\Delta \op, \mathrm{dSt}_k),
    \]
introduced in
\cite[\S 9.2.2]{Gaitsgory_Study_II}. Recall that $\mathrm{B}^{n}_\mathrm{scaled}$ is obtained by gluing $n+1$ copies of $\bbA^1_k$ together along
$0 \in \bbA^1_k$.

\begin{construction} \label{construction:def_to_the_normal_bundle}
    Let $f \colon X \to Y$ denote a morphism in the \infcat $\mathrm{dSt}_k^\laft$.
    In \cite[\S 9.3]{Gaitsgory_Study_II}, the authors introduced the \emph{deformation to the normal bundle associated to the morphism $f \colon X \to Y$}, as the pullback
        \[
        \begin{tikzcd}
            \cD^\bullet_{X/Y} \ar{r} \ar{d} &   Y \times \bbA^1_k \ar{d} \\
            \Map_{/ \bbA^1_k}(\mathrm{B}^\bullet_{\mathrm{scaled}}, X \times \bbA^1_k) \ar{r} & \Map_{/ \bbA^1_k}(\mathrm{B}^\bullet_{\mathrm{scaled}}, Y \times \bbA^1_k),
        \end{tikzcd}
        \]
    where both $X \times \bbA^1_k$ and $Y \times \bbA^1_k$ are considered as constant simplicial objects in the \infcat $\dSt^\laft_k$. Suppose now that
    the morphism $f \colon X \to Y$ is a closed immersion in $\mathrm{dSt}^\laft_k$ such that both $X$ and $Y$ admit a \emph{deformation theory} in the sense of
    \cite[\S 1]{Gaitsgory_Study_II}.
    
    The authors proved in \cite[Theorem 9.2.3.4]{Gaitsgory_Study_II} that each component of the simplicial object
    $\cD_{X/Y}^\bullet$ admits a deformation theory itself. In particular, the object $\cD^\bullet_{X/ Y}$ defines a formal groupoid over
    $X \times \bbA^1_k$. Moreover, Theorem 5.2.3.4 in loc. cit. allows us to associate to $\cD^\bullet_{X/Y}$ a
    formal moduli problem under $X \times \bbA^1_k$
           \[
               \cD_{X/Y} \in \mathrm{FMP}_{X \times \bbA^1_k/ },
           \]
    obtained via the construction provided in \S 5.2.4 in loc. cit. Moreover, by construction the object $\cD_{X/Y} \in (\dSt^\laft_k)_{/\bbA^1_k}$.
\end{construction}

\begin{notation}
    Notice that the object $\cD_{X/ Y} \in \mathrm{FMP}_{X \times \bbA^1_k/ }$ admits a natural morphism to $Y \times \bbA^1_k$. We shall denote from now on the \infcat
    of formal moduli problems under $X \times \bbA^1_k$ together with a morphism to $Y \times \bbA^1_k$, in $\mathrm{dSt}_k^\laft$, by $\mathrm{FMP}_{X \times \bbA^1_k/ /Y \times \bbA^1_k}$.
\end{notation}
We shall now describe certain formal properties of the object $\cD_{X/Y} \in \mathrm{FMP}_{X \times \bbA^1_k/ /Y \times \bbA^1_k}$:

\begin{prop} Let $f \colon X \to Y$ denote a closed immersion in the \infcat $\dSt_k^\laft$.
    The following assertions hold:
    \begin{enumerate}
        \item The fiber $(\cD_{X/Y})_{0}$ at $0$ of the morphism structural morphism $\cD_{X/Y} \to \bbA^1_k$ identifies with
        the formal moduli problem $\rT_{X/Y}[-1]^\wedge$ obtained from the shifted tangent bundle $\rT_{X/Y}[-1] \to X$ by completing along the
        zero section
            \[
                s_0 \colon X \to \rT_{X/Y}[-1].  
            \]
        \item The fiber $(\cD_{X/Y})_{\lambda}$ for $\lambda \neq 0$ identifies with the formal completion
            \[
                Y^\wedge_X \in \dSt_k,  
            \]
        along the morphism $f$.
        \item There exists a natural sequence of formal moduli problems
            \[
                X \times \bbA^1_k = \cD_{X/Y}^{(0)} \to \cD_{X/Y}^{(1)}  \to  \dots \to \cD_{X/Y}^{(n)} \to \dots \to \dots \to Y,
            \]
        in $\mathrm{FMP}_{X\times\bbA^1_k/ /Y\times\bbA^1_k}$ such that, for each $n \ge 0$, the morphism
            \[
                \cD^{(n)}_{X/Y} \to \cD^{(n+1)}_{X/Y},  
            \]
        has the structure of a square-zero extension by an element in $\Coh^+(\cD_{X/Y}^{(n)})^{\ge n}$. In particular, if $X \in \dSt_k^\laft$
        is a geometric stack, then so it is each $\cD^{(n)}_{X/Y}$.
        \item The natural morphism
            \[\colim_{n} \cD^{(n)}_{X/Y} \to \cD_{X/Y},\]
        in the \infcat $\mathrm{FMP}_{X \times \bbA^1_k/ /Y\times\bbA^1_k}$ is an equivalence
        (and thus so it is when computed in the \infcat $(\dSt_k^\laft)_{X\times\bbA^1_k/ /Y\times\bbA^k})$. In particular, if $X \in \dSt_k^\laft$
        is a geometric stack then it follows by (iii) that the colimit 
            \[\cD_{X/Y} \simeq \colim_n \cD_{X/Y}^{(n)},\]
        is also geometric.
    \end{enumerate}
\end{prop}

\begin{rem}
    The existence of the object $\{ \cD^{(n)}_{X/ Y} \}_{n \ge 0}$ define a filtration on the global sections of the formal completion $Y^\wedge_X$, which we shall
    refer to as the \emph{Hodge filtration associated to the morphism $f$}.
\end{rem}

\begin{rem} \todo{Probably it is better to have this as a Corollary.}
As an immediate of the above Proposition we deduce that the relative algebraic de rham cohomology associated to the morphism $f \colon X \to Y$ can be computed
as the derived adic completion global section of the formal completion $Y^\wedge_X$. In particular, the latter does not depend on the underlying derived structure
on both $X$ and $Y$. Nonetheless, the \emph{Hodge filtration} considered depends heavily on the derived structure of hte latter objects in $\dSt^\laft_k$.
\end{rem}

\subsection{The construction of the deformation in the affinoid case} The goal in this section is to study the deformation to the normal cone in the non-archimedean
setting. Let us assume that we are given a closed immersion
    \[
        f \colon X \to Y,
    \]  
in the \infcat $\dAfd_k$. Consider the object
    \[
        \mathbf{B}^{\mathrm{an}, \bullet}_{\mathrm{scaled}} \colon \bDelta \op \to (\dAnSt_k)_{/ \mathbf A^1_k(0)},
    \]
obtained as the analytification of the simplicial object $\mathrm{B}^\bullet_{\mathrm{scaled}} \in \Fun(\Delta \op, \mathrm{dSt}^\laft_k)$ described in the previous section.
We define the deformation to the normal bundle via the pullback diagram
    \[
    \begin{tikzcd}
        \cD_{X/Y}^{\mathrm{an}, \bullet} \ar{r} \ar{d} & Y \times \bA^1_k \ar{d} \\
        \bfMap_{/ \bA^1_k}(\mathbf{B}^{\mathrm{an}, \bullet}, X \times \bA^1_k) \ar{r} & \bfMap_{/ \bA^1_k}(\mathbf{B}^{\mathrm{an, \bullet}}, Y \times \bA^1_k)
    \end{tikzcd},
    \]
computed in the \infcat $\mathrm{Fun}(\bDelta \op, \dAnSt_k)$.
Consider the natural projection map
    \[
        p \colon \cD_{X/Y}^{\mathrm{an}, \bullet} \to \bA^1_k. 
    \]
We now proceed to identifiy its fiber at $\lambda = 0$.

\begin{notation}
    Let $A \coloneqq \Gamma(Y, \cO_Y\alg)$ and $B \coloneqq \Gamma(X, \cO_X\alg)$ and define $Y\alg \coloneqq \Spec A$ and $X\alg \coloneqq \Spec B$.
\end{notation}

We have an induced closed immersion
    \[
        f\alg \colon X\alg \to Y\alg,  
    \]
of Noetherian affine schemes. Consider the object $\cD_{X\alg/ Y\alg}^\bullet \in \Fun(\bDelta \op, \dAff_k)$ as in \cref{construction:def_to_the_normal_bundle}.
Thanks to \cite[Proposition 4.6.1.3]{Lurie_SAG} we can find a presentation
    \[
        \lim_{\alpha \in A\op} f_{\alpha } \colon \lim_{\alpha \in A\op} X_\alpha\alg \to \lim_{\alpha \in A\op} Y_\alpha\alg,  
    \]
such that $A$ is a filtered \infcat and for each $\alpha \in A$, the morphism
    \[
        f_\alpha \colon X_\alpha\alg \to Y_\alpha\alg,  
    \]
is a closed immersion of almost of finite presentation affine schemes
    \[X_\alpha\alg , Y_\alpha\alg \in \dAff^\laft.\]
It is clear from the descritption
of $\cD_{X\alg/Y\alg}^\bullet$ given in \cref{construction:def_to_the_normal_bundle} that one has a natural equivalence of derived stacks
    \[
        \cD_{X\alg/Y\alg}^\bullet \simeq \lim_{\alpha \in A\op} \cD_{X_\alpha\alg/ Y_\alpha\alg}^\bullet,
    \]
in the \infcat $\mathrm{Fun}(\bDelta \op, \dSt_k)$.
Recall further the notion of \emph{relative analytification} introduced in \cite[\S 6.1]{Porta_Yu_NQK}. We shall first need a preliminary lemma:

\begin{lem} \label{lem:relative_anlaytification_of_closed_immersions_are_compatible_with_alg_construction}
    Let $f \colon X \to Y$ be a closed immersion in $\dAfd_k$. Then one has a natural equivalence
        \[
            (f \alg)\an_Y \simeq f,  
        \]
    in the \infcat $\Fun(\Delta^1, \dAfd_k)$.
\end{lem}

\begin{proof}
    Let $f \alg \colon \Spec B \to \Spec A$ be the induced morphism, where $B \coloneqq \Gamma(X, \cO_X\alg)$ and $A \coloneqq \Gamma(Y, \cO_Y\alg)$. It follows readily from the definitions
    that the morphism $f\alg$ of derived affine schemes is a closed immersion. Moreover,
    we have a natural
    identification
        \[(\Spec(A)\alg)\an_Y \simeq Y.\]
    For this reason, the commutative diagram
        \[
        \begin{tikzcd}
            X \ar{r}{f} \ar{d} & Y \ar{d}{\simeq} \\
            (X\alg)\an_Y \ar{r}{(f\alg)\an_Y} & (Y\alg)\an_Y,  
        \end{tikzcd}
        \]
    induces
    an equivalence at the level of global sections
        \[
            \Gamma(X, \cO_X\alg) \simeq \Gamma((X\alg)_Y\an, \cO_{(X\alg)\an_Y}\alg).  
        \]
    The result now follows immediately from the fact that both $X$ and $(X\alg)_Y\an$ are derived $k$-affinoid spaces together with the derived Tate acyclity's theorem,
    c.f. \cite[Theorem 3.1]{Porta_Yu_Derived_Hom_spaces}.
\end{proof}

\begin{lem} \label{lem:analytification_of_simplicial_object_deformation}
    Let notations be as above. Then one has a natural equivalence
        \[
            \cD_{X/Y}^{\mathrm{an}, \bullet} \simeq (\cD_{X\alg/ Y\alg}^\bullet)\an_Y  ,
        \]
    where $(-)_Y\an$ denotes the relative analytification with respect to $Y$.
\end{lem}

\begin{proof}
    As before, write the closed immersion of affine schemes
        \[
            f \colon X\alg \to Y\alg,  
        \]
    as a cofiltered limit 
        \[\lim_\alpha f_\alpha \colon \lim_\alpha X_\alpha\alg \to \lim_\alpha Y_\alpha\alg,\]
    where for each index $\alpha \in A$, both $X_\alpha$
    and $Y_\alpha$ are affine schemes of almost of finite type. Since the natural projection morphism
        \[
            \mathrm{B}_\mathrm{scaled}^\bullet \to \bbA^1_k, 
        \]
    where we regard $\bbA^1_k$ as a constant simplicial object in the \infcat $\dSt_k$, is a proper morphism, we deduce from \cite[Theorem 6.13]{Holstein_Analytification_of_mapping_stacks}
    that for every $\alpha$, the natural morphisms
        \begin{align*}
            \Map_{/\bbA^1_k}(\mathrm{B}^\bullet_\mathrm{scaled}, X_\alpha\alg \times \bbA^1_k)\an & \to \bfMap_{/ \bA^1_k}(\mathbf{B}^{\bullet, \mathrm{an}}_\mathrm{scaled} , (X_\alpha\alg)\an \times \bA^1_k) \\
            \Map_{/\bbA^1_k}(\mathrm{B}^\bullet_\mathrm{scaled}, Y_\alpha\alg \times \bbA^1_k)\an & \to \bfMap_{/ \bA^1_k}(\mathbf{B}^{\bullet, \mathrm{an}}_\mathrm{scaled} , (Y_\alpha\alg)\an \times \bA^1_k)
        \end{align*}
    are equivalences in the \infcat $\dAnSt_k$. Therefore, the natural morphism
        \[
            (\cD_{X\alg_\alpha/Y\alg_\alpha}^\bullet )\an  \to \cD_{(X_\alpha \alg)\an/ (Y_\alpha \alg)\an}^{\bullet, \mathrm{an}},
        \]
    is an equivalence in the \infcat $\dAnSt_k$. Observe further that for every $\alpha \in A$ the mapping stacks
        \[
            \Map_{/ \bbA^1_k}(\mathrm{B}^\bullet_\mathrm{scaled}, X_\alpha\alg \times \bbA^1_k) \quad \mathrm{and} \quad \Map_{/ \bbA^1_k}(\mathrm{B}^\bullet_\mathrm{scaled}, Y_\alpha\alg \times \bbA^1_k)  
        \]
    are affine schemes and therefore it follows by the construction of the analytification functor
        \[
            (-)\an \colon \dAff_k \to \dAnSt_k,  
        \]
    as a right Kan extension of the usual analytification functor
        \[
            (-)\an \colon \dAff^\laft_k \to \dAn_k,  
        \]
    that we have natural equivalences
        \begin{align*}
           ( \lim_\alpha \Map_{/\bbA^1_k}(\mathrm{B}^\bullet_\mathrm{scaled}, X_\alpha\alg \times \bbA^1_k)) \an & \simeq \bfMap_{/\bbA^1_k}(\mathrm{B}^{\bullet, \mathrm{an}}_\mathrm{scaled}, (X\alg)\an_Y \times \bA^1_k) \\
           ( \lim_\alpha \Map_{/\bbA^1_k}(\mathrm{B}^\bullet_\mathrm{scaled}, Y_\alpha\alg \times \bbA^1_k)) \an & \simeq \bfMap_{/\bbA^1_k}(\mathrm{B}^{\bullet, \mathrm{an}}_\mathrm{scaled}, (Y\alg)\an_Y \times \bA^1_k).
        \end{align*}
    in the \infcat $(\dAnSt_k)_{X \times \bbA^1/ /Y \times \bbA^1_k}$.
    The result now follows from the existence of a commutative cube
        \[
        \begin{tikzcd}[column sep=0.125in,row sep=0.125in]
            (\cD_{X\alg/ Y\alg}^\bullet)\an_Y \ar{rr} \ar{dd} \ar{rd} &  & Y \times \bA^1_k \ar{rd} \ar{dd} & \\
            & (\cD_{X\alg/ Y\alg}^\bullet)\an \ar{rr} \ar{dd} &  & (Y\alg \times \bbA^1_k)\an \ar{dd} \\
            \bfMap_{/\bA^1_k}(\mathbf{B}^\bullet, X \times \bA^1_k) \ar{rr} \ar{rd} & & \bfMap_{/\bA^1_k}(\mathbf{B}^\bullet, Y \times \bA^1_k) \ar{rd} \\
            & \bfMap_{/\bA^1_k}(\mathbf{B}^\bullet, (X\alg)\an \times \bA^1_k) \ar{rr} &  & \bfMap_{/\bA^1_k}(\mathbf{B}^\bullet, (Y\alg)\an \times \bA^1_k)
        \end{tikzcd}
        \]
    whose top and front squares are fiber products in $\dAnSt_k$, and thus so it is the back square, as desired.
\end{proof}

\begin{cor}
    Let $f \colon X \to Y$ be a closed morphism in $\dAfdk$. Then one has a natural identification
        \[
            (\cD^{\bullet, \mathrm{an}}_{X/Y})_0 \simeq (\bT^{\mathrm{an}, \bullet}_{X/Y})^\wedge[-1] ,
        \]
    where the latter denotes the commutative group object associated to the formal completion of the tangent bundle of $f$ along the zero section $s_0 \colon
    X \to \bT^{\mathrm{an}, \bullet}_{X/Y}[-1]$.
\end{cor}

\begin{proof}
    The result follows from \cite[Proposition 9.2.3.6]{Gaitsgory_Study_II} combined with the fact that relative analytification is defined via objects of almost of finite type
    plus the fact that analytification commutes with tangent bundles and formal completions, \cite[Corollary 5.20]{Holstein_Analytification_of_mapping_stacks}. \personal{I don't think the latter reference is completely
    fit.}
\end{proof}

The following result is an immediate consequence of \cref{lem:analytification_of_simplicial_object_deformation}:

\begin{lem} \label{lem:deformation_theory_for_D^an_bullet_X/Y}
    For each $[n] \in \bDelta$, the object $\cD_{X/Y}^{\mathrm{an}, [n]} \in (\dAnSt_k)_{X \times \bA^1_k/ /Y \times \bA^1_k}$ admits a deformation theory and it is furthermore affinoid.
\end{lem}

\begin{proof}
    We shall prove that $\cD^{ \mathrm{an}, \bullet}_{X/ Y} \in \Fun(\bDelta \op, \anFMP_{X \times \bA^1_k/ /Y\times\bA^1_k})$. By \cite[Lemma 2.3.2]{Gaitsgory_Study_II}, it follows
    that $\cD^\bullet_{X\alg/ /Y\alg}$ is an object in $\mathrm{FMP}_{X\alg/ /Y\alg}$. The result now follows by applying the relative analytification functor $(-)_Y\an$ together with
    \cite[Proposition 6.10]{Porta_Yu_NQK}. The fact that
    $\cD^{\mathrm{an}, [n]}_{X/Y}$ is affinoid, for every $[n] \in \bDelta \op$, follows readily from the observation that for every $\lambda \in \bbA^1_k$ the fiber
        \[
            (\cD^{\bullet, \mathrm{an}}_{X/ Y})_\lambda \in \dAfd_k.  
        \]
\end{proof}


\begin{construction}
    As we proved in \S 2.5, the \infcat $\anFMP_{X\times\bA^1_k/ /Y\times\bA^1_k}$ admits sifted colimits. Thanks to \cref{lem:deformation_theory_for_D^an_bullet_X/Y},
    it follows that we compute the sifted colimit
        \[
            \cD\an_{X/Y} \coloneqq \colim_{\bDelta \op} \cD^{\mathrm{an}, \bullet}_{X/Y} \in \anFMP_{X\times\bA^1_k/ /Y\times\bA^1_k}.  
        \]
    Similarly, we consider the sifted colimit
        \[
            \cD_{X\alg/ Y\alg} \coloneqq \colim_{\bDelta \op} \cD^\bullet_{X\alg/ Y\alg} \in \mathrm{FMP}_{X\alg \times \bbA^1_k/ /Y\alg \times \bbA^1_k} ,
        \]
    computed in the \infcat $\mathrm{FMP}_{X\alg\times\bbA^1_k/ /Y\alg\times\bbA^1_k}$. We can consider the latter as an object in
        \[
            \cD_{X\alg/ Y\alg} \in \dSt_{X\alg \times \bbA^1_k/ /Y\alg \times \bbA^1_k},  
        \]
    and therefore consider its relative analytification $(\cD_{X\alg/ Y\alg})\an_Y \in \dAnSt_{X\alg \times \bA^1_k/ /Y\alg \times \bA^1_k}$. Thanks to
    \cref{lem:analytification_of_simplicial_object_deformation} it follows that we have a natural morphism
        \[
            \cD\an_{X/ Y} \to (\cD_{X\alg/ Y\alg})\an_Y,  
        \]
    in the \infcat $\dAnSt_{X \times \bA^1_k/ /Y \times \bA^1_k}$.
\end{construction}

\begin{lem} \label{lem:equivalence_of_anFMP_X/ /Y_and_anFMP_X/ /X}
    Consider the base change functor
        \[
            (-)_X \colon \anFMP_{X/ /Y} \to \anFMP_{X/ /X},  
        \]
    given on objects by the formula
        \[
            F \in \anFMP_{X/ /Y} \mapsto F \times_Y X \in \anFMP_{X/ /X}.  
        \]
    Then the functor $(-)_X$ is an equivalence of \infcats.
\end{lem}

\begin{proof}
    The functor is clearly a left adjoint to the usual forgetful functor along $f$
        \[
            \anFMP_{X/ /X} \to \anFMP_{X/ /Y} .
        \]
    It suffices to prove that for every $F \in \anFMP_{X/ /Y}$ and $U \in \anNil_{X/}$ we have a natural equivalence of mapping spaces
        \[
            \Map_{X/ /Y}(U, F) \to \Map_{X/ /X}(U\times_Y X, F \times_Y X),  
        \]
    which is an immediate consequence of the universal property of fiber products.
\end{proof}

\begin{prop} \label{prop:rel_analytification_preserves_the_deformation}
    The natural morphism 
        \[
            \theta_{X/ Y}\colon\cD\an_{X/Y}\to(\cD_{X\alg/ Y\alg})\an_Y,
        \]
    is an equivalence of derived $k$-analytic stacks.
\end{prop}

\begin{proof}
    Consider the object $(\cD_{X\alg/ /Y\alg})\an_Y \in \anFMP_{X/ /Y}$. Under the equivalence of \infcats provided in \cref{lem:equivalence_of_anFMP_X/ /Y_and_anFMP_X/ /X},
    we observe that
        \[
            (\cD_{X\alg/ /Y\alg})\an_Y \times_Y X \simeq (\cD_{X\alg/ /Y\alg})\an_X ,
        \]
    where the latter denotes the relative analytification of $\cD_{X\alg/ /Y\alg}$ along the composite morphism
        \[
            \eta \colon X \times \bA^1_k \to Y \times \bA^1_k \to (Y\alg)\an \times \bA^1_k.  
        \]
    Thanks to \cref{lem:equivalence_of_anFMP_X/ /Y_and_anFMP_X/ /X} we are reduced to show that the natural morphism $\theta_{X/ Y}$ induces an equivalence
        \[
            \theta_{X/ Y} \times_Y X \colon (\cD_{X\alg/ Y\alg})\an_X \to \cD_{X/ Y}\an \times_Y X , 
        \]
    in the \infcat $\anFMP_{X \times \bA^1_k/ /X \times \bA^1_k} \simeq \mathrm{Ptd}(\anFMP_{/X\times\bA^1_k})$.
    Moreover, the equivalence of \infcats provided in \cite[Theorem 6.12]{Porta_Yu_NQK} implies that 
        \[
            (-)_X\an \colon \mathrm{FMP}_{/ X\alg} \to \anFMP_{/ X}  ,
        \]
    is an equivalence of \infcats. In particular, we have an induced equivalence of \infcats
        \[
            (-)_X\an \colon \mathrm{Ptd}(\mathrm{FMP}_{/X \alg\times\bA^1_k}) \to \mathrm{Ptd}(\anFMP_{/X\times\bA^1_k}).
        \]
    Since $\cD_{X\alg/ Y\alg}$ is computed as the sifted colimit of the object $\cD^\bullet_{X\alg/ Y\alg}$ in $\anFMP_{X\alg \times \bA^1_k/ Y\alg \times \bA^1_k}$
    and similarly $\cD\an_{X/ Y}$ is computed as the sifted colimit of $\cD^{\mathrm{an}, \bullet}_{X/ Y}$ in the \infcat $\anFMP_{X\times\bA^1_k/ /Y\times\bA^1_k}$.
    
    We deduce from \cref{lem:analytification_of_simplicial_object_deformation} that the natural morphism $\theta_{X/Y} \times_Y X$ is an equivalence in $\anFMP_{X \times \bA^1_k/ /X \times \bA^1_k}$,
    and therefore $\theta_{X/ Y}$ is an equivalence in the \infcat $\anFMP_{X\times\bA^1_k/ /Y\times\bA^1_k}$, as desired.
\end{proof}

\begin{rem}
    Consider the natural projection morphism
        \[
            q \colon \cD_{X/ Y}\an \to \bA^1_k.
        \]
    Its fiber at $\lambda \neq 0$ coincides with the formal completion
        \[
            Y^\wedge_X,  
        \]
    and its fiber at $0$ with the completion along the zero section $s_0 \colon X \to \bT\an_{X/ Y}[-1]$ of the shifted relative tangent bundle,
        \[
            \bT\an_{X/ Y} [-1]^{\wedge}.
        \]
\end{rem}


\begin{construction} \label{const:pullbacks_of_deformations}
    Let $g \colon U \to Y$ be a morphism in $\dAfd_k$. Consider the pullback diagram
        \[
        \begin{tikzcd}
            X_U \ar{r} \ar{d} & U \ar{d} \\
            X \ar{r}{f} & Y,  
        \end{tikzcd}
        \]
    computed in the \infcat $\dAfd_k$. It follows from the definitions that we have a natural pullback square of simplicial objects
        \[
        \begin{tikzcd}
            \cD_{X_U/ U}^{\mathrm{an}, \bullet} \ar{r} \ar{d} & \cD_{X/Y}^{\mathrm{an}, \bullet} \ar{d} \\
            U \ar{r} & Y,
        \end{tikzcd}
        \]
    in the \infcat $\dAnSt_k$. For this reason, we obtain a natural commutative diagram
        \[
        \begin{tikzcd}
            \cD_{X_U/ U}\an \ar{r} \ar{d} & \cD_{X/ Y} \an \ar{d} \\
            U \ar{r} & Y,
        \end{tikzcd}
        \]
    in the \infcat $\dAnSt_k$. Similarly, consider the pullback diagram
        \[
        \begin{tikzcd}
            X_U\alg \ar{r} \ar{d} & U \alg \ar{d} \\
            X\alg \ar{r}{f\alg} & Y\alg
        \end{tikzcd},
        \]
    in the \infcat $\dAff_k$. Reasoning as above, we have a natural commutative square
      \[
        \begin{tikzcd}
        \cD_{X_U\alg/ U\alg} \ar{r} \ar{d} & \cD_{X\alg/ Y\alg} \ar{d} \\
        U\alg \ar{r} & Y \alg,  
       \end{tikzcd}
      \]
    in the \infcat $\dSt_k$.
\end{construction}

\begin{prop} \label{prop:gluing_the_deformation}
    The commutative square
        \[
        \begin{tikzcd}
            \cD_{X_U/ U}\an \ar{r} \ar{d} & \cD_{X/ Y} \an \ar{d} \\
            U \ar{r} & Y, 
        \end{tikzcd}
        \]
    of \cref{const:pullbacks_of_deformations}, is a pullback square in the \infcat $\dAnSt_k$.
\end{prop}

\begin{proof}
    In order to show the assertion of the proposition, we are reduced to show that the natural morphism
        \[
            \cD_{X_U/ U}\an \to \cD\an_{X/ Y} \times_Y U,  
        \]
    is an equivalence in the \infcat $\anFMP_{X_U/ /U}$. Thanks to \cref{prop:conservativity_and_preservation_of_sifted_colimits_of_tangent_complex}, we are reduced to show
    that the induced morphism
        \[
            \bT\an_{X_U/ \cD_{X_U/ U}\an} \to \bT\an_{X_U/ \cD_{X/ Y} \times_Y U},
        \]
    is an equivalence in the \infcat $\QCoh(X_U)$. By the fact that the relative analytic tangent complex commutes with sifted colimits we are reduced to show that
    the natural morphism of simplicial objects
        \[
            \{ \bT\an_{X_U/ \cD_{X_U/ U}^{\mathrm{an}, \bullet}} \} \to \{ \bT\an_{X_U/ \cD_{X/ Y}^{\mathrm{an}, \bullet} \times_Y U} \},
        \]
    is an equivalence in $\Fun(\bDelta\op, \QCoh(X))$. We have a pullback square
        \[
        \begin{tikzcd}
            \cD_{X_U/ U}^{\mathrm{an}, \bullet} \ar{r} \ar{d} & \cD_{X/ Y}^{\mathrm{an}, \bullet} \ar{d} \\
            U \ar{r} & Y  
        \end{tikzcd}
        \]
    in $\mathrm{Fun}(\bDelta \op, \dAfd_k)$ and therefore, \cite[Proposition 5.12]{Porta_Yu_Representability} implies that we have necessarily that
        \[
            \{ \bbL\an_{X_U/ \cD_{X_U/ U}^{\mathrm{an}, \bullet}} \} \simeq \{ \bbL\an_{X_U /\cD_{X/ Y}^{\mathrm{an}, \bullet} \times_Y U} \},    
        \]
    in the \infcat $\QCoh(X_U)$, and the result follows.
\end{proof}





\subsection{Gluing the Deformation} In this \S, we globalize the results proved so far in \S 3.2. Let $f \colon X \to Y$ be a closed immersion
in the \infcat $\dAnSt_k$, where we assume $Y$ to be a geometric derived $k$-analytic stack.
Then we can consider as before the \emph{deformation to the normal bundle of the morphism $f$} constructed via the pullback diagram
    \[
    \begin{tikzcd}
        \cD^{\mathrm{an}, \bullet}_{X/ Y} \ar{r} \ar{d} & Y \times \bA^1_k \ar{d} \\
        \bfMap_{/ \bA^1_k} (\rB^\bullet_\mathrm{scaled}, X \times \bA^1_k) \ar{r} & \bfMap_{/ \bA^1_k}(\rB^\bullet_{\mathrm{scaled}}, Y \times \bA^1_k),
    \end{tikzcd}
    \]
in the \infcat $\dAnSt_k$. As in the previous \S, one can show that the simplicial object 
    \[
        \cD^{\mathrm{an}, \bullet}_{X/ Y} \in \anFMP_{X \times \bA^1_k/ /Y \times \bA^1_k},  
    \]
and in particular we have:

\begin{prop}
    The simplicial object
        \[
            \cD_{X/ Y}^{\mathrm{an}, \bullet} \colon \bDelta \op \to \anFMP_{X/ /Y},  
        \]
    admits a sifted colimit $\cD_{X/ Y}\an \in \anFMP_{X/ }$.
\end{prop}

\begin{proof} Let $U_\bullet \to Y$ be a derived $k$-affinoid admissible open covering of $Y$, in $\dAnSt_k$. We have a natural equivalence of \infcats
        \[
            \Psi \colon (\dAnSt_k)_{/ Y}  \to \lim_{\bDelta \op} (\dAnSt_k)_{/U_\bullet}.
        \]
    It is clear by the construction, that the simplicial object $\cD\an_{X/ Y}$ satisfies
        \begin{align*}
            \Psi(\cD^{\mathrm{an}, \bullet}_{X/ Y}) & \simeq \{ \cD^{\mathrm{an}, \bullet}_{X_{U_\bullet}/ U_\bullet} \} \\
                                                    & \in \lim_{\bDelta \op} (\dAnSt_k)_{/ U_\bullet}.
        \end{align*}
    \cref{prop:gluing_the_deformation} implies that we have a canonically defined object
        \[
            \cD\an_{X/ Y} \in \dAnSt_k,  
        \]
    obtained by gluing the object
        \[
            \{ \cD\an_{X_{U_\bullet}/ U_\bullet} \} \in \lim_{\bDelta \op} (\dAnSt_k)_{/ U_\bullet}.
        \]
    It is clear from the definition that $\cD\an_{X/ Y} \in \anFMP_{X/ /Y}$. We claim that the latter is a colimit of the diagram
        \[  
            \cD^{\mathrm{an}, \bullet}_{X/ Y}.
        \]
    We need to show that for every $Z \in \anFMP_{X/ /Y}$ together with a morphism
        \[
            \cD^{\mathrm{an}, \bullet}_{X/ Y} \to Z,   
        \]
    then there exists a uniquely defined (up to a contractible space of choices) morphism
        \[
            \cD\an_{X/ Y} \to Z,  
        \]
    in the \infcat $\anFMP_{X/ /Y}$. Moreover, we are reduced to check this property locally on $Y$, in which case the assertion follows immediately
    from the construction.
\end{proof}


\subsection{The Hodge filtration} In this \S, we will describe the construction of the Hodge filtration on the object $Y^\wedge_X$.


\begin{construction} \label{const:construction_of_Hodge_filtration_in_the_lafp_case}
    Let $f \colon X \to Y$ be a closed immersion in the \infcat $\dAfd_k$. Consider the induced morphism
        \[
            f \alg \colon X\alg \to Y\alg,   
        \]
    where one sets as usual $X \alg = \Spec A$ and $Y\alg = \Spec B$, where $A \coloneqq \Gamma(X, \cO_X\alg)$ and $B \coloneqq \Gamma(Y, \cO_Y \alg)$.
    Moreover, the morphism $f\alg$ is a closed immersion. Consider the induced map
        \[
            g \coloneqq f \alg \times \id_{\bbA^1_k} \colon X \times \bbA^1_k \to Y \times \bbA^1_k,  
        \]
    in $\dAff_k$. By Noetherian approximation, we can write $g$ as an inverse limit of the form
        \[
            \lim_{\alpha \in A \op}  g_\alpha \colon \lim_{\alpha \in A \op} X_\alpha \times \bbA^1_k \to \lim_{\alpha \in A \op} Y_\alpha \times \bbA^1_k,  
        \]
    where $A$ is a filtered \infcat and for each index $\alpha \in A$, we have that
        \[
            g_\alpha \colon X_\alpha \times \bbA^1_k \to Y_\alpha \times \bbA^1_k,  
        \]
    is a closed immersion in the \infcat $\dAff_k^\laft$. Fix some $\alpha \in A$. Thanks to \cite[Theorem 9.5.1.3]{Gaitsgory_Study_II},
    there exists a sequence of square-zero extensions of the form
        \[
            X_\alpha \times \bbA^1_k = X^{(0)} \hookrightarrow X_\alpha^{(1)} \hookrightarrow \dots \hookrightarrow X_\alpha^{(n)} \hookrightarrow \dots \to Y_\alpha \times \bbA^1_k,  
        \]
    defined inductively as follows: assume that for $n \ge 0$ we already have defined $X^{(0)}_\alpha \to X^{(n)}_\alpha$ together with a natural morphism
    to $Y_\alpha \times \bbA^1_k$. Then \cite[Theorem 9.5.1.3]{Gaitsgory_Study_II} implies that we a naturally defined derivation
        \[
            d^{(n)} \colon \bbL_{X_\alpha^{(n)}} \to \bbL_{X_\alpha^{(n)}/ Y_\alpha} \to \cF^{(n)},
        \]
    where $\cF^{(n)} \in \Coh^{\ge n}(X_\alpha^{(n)})$. We then define $X^{(n)}_\alpha \to X_\alpha^{(n+1)}$ as the square-zero extension associated to $d^{(n)}$.
    Moreover, as a consequence of the discussion followin \cite[\S 9.5.1]{Gaitsgory_Study_II}, it follows that
    $X^{(n+1)}$ admits a natural morphism to $Y \times \bbA^1_k$. We further observe that for each $\alpha \in A$ and $n \ge 0$, the object
    $X^{(n)}_\alpha$ is a derived affine scheme almost of finite presentation.
\end{construction}

\begin{rem} \label{rem:identification_of_equivalence_of_derivation_used_in_the_construction_of_the_Hodge_filtration}
    With notations as in \cref{const:construction_of_Hodge_filtration_in_the_lafp_case}, we are able to explicitly identify $\cF^{(n)} \in \Coh^+{\ge n}(X^{(n)})$ as
    follows: in \cite[\S 9.5.1]{Gaitsgory_Study_II} the authors construct the required derivation as the \emph{Serre dual} of a natural morphism of the form
        \[
            (i_{n-1}^*)^{\mathrm{IndCoh}}  \Sym^n(\bbT_{X/ Y}[-1])[1]) \to \bbT_{X^{(n)}/ Y} \to \bbT_{X^{(n)}},
        \]
    in the \infcat of \emph{ind-coherent sheaves on $X^{(n)}$}. Applying the Serre dual functor we obtain a natural morphism
        \[
            \bbL_{X^{(n)}} \to \bbL_{X^{(n)}/ Y} \to \bD^{\mathrm{Serre}}_{X^{(n)}} ( (i_{n-1}^*)^{\mathrm{IndCoh}}  \Sym^n(\bbT_{X/ Y}[-1])[1])) .
        \]
    Since the morphism $i_{n-1} \colon X \times \bbA^1_k \to X^{(n)}$ is a proper morphism, we are able to identify 
        \[
            \bD^{\mathrm{Serre}}_{X^{(n)}} ( (i_{n-1}^*)^{\mathrm{IndCoh}}  \Sym^n(\bbT_{X/ Y}[-1])[1])) \simeq i_{n-1, *} \Sym^n(\bbL_{X/ Y}[-1])[1]),
        \]
    in $\QCoh(X^{(n)})$, via \cite[Corollary 9.5.9 (b)]{Gaitsgory_IndCoh} combined with and \cite[\S 3.6.6]{Gaitsgory_IndCoh} and \cite[Corollary 1.4.4.2]{Gaitsgory_Study_II}.
\end{rem}

\begin{lem} \label{lem:naturality_of_Hodge_filtration_w.r.t_Noetherian_approximation}
    Let  $n \ge 0$, and let $\alpha \to \beta$ be a morphism in $A \op$. Then the transition morphism
        \[
            X_\alpha \times \bbA^1_k \to X_\beta \times \bbA^1_k,  
        \]
    lifts to a well defined induced morphism
        \[
            X_\alpha^{(n)} \to X_\beta^{(n)},
        \]
    in $\dAff_k^\laft$.
\end{lem}

\begin{proof}
    The result follows immediately from the naturality of the construction in \cite[\S 9.5.1]{Gaitsgory_Study_II}.
\end{proof}

\begin{prop}$\mathrm{(}$\cite[Corollary 9.5.2]{Gaitsgory_Study_II}$\mathrm{)}$ \label{prop:computation_of_filtered_colimit_of_the_X_alpha^n} Fix $\alpha \in A$. Then there exists a natural morphism
    \[\colim_{n \ge 0} X_\alpha^{(n)} \to \cD_{X_\alpha/ Y_\alpha},\]
    which is furthermore an equivalence in the \infcat $\mathrm{FMP}_{X_\alpha \times \bbA^1_k/ /Y_\alpha \times \bbA^1_k}$ (and thus in the \infcat $\dSt_k$).
\end{prop}

\begin{cor} \label{cor:commutation_between_limit_and_colimit_on_Hodge_filtration}
    The derived stack $\lim_{\alpha} \colim_{n \ge 0} X_\alpha^{(n)}$ is an affine scheme. Moreover, the natural morphism
        \[
            \gamma \colon \colim_{n \ge 0} X^{(n), \mathrm{alg}} \to \lim_{\alpha \in A \op} \colim_{n \ge 0} X^{(n)}_\alpha,
        \] 
    is an equivalence of derived affine schemes.
\end{cor}

\begin{proof}
    The first assertion follows immediately from \cref{lem:naturality_of_Hodge_filtration_w.r.t_Noetherian_approximation} together with the fact that
    for each $n \ge 0$, $X^{(n+1)}_\alpha$ is a square-zero extension on $X^{(n)}_\alpha$ via an object
        \begin{equation} \label{eq:n-th_part_of_the_Hodge_filtration_for_X_alpha}
            \cF^{(n+1)} \in \Coh^{\ge n}(X^{(n)}_\alpha).  
        \end{equation}
    We further deduce that
        \[
            \colim_{n \ge 0} X^{(n), \mathrm{alg}} \in \mathrm{dAff}_k.  
        \]
    Indeed, the result follows immediately from the fact that for each $X^{(n+1), \mathrm{alg}}$ is a square-zero extension of $X^{(n), \mathrm{alg}}$ by an object
        \begin{equation} \label{eq:n-th_part_of_the_Hodge_filtration_for_X}
            \cG^{(n+1)} \in \Coh^{\ge n}(X^{(n), \mathrm{alg}}).  
        \end{equation}
    Let $A, B \in \CAlg_k$ denote the derived rings of derived global sections of both
        \[
            \colim_{n \ge 0} X^{(n), \mathrm{alg}} \quad \mathrm{and} \quad \lim_{\alpha \in A \op} \colim_{n \ge 0} X^{(n)}_\alpha.
        \]
    Thanks to \eqref{eq:n-th_part_of_the_Hodge_filtration_for_X_alpha} and \eqref{eq:n-th_part_of_the_Hodge_filtration_for_X} we deduce that
        \[
            \pi_i(A) \simeq \pi_i(A^{(n)}) \quad \mathrm{and} \quad \pi_i(B) \simeq \pi_i(B^{(n)}),  
        \]
    for $i \ge 0$. Moreover, we have by construction that
        \[
            \pi_i(A^{(n)}) \simeq \colim_{\alpha \in A \op} \pi_i(A_\alpha^{(n)}) \quad \mathrm{and} \quad \pi_i(B) \simeq \pi_i(A^{(n)}_\alpha),  
        \]
    and the result follows immediately.
\end{proof}

\begin{defin}
    Consider the morphism $g \colon X\alg \times \bbA^1_k \to Y\alg \times \bbA^1_k$ above. \cref{lem:naturality_of_Hodge_filtration_w.r.t_Noetherian_approximation}
    implies that for each $n \ge 0$, the object
        \[
            X^{(n), \mathrm{alg}} \coloneqq \lim_{\alpha \in A\op} X_\alpha^{(n)} \in \dAff,  
        \]
    is well defined and it fits into a sequence of square-zero extensions
        \[
            X\times \bbA^1_k = X^{(0), \mathrm{alg}} \hookrightarrow X^{(1), \mathrm{alg}}  \hookrightarrow \dots \hookrightarrow X^{(n), \mathrm{alg}} \hookrightarrow \dots \to Y\alg,
        \]
    in the \infcat $\dAff_k$. We shall refer to this sequence as the \emph{Hodge filtration} associated to the morphism $f$. Moreover, thanks to
    \cref{prop:computation_of_filtered_colimit_of_the_X_alpha^n} it follows that we have a natural morphism
        \begin{equation} \label{eq:map_Hodge_filtration_to_deformation}
            \colim_{n \ge 0} X^{(n), \mathrm{alg}} \to \lim_{\alpha \in \op} \cD_{X_\alpha/ Y_\alpha},  
        \end{equation}
    in the \infcat $\mathrm{FMP}_{X \times \bbA^1_k/ /Y \times \bbA^1_k}$.
\end{defin}

We will need the following auxiliary result:

\begin{lem}
    The object $\lim_{\alpha \in A \op} \cD_{X_\alpha / Y_\alpha} \in \mathrm{FMP}_{X\alg \times \bbA^1_k/ /Y\alg \times \bbA^1_k}$ and moreover, the natural morphism
        \[
            \cD_{X\alg/ Y\alg} \to \lim_{\alpha \in A \op} \cD_{X_\alpha/ Y_\alpha},  
        \]
    is an equivalence in $\mathrm{FMP}_{X\alg \times \bbA^1_k/ /Y\alg \times \bbA^1_k}$.
\end{lem}

\begin{proof}  For each $\alpha \in A$ let us denote by
        \[
            h_\alpha \colon X \to X_\alpha,
        \]  
    the structural morphism. Thanks to \cite[Corollary 4.4.1.3]{Lurie_SAG} combined with \cite[Corollary  4.5.1.3]{Lurie_SAG} we conclude that the natural morphisms
        \[
            \theta_\alpha \colon h_\alpha^* \bbL_{X_\alpha} \to \bbL_{X} , \quad \mathrm{for each } \alpha \in A,
        \]
    assemble to provide an equivalence
        \[
            \theta \colon \colim_{\alpha \in A} h_\alpha^* \bbL_{X_\alpha} \to \bbL_X,   
        \]
    in the \infcat $\QCoh(X\alg)$. As a consequence, given any pushout diagram
        \[
        \begin{tikzcd}
            S \ar{r}{g_S} \ar{d} & S' \ar{d} \\
            T \ar{r} & T',
        \end{tikzcd}
        \]
    in the \infcat $\mathrm{Nil}_{X/ }$, where $g_S$ is a square-zero extension, we can write it as an inverse limit of pushout diagrams of the form
        \[
        \begin{tikzcd}
            S_\alpha \ar{r}{g_{S_\alpha}} \ar{d} & S'_\alpha \ar{d} \\
            T_\alpha \ar{r} & T'_\alpha,  
        \end{tikzcd}
        \]
    in $\mathrm{Nil}_{X_\alpha/ }$, for each $\alpha \in A$. It then follows that we have a chain of natural equivalences of the form
        \begin{align*}
            \lim_{\alpha \in A\op}\cD_{X_\alpha/ Y_\alpha} (T') & \simeq \lim_{\alpha \in A\op}\cD_{X_\alpha/ Y_\alpha}(\lim_{\alpha \in A\op} T_\alpha ') \\
                                                                & \simeq \lim_{\alpha \in A\op} \cD_{X_\alpha/ Y_\alpha}(T'_\alpha) \\
                                                                & \simeq \lim_{\alpha \in A\op}\cD_{X_\alpha/ Y_\alpha}(S'_\alpha \times_{S_\alpha} T_\alpha) \\
                                                                & \simeq \lim_{\alpha \in A\op} \cD_{X_\alpha/ Y_\alpha}(S'_\alpha) \times_{\underset{\alpha \in A\op}{\lim} \cD_{X_\alpha/ Y_\alpha}(S_\alpha)}  \lim_{\alpha \in A\op} \cD_{X_\alpha/ Y_\alpha}(T_\alpha),
        \end{align*}
    thus $\lim_{\alpha \in A\op} \cD_{X_\alpha/ Y_\alpha} \in \mathrm{FMP}_{X/ /Y}$, as desired. Consider now the canonical morphism
        \[\theta \colon \cD_{X\alg/ Y\alg} \to \lim_{\alpha \in A\op} \cD_{X_\alpha/ Y_\alpha},\]
    in the \infcat $\mathrm{FMP}_{X/ Y}$. Thanks to the analogue of \cref{prop:conservativity_of_relative_an_cot_complex} in the algebraic setting, it suffices to show that
    the morphism $\theta$ induces an equivalence at the level of cotangent complexes
        \begin{equation} \label{eq:cot_complex_of_limit_of_deformations_on_Noetherian_approximation}
            \bbL_{X \times \bbA^1_k / \lim_{\alpha \in A\op} \cD_{X_\alpha/ Y_\alpha}} \to \bbL_{X \times \bbA^1_k/ \cD_{X\alg/ Y\alg}},
        \end{equation}
    in the \infcat $\QCoh(X)$. Since both $\cD_{X/ Y}$ and $\cD_{X_\alpha/ Y_\alpha}$ for each $\alpha \in A$, are affine, it follows that we can identify both the left and right
    hand sides of \eqref{eq:cot_complex_of_limit_of_deformations_on_Noetherian_approximation} canonically with
        \begin{align*}
            \bbL_{X \times \bbA^1_k/ \lim_{\alpha \in A\op} \cD_{X_\alpha/ Y_\alpha}} & \simeq \colim_{\alpha \in A} \bbL_{X_\alpha \times \bbA^1_k/ \cD_{X_\alpha/ Y_\alpha}} \\
                                                                                      & \simeq  \colim_{\alpha \in A} \bbL_{X_\alpha \times \bbA^1_k/ Y_\alpha \times \bbA^1_k},
        \end{align*}
    and similarly we deduce that 
        \begin{align*}
            \bbL_{X \times \bbA^1_k/ \cD_{X/ Y}} & \simeq \bbL_{X \times \bbA^1_k/ Y \times \bbA^1_k}  \\
                                                 & \simeq \colim_{\alpha \in A} \bbL_{X_\alpha \times \bbA^1_k/ Y_\alpha \times \bbA^1_k},
        \end{align*}
    and the result follows.
\end{proof}

Thanks to the above Lemma, we can consider the natural morphism in \eqref{eq:map_Hodge_filtration_to_deformation} as
    \[
        \beta \colon \colim_{n \ge 0} X^{(n), \mathrm{alg}} \to \cD_{X\alg/ Y\alg}  ,
    \]
in the \infcat $\mathrm{FMP}_{X \times \bbA^1_k/ /Y\times \bbA^1_k}$. We have:

\begin{prop} \label{prop:colim_n_ge_0_X^n_to_deformation_is_an_equivalence}
    The morphism
        \[
            \beta \colon \colim_{n \ge 0} X^{(n), \mathrm{alg}} \to \cD_{X\alg/ Y\alg},  
        \]
    above is an equivalence in the \infcat $\mathrm{FMP}_{X \times \bbA^1_k/ /Y \times \bbA^1_k}$.
\end{prop}

\begin{proof}
    Thanks to \cref{prop:conservativity_of_relative_an_cot_complex}, it suffices to show that the natural morphism
        \[
            \bbL_{X/ \underset{n \ge 0}{\colim} X^{(n), \mathrm{alg}}} \to \bbL_{X/ \cD_{X\alg/ Y\alg}},  
        \]
    is an equivalence in $\QCoh(X)$. Moreover, \cref{cor:commutation_between_limit_and_colimit_on_Hodge_filtration} implies that we have a natural equivalence
        \begin{align*}
            \bbL_{X/ \underset{n \ge 0}{\colim} X^{(n), \mathrm{alg}}} & \simeq \colim_{\alpha \in A} \bbL_{X_\alpha/ \underset{n \ge 0}{\colim} X^{(n)_\alpha}} \\
                                                                       & \simeq \colim_{\alpha \in A} \bbL_{X_\alpha/ \cD_{X_\alpha/ Y_\alpha}} \\
                                                                       & \simeq \bbL_{X/ \cD_{X \alg/ Y \alg}},
        \end{align*}
    where the second equivalence follows from \cref{prop:computation_of_filtered_colimit_of_the_X_alpha^n}. The result is now an immediate consequence of
    \cref{prop:conservativity_of_relative_an_cot_complex}.
\end{proof}.


\begin{defin}
    Let $f \colon X \to Y$ be a closed immersion in the \infcat $\dAfd_k$. For each $n \ge 0$, we define the square-zero extension
        \[
            X \times \bA^1_k \hookrightarrow X^{(n)},  
        \]
    as the relative analytification, $(-) \an_Y$, of the natural square-zero extension
        \[
            X \alg \times \bbA^1_k \hookrightarrow X^{(n), \mathrm{alg}}.  
        \]
    By construction, for each $n \ge 0$, we have natural morphisms
        \[
            X^{(n)} \to Y \times \bA^1_k.  
        \]
\end{defin}

Putting together the above results we can easily deduce:

\begin{cor}
    There exists a natural morphism
        \[
            \colim_{n \ge 0} X^{(n)} \to \cD\an_{X/ Y},  
        \]
    which is furthermore an equivalence in the \infcat $\anFMP_{X \times \bA^1_k/ /Y \times \bA^1_k}$.
\end{cor}

\begin{proof}
    It is an immediate consequence of \cref{prop:colim_n_ge_0_X^n_to_deformation_is_an_equivalence} combined with \cref{prop:rel_analytification_preserves_the_deformation}.
\end{proof}

We now globalize the Hodge filtration on the deformation, $\cD\an_{X/ Y}$:

\begin{construction}
    Let $f \colon X \to Y$ be a closed immersion of derived $k$-analytic stacks, where $Y$ is assumed to be a geometric derived $k$-analytic stack. Suppose we are given a morphism
        \[
            U \to Y,  
        \]
    in $\dAnSt_k$, where $U \in \dAfd_k$ and form the pullback diagram
        \[
        \begin{tikzcd}
            X_U \ar{r} \ar{d} & U \ar{d} \\
            X \ar{r}{f} & Y  
        \end{tikzcd}
        \]
    in $\dAnSt_k$. Suppose we are now given a morphism of derived $k$-affinoid spaces $V \to U$, then we have a natural pullback diagram
        \[
        \begin{tikzcd}
            X_V \ar{r} \ar{d} & X_U \ar{d} \\
            V \ar{r} & U,  
        \end{tikzcd}
        \]
    in the \infcat $\dAfd_k$. This provides us for each $n \ge 0$, with natural commutative diagrams
        \begin{equation} \label{eq:gluing_diagram_of_Hodge_filtrations}
        \begin{tikzcd}
            X_V^{(n)} \ar{r} \ar{d} & X_U^{(n)} \ar{d} \\
            V \ar{r} & U,
        \end{tikzcd}
        \end{equation}
    of the $n$-th pieces of the Hodge filtrations on $\cD_{X_U/ U}$ and $\cD_{X_V/ V}$, respectively.
\end{construction}


\begin{lem} \label{lem:preservation_of_Hodge_filtration_under_pullback}
    The commutative square in \eqref{eq:gluing_diagram_of_Hodge_filtrations} is a pullback square in the \infcat $\dAfd_k$.
\end{lem}

\begin{proof} Thanks to \cref{lem:relative_anlaytification_of_closed_immersions_are_compatible_with_alg_construction}
    it suffices to prove that the induced diagram
        \[
        \begin{tikzcd}
            X_V\alg \ar{r} \ar{d} & X_U \alg \ar{d} \\
            V \alg \ar{r} & U\alg  
        \end{tikzcd},
        \]
    is a pullback square in $\dAff_k$. By a standard argument of Noetherian approximation, we might assume that $U' \coloneqq U\alg $ and $V' \coloneqq V\alg$ are
    both almost of finite presentation derived affined $k$-schemes. The result is now a direct consequence of \cref{rem:identification_of_equivalence_of_derivation_used_in_the_construction_of_the_Hodge_filtration}
    combined with \cite[Proposition 5.12]{Porta_Yu_Representability}.
\end{proof}

\begin{construction}
    Consider the relative analytification functor
        \begin{align*}
            (-)\an_Y \colon \lim_{U \in Y^{\mathrm{afd}}_{\textrm{\'et}}} (\dSt_k)_{/ U\alg \times \bbA^1_k} & \to \lim_{U \in Y^{\mathrm{afd}}_\textrm{\'et}} (\dAnSt_k)_{/ U \times \bA^1_k}  \\
                                                                                 & \simeq (\dAnSt_k)_{/ Y \times \bA^1_k}.
        \end{align*}
    Thanks to \cref{lem:preservation_of_Hodge_filtration_under_pullback}, for each $n \ge 0$ the object
        \[
            \{ X^{(n)}_{U} \}_{U \in Y^\mathrm{afd}_\textrm{\'et}}  \in \lim_{n \ge 0} (\dSt_k)_{/ U\alg}.
        \]
    Therefore, taking its relative analytification produces well defined objects $X^{(n)} \in (\dAnSt_k)_{Y \times \bA^1_k}$ which restricts to the usual
    Hodge filtration for every morphism
        \[
            g \colon U \to Y,  
        \]
    in $Y^\mathrm{afd}_\textrm{\'et}$. Moreover, by construction we have a natural morphism
        \[
            \colim_{n \ge 0} X^{(n)} \to \cD_{X/ Y} \an,  
        \]
    which is an equivalence in the \infcat $\dAnSt_k$.
\end{construction}











\bibliographystyle{plain}
\bibliography{dahema}

\end{document}